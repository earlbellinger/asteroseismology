\clearpage

\begin{figure}
   \centering
   \includegraphics[width=9.0cm]{KIC003744043_fwhm.png}
      \caption{Mode linewidths for KIC~3744043 as a function of the corresponding oscillation frequencies. \textit{Top panel}: linewidth measurements as defined by Eq.~(\ref{eq:resolved_profile}) for each angular degree ($\ell = 0$ blue squares, $\ell = 2$ green triangles, $\ell = 3$ yellow stars, and resolved $\ell = 1$ mixed modes red circles). Open symbols represent modes with detection probability under the suggested threshold (see Sect.~\ref{sec:test}). The 68\,\% credible intervals for the linewidths as derived by \diamonds\,\,are shown for each data point. The red solid line represents a polynomial fit to the linewidths of the $\ell = 1$ mixed modes, included to emphasize the trend with frequency. The shaded region represents the range $\numax \pm \sigma_\mathrm{env}$, with $\numax$ from Table~\ref{tab:bkg2} indicated by the dashed vertical line. \textit{Bottom panel}: the normalized fraction of resolved mixed modes with respect to unresolved ones, $f_\mathrm{res}$ (black dots), defined by Eq.~(\ref{eq:fraction_resolved}). The frequency position of each point is the average frequency of the resolved dipole mixed modes falling in each radial order (or that of the unresolved mixed modes if no resolved mixed modes are present). The horizontal dotted line represents the limit of resolved-dominated regime, as defined in Sect.~\ref{sec:fwhm}, while the horizontal dashed red line marks the average $f_\mathrm{res}$ given by Eq.~(\ref{eq:average_fraction}).}
    \label{fig:3744043fwhm}
\end{figure}

\begin{figure}
   \centering
   \includegraphics[width=9.0cm]{KIC003744043_amplitude.png}
      \caption{Mode amplitudes for KIC~3744043 as a function of the corresponding oscillation frequencies. Amplitude measurements as defined by Eq.~(\ref{eq:resolved_profile}) for each angular degree ($\ell = 0$ blue squares, $\ell = 2$ green triangles, $\ell = 3$ yellow stars, and resolved $\ell = 1$ mixed modes red circles). Open symbols represent modes with detection probability under the suggested threshold (see Sect.~\ref{sec:test}). The 68\,\% credible intervals for the amplitudes as derived by \diamonds\,\,are shown for each data point. The solid red line represents a polynomial fit to the amplitudes of the $\ell = 1$ mixed modes, included to emphasize the trend with frequency. The dashed vertical line indicates the $\numax$ value listed in Table~\ref{tab:bkg2}.}
    \label{fig:3744043amplitude}
\end{figure}
\clearpage



\begin{figure}
   \centering
   \includegraphics[width=9.0cm]{KIC006117517_fwhm.png}
      \caption{Mode linewidths for KIC~6117517 as a function of the corresponding oscillation frequencies. \textit{Top panel}: linewidth measurements as defined by Eq.~(\ref{eq:resolved_profile}) for each angular degree ($\ell = 0$ blue squares, $\ell = 2$ green triangles, $\ell = 3$ yellow stars, and resolved $\ell = 1$ mixed modes red circles). Open symbols represent modes with detection probability under the suggested threshold (see Sect.~\ref{sec:test}). The 68\,\% credible intervals for the linewidths as derived by \diamonds\,\,are shown for each data point. The red solid line represents a polynomial fit to the linewidths of the $\ell = 1$ mixed modes, included to emphasize the trend with frequency. The shaded region represents the range $\numax \pm \sigma_\mathrm{env}$, with $\numax$ from Table~\ref{tab:bkg2} indicated by the dashed vertical line. \textit{Bottom panel}: the normalized fraction of resolved mixed modes with respect to unresolved ones, $f_\mathrm{res}$ (black dots), defined by Eq.~(\ref{eq:fraction_resolved}). The frequency position of each point is the average frequency of the resolved dipole mixed modes falling in each radial order (or that of the unresolved mixed modes if no resolved mixed modes are present). The horizontal dotted line represents the limit of resolved-dominated regime, as defined in Sect.~\ref{sec:fwhm}, while the horizontal dashed red line marks the average $f_\mathrm{res}$ given by Eq.~(\ref{eq:average_fraction}).}
    \label{fig:6117517fwhm}
\end{figure}

\begin{figure}
   \centering
   \includegraphics[width=9.0cm]{KIC006117517_amplitude.png}
      \caption{Mode amplitudes for KIC~6117517 as a function of the corresponding oscillation frequencies. Amplitude measurements as defined by Eq.~(\ref{eq:resolved_profile}) for each angular degree ($\ell = 0$ blue squares, $\ell = 2$ green triangles, $\ell = 3$ yellow stars, and resolved $\ell = 1$ mixed modes red circles). Open symbols represent modes with detection probability under the suggested threshold (see Sect.~\ref{sec:test}). The 68\,\% credible intervals for the amplitudes as derived by \diamonds\,\,are shown for each data point. The solid red line represents a polynomial fit to the amplitudes of the $\ell = 1$ mixed modes, included to emphasize the trend with frequency. The dashed vertical line indicates the $\numax$ value listed in Table~\ref{tab:bkg2}.}
    \label{fig:6117517amplitude}
\end{figure}
\clearpage

\begin{figure}
   \centering
   \includegraphics[width=9.0cm]{KIC006144777_fwhm.png}
      \caption{Mode linewidths for KIC~6144777 as a function of the corresponding oscillation frequencies. \textit{Top panel}: linewidth measurements as defined by Eq.~(\ref{eq:resolved_profile}) for each angular degree ($\ell = 0$ blue squares, $\ell = 2$ green triangles, $\ell = 3$ yellow stars, and resolved $\ell = 1$ mixed modes red circles). Open symbols represent modes with detection probability under the suggested threshold (see Sect.~\ref{sec:test}). The 68\,\% credible intervals for the linewidths as derived by \diamonds\,\,are shown for each data point. The red solid line represents a polynomial fit to the linewidths of the $\ell = 1$ mixed modes, included to emphasize the trend with frequency. The shaded region represents the range $\numax \pm \sigma_\mathrm{env}$, with $\numax$ from Table~\ref{tab:bkg2} indicated by the dashed vertical line. \textit{Bottom panel}: the normalized fraction of resolved mixed modes with respect to unresolved ones, $f_\mathrm{res}$ (black dots), defined by Eq.~(\ref{eq:fraction_resolved}). The frequency position of each point is the average frequency of the resolved dipole mixed modes falling in each radial order (or that of the unresolved mixed modes if no resolved mixed modes are present). The horizontal dotted line represents the limit of resolved-dominated regime, as defined in Sect.~\ref{sec:fwhm}, while the horizontal dashed red line marks the average $f_\mathrm{res}$ given by Eq.~(\ref{eq:average_fraction}).}
    \label{fig:6144777fwhm}
\end{figure}

\begin{figure}
   \centering
   \includegraphics[width=9.0cm]{KIC006144777_amplitude.png}
      \caption{Mode amplitudes for KIC~6144777 as a function of the corresponding oscillation frequencies. Amplitude measurements as defined by Eq.~(\ref{eq:resolved_profile}) for each angular degree ($\ell = 0$ blue squares, $\ell = 2$ green triangles, $\ell = 3$ yellow stars, and resolved $\ell = 1$ mixed modes red circles). Open symbols represent modes with detection probability under the suggested threshold (see Sect.~\ref{sec:test}). The 68\,\% credible intervals for the amplitudes as derived by \diamonds\,\,are shown for each data point. The solid red line represents a polynomial fit to the amplitudes of the $\ell = 1$ mixed modes, included to emphasize the trend with frequency. The dashed vertical line indicates the $\numax$ value listed in Table~\ref{tab:bkg2}.}
    \label{fig:6144777amplitude}
\end{figure}
\clearpage


\begin{figure}
   \centering
   \includegraphics[width=9.0cm]{KIC007060732_fwhm.png}
      \caption{Mode linewidths for KIC~7060732 as a function of the corresponding oscillation frequencies. \textit{Top panel}: linewidth measurements as defined by Eq.~(\ref{eq:resolved_profile}) for each angular degree ($\ell = 0$ blue squares, $\ell = 2$ green triangles, $\ell = 3$ yellow stars, and resolved $\ell = 1$ mixed modes red circles). Open symbols represent modes with detection probability under the suggested threshold (see Sect.~\ref{sec:test}). The 68\,\% credible intervals for the linewidths as derived by \diamonds\,\,are shown for each data point. The red solid line represents a polynomial fit to the linewidths of the $\ell = 1$ mixed modes, included to emphasize the trend with frequency. The shaded region represents the range $\numax \pm \sigma_\mathrm{env}$, with $\numax$ from Table~\ref{tab:bkg2} indicated by the dashed vertical line. \textit{Bottom panel}: the normalized fraction of resolved mixed modes with respect to unresolved ones, $f_\mathrm{res}$ (black dots), defined by Eq.~(\ref{eq:fraction_resolved}). The frequency position of each point is the average frequency of the resolved dipole mixed modes falling in each radial order (or that of the unresolved mixed modes if no resolved mixed modes are present). The horizontal dotted line represents the limit of resolved-dominated regime, as defined in Sect.~\ref{sec:fwhm}, while the horizontal dashed red line marks the average $f_\mathrm{res}$ given by Eq.~(\ref{eq:average_fraction}).}
    \label{fig:7060732fwhm}
\end{figure}

\begin{figure}
   \centering
   \includegraphics[width=9.0cm]{KIC007060732_amplitude.png}
      \caption{Mode amplitudes for KIC~7060732 as a function of the corresponding oscillation frequencies. Amplitude measurements as defined by Eq.~(\ref{eq:resolved_profile}) for each angular degree ($\ell = 0$ blue squares, $\ell = 2$ green triangles, $\ell = 3$ yellow stars, and resolved $\ell = 1$ mixed modes red circles). Open symbols represent modes with detection probability under the suggested threshold (see Sect.~\ref{sec:test}). The 68\,\% credible intervals for the amplitudes as derived by \diamonds\,\,are shown for each data point. The solid red line represents a polynomial fit to the amplitudes of the $\ell = 1$ mixed modes, included to emphasize the trend with frequency. The dashed vertical line indicates the $\numax$ value listed in Table~\ref{tab:bkg2}.}
    \label{fig:7060732amplitude}
\end{figure}
\clearpage

\begin{figure}
   \centering
   \includegraphics[width=9.0cm]{KIC007619745_fwhm.png}
      \caption{Mode linewidths for KIC~7619745 as a function of the corresponding oscillation frequencies. \textit{Top panel}: linewidth measurements as defined by Eq.~(\ref{eq:resolved_profile}) for each angular degree ($\ell = 0$ blue squares, $\ell = 2$ green triangles, $\ell = 3$ yellow stars, and resolved $\ell = 1$ mixed modes red circles). Open symbols represent modes with detection probability under the suggested threshold (see Sect.~\ref{sec:test}). The 68\,\% credible intervals for the linewidths as derived by \diamonds\,\,are shown for each data point. The red solid line represents a polynomial fit to the linewidths of the $\ell = 1$ mixed modes, included to emphasize the trend with frequency. The shaded region represents the range $\numax \pm \sigma_\mathrm{env}$, with $\numax$ from Table~\ref{tab:bkg2} indicated by the dashed vertical line. \textit{Bottom panel}: the normalized fraction of resolved mixed modes with respect to unresolved ones, $f_\mathrm{res}$ (black dots), defined by Eq.~(\ref{eq:fraction_resolved}). The frequency position of each point is the average frequency of the resolved dipole mixed modes falling in each radial order (or that of the unresolved mixed modes if no resolved mixed modes are present). The horizontal dotted line represents the limit of resolved-dominated regime, as defined in Sect.~\ref{sec:fwhm}, while the horizontal dashed red line marks the average $f_\mathrm{res}$ given by Eq.~(\ref{eq:average_fraction}).}
    \label{fig:7619745fwhm}
\end{figure}


\begin{figure}
   \centering
   \includegraphics[width=9.0cm]{KIC007619745_amplitude.png}
      \caption{Mode amplitudes for KIC~7619745 as a function of the corresponding oscillation frequencies. Amplitude measurements as defined by Eq.~(\ref{eq:resolved_profile}) for each angular degree ($\ell = 0$ blue squares, $\ell = 2$ green triangles, $\ell = 3$ yellow stars, and resolved $\ell = 1$ mixed modes red circles). Open symbols represent modes with detection probability under the suggested threshold (see Sect.~\ref{sec:test}). The 68\,\% credible intervals for the amplitudes as derived by \diamonds\,\,are shown for each data point. The solid red line represents a polynomial fit to the amplitudes of the $\ell = 1$ mixed modes, included to emphasize the trend with frequency. The dashed vertical line indicates the $\numax$ value listed in Table~\ref{tab:bkg2}.}
    \label{fig:7619745amplitude}
\end{figure}
\clearpage



\begin{figure}
   \centering
   \includegraphics[width=9.0cm]{KIC008366239_fwhm.png}
      \caption{Mode linewidths for KIC~8366239 as a function of the corresponding oscillation frequencies. \textit{Top panel}: linewidth measurements as defined by Eq.~(\ref{eq:resolved_profile}) for each angular degree ($\ell = 0$ blue squares, $\ell = 2$ green triangles, $\ell = 3$ yellow stars, and resolved $\ell = 1$ mixed modes red circles). Open symbols represent modes with detection probability under the suggested threshold (see Sect.~\ref{sec:test}). The 68\,\% credible intervals for the linewidths as derived by \diamonds\,\,are shown for each data point. The red solid line represents a polynomial fit to the linewidths of the $\ell = 1$ mixed modes, included to emphasize the trend with frequency. The shaded region represents the range $\numax \pm \sigma_\mathrm{env}$, with $\numax$ from Table~\ref{tab:bkg2} indicated by the dashed vertical line. \textit{Bottom panel}: the normalized fraction of resolved mixed modes with respect to unresolved ones, $f_\mathrm{res}$ (black dots), defined by Eq.~(\ref{eq:fraction_resolved}). The frequency position of each point is the average frequency of the resolved dipole mixed modes falling in each radial order (or that of the unresolved mixed modes if no resolved mixed modes are present). The horizontal dotted line represents the limit of resolved-dominated regime, as defined in Sect.~\ref{sec:fwhm}, while the horizontal dashed red line marks the average $f_\mathrm{res}$ given by Eq.~(\ref{eq:average_fraction}).}
    \label{fig:8366239fwhm}
\end{figure}

\begin{figure}
   \centering
   \includegraphics[width=9.0cm]{KIC008366239_amplitude.png}
      \caption{Mode amplitudes for KIC~8366239 as a function of the corresponding oscillation frequencies. Amplitude measurements as defined by Eq.~(\ref{eq:resolved_profile}) for each angular degree ($\ell = 0$ blue squares, $\ell = 2$ green triangles, $\ell = 3$ yellow stars, and resolved $\ell = 1$ mixed modes red circles). Open symbols represent modes with detection probability under the suggested threshold (see Sect.~\ref{sec:test}). The 68\,\% credible intervals for the amplitudes as derived by \diamonds\,\,are shown for each data point. The solid red line represents a polynomial fit to the amplitudes of the $\ell = 1$ mixed modes, included to emphasize the trend with frequency. The dashed vertical line indicates the $\numax$ value listed in Table~\ref{tab:bkg2}.}
    \label{fig:8366239amplitude}
\end{figure}
\clearpage


\begin{figure}
   \centering
   \includegraphics[width=9.0cm]{KIC008475025_fwhm.png}
      \caption{Mode linewidths for KIC~8475025 as a function of the corresponding oscillation frequencies. \textit{Top panel}: linewidth measurements as defined by Eq.~(\ref{eq:resolved_profile}) for each angular degree ($\ell = 0$ blue squares, $\ell = 2$ green triangles, $\ell = 3$ yellow stars, and resolved $\ell = 1$ mixed modes red circles). Open symbols represent modes with detection probability under the suggested threshold (see Sect.~\ref{sec:test}). The 68\,\% credible intervals for the linewidths as derived by \diamonds\,\,are shown for each data point. The red solid line represents a polynomial fit to the linewidths of the $\ell = 1$ mixed modes, included to emphasize the trend with frequency. The shaded region represents the range $\numax \pm \sigma_\mathrm{env}$, with $\numax$ from Table~\ref{tab:bkg2} indicated by the dashed vertical line. \textit{Bottom panel}: the normalized fraction of resolved mixed modes with respect to unresolved ones, $f_\mathrm{res}$ (black dots), defined by Eq.~(\ref{eq:fraction_resolved}). The frequency position of each point is the average frequency of the resolved dipole mixed modes falling in each radial order (or that of the unresolved mixed modes if no resolved mixed modes are present). The horizontal dotted line represents the limit of resolved-dominated regime, as defined in Sect.~\ref{sec:fwhm}, while the horizontal dashed red line marks the average $f_\mathrm{res}$ given by Eq.~(\ref{eq:average_fraction}).}
    \label{fig:8475025fwhm}
\end{figure}

\begin{figure}
   \centering
   \includegraphics[width=9.0cm]{KIC008475025_amplitude.png}
      \caption{Mode amplitudes for KIC~8475025 as a function of the corresponding oscillation frequencies. Amplitude measurements as defined by Eq.~(\ref{eq:resolved_profile}) for each angular degree ($\ell = 0$ blue squares, $\ell = 2$ green triangles, $\ell = 3$ yellow stars, and resolved $\ell = 1$ mixed modes red circles). Open symbols represent modes with detection probability under the suggested threshold (see Sect.~\ref{sec:test}). The 68\,\% credible intervals for the amplitudes as derived by \diamonds\,\,are shown for each data point. The solid red line represents a polynomial fit to the amplitudes of the $\ell = 1$ mixed modes, included to emphasize the trend with frequency. The dashed vertical line indicates the $\numax$ value listed in Table~\ref{tab:bkg2}.}
    \label{fig:8475025amplitude}
\end{figure}
\clearpage

\begin{figure}
   \centering
   \includegraphics[width=9.0cm]{KIC008718745_fwhm.png}
      \caption{Mode linewidths for KIC~8718745 as a function of the corresponding oscillation frequencies. \textit{Top panel}: linewidth measurements as defined by Eq.~(\ref{eq:resolved_profile}) for each angular degree ($\ell = 0$ blue squares, $\ell = 2$ green triangles, $\ell = 3$ yellow stars, and resolved $\ell = 1$ mixed modes red circles). Open symbols represent modes with detection probability under the suggested threshold (see Sect.~\ref{sec:test}). The 68\,\% credible intervals for the linewidths as derived by \diamonds\,\,are shown for each data point. The red solid line represents a polynomial fit to the linewidths of the $\ell = 1$ mixed modes, included to emphasize the trend with frequency. The shaded region represents the range $\numax \pm \sigma_\mathrm{env}$, with $\numax$ from Table~\ref{tab:bkg2} indicated by the dashed vertical line. \textit{Bottom panel}: the normalized fraction of resolved mixed modes with respect to unresolved ones, $f_\mathrm{res}$ (black dots), defined by Eq.~(\ref{eq:fraction_resolved}). The frequency position of each point is the average frequency of the resolved dipole mixed modes falling in each radial order (or that of the unresolved mixed modes if no resolved mixed modes are present). The horizontal dotted line represents the limit of resolved-dominated regime, as defined in Sect.~\ref{sec:fwhm}, while the horizontal dashed red line marks the average $f_\mathrm{res}$ given by Eq.~(\ref{eq:average_fraction}).}
    \label{fig:8718745fwhm}
\end{figure}

\begin{figure}
   \centering
   \includegraphics[width=9.0cm]{KIC008718745_amplitude.png}
      \caption{Mode amplitudes for KIC~8718745 as a function of the corresponding oscillation frequencies. Amplitude measurements as defined by Eq.~(\ref{eq:resolved_profile}) for each angular degree ($\ell = 0$ blue squares, $\ell = 2$ green triangles, $\ell = 3$ yellow stars, and resolved $\ell = 1$ mixed modes red circles). Open symbols represent modes with detection probability under the suggested threshold (see Sect.~\ref{sec:test}). The 68\,\% credible intervals for the amplitudes as derived by \diamonds\,\,are shown for each data point. The solid red line represents a polynomial fit to the amplitudes of the $\ell = 1$ mixed modes, included to emphasize the trend with frequency. The dashed vertical line indicates the $\numax$ value listed in Table~\ref{tab:bkg2}.}
    \label{fig:8718745amplitude}
\end{figure}
\clearpage

\begin{figure}
   \centering
   \includegraphics[width=9.0cm]{KIC009145955_fwhm.png}
      \caption{Mode linewidths for KIC~9145955 as a function of the corresponding oscillation frequencies. \textit{Top panel}: linewidth measurements as defined by Eq.~(\ref{eq:resolved_profile}) for each angular degree ($\ell = 0$ blue squares, $\ell = 2$ green triangles, $\ell = 3$ yellow stars, and resolved $\ell = 1$ mixed modes red circles). Open symbols represent modes with detection probability under the suggested threshold (see Sect.~\ref{sec:test}). The 68\,\% credible intervals for the linewidths as derived by \diamonds\,\,are shown for each data point. The red solid line represents a polynomial fit to the linewidths of the $\ell = 1$ mixed modes, included to emphasize the trend with frequency. The shaded region represents the range $\numax \pm \sigma_\mathrm{env}$, with $\numax$ from Table~\ref{tab:bkg2} indicated by the dashed vertical line. \textit{Bottom panel}: the normalized fraction of resolved mixed modes with respect to unresolved ones, $f_\mathrm{res}$ (black dots), defined by Eq.~(\ref{eq:fraction_resolved}). The frequency position of each point is the average frequency of the resolved dipole mixed modes falling in each radial order (or that of the unresolved mixed modes if no resolved mixed modes are present). The horizontal dotted line represents the limit of resolved-dominated regime, as defined in Sect.~\ref{sec:fwhm}, while the horizontal dashed red line marks the average $f_\mathrm{res}$ given by Eq.~(\ref{eq:average_fraction}).}
    \label{fig:9145955fwhm}
\end{figure}

\begin{figure}
   \centering
   \includegraphics[width=9.0cm]{KIC009145955_amplitude.png}
      \caption{Mode amplitudes for KIC~9145955 as a function of the corresponding oscillation frequencies. Amplitude measurements as defined by Eq.~(\ref{eq:resolved_profile}) for each angular degree ($\ell = 0$ blue squares, $\ell = 2$ green triangles, $\ell = 3$ yellow stars, and resolved $\ell = 1$ mixed modes red circles). Open symbols represent modes with detection probability under the suggested threshold (see Sect.~\ref{sec:test}). The 68\,\% credible intervals for the amplitudes as derived by \diamonds\,\,are shown for each data point. The solid red line represents a polynomial fit to the amplitudes of the $\ell = 1$ mixed modes, included to emphasize the trend with frequency. The dashed vertical line indicates the $\numax$ value listed in Table~\ref{tab:bkg2}.}
    \label{fig:9145955amplitude}
\end{figure}
\clearpage


\begin{figure}
   \centering
   \includegraphics[width=9.0cm]{KIC009267654_fwhm.png}
      \caption{Mode linewidths for KIC~9267654 as a function of the corresponding oscillation frequencies. \textit{Top panel}: linewidth measurements as defined by Eq.~(\ref{eq:resolved_profile}) for each angular degree ($\ell = 0$ blue squares, $\ell = 2$ green triangles, $\ell = 3$ yellow stars, and resolved $\ell = 1$ mixed modes red circles). Open symbols represent modes with detection probability under the suggested threshold (see Sect.~\ref{sec:test}). The 68\,\% credible intervals for the linewidths as derived by \diamonds\,\,are shown for each data point. The red solid line represents a polynomial fit to the linewidths of the $\ell = 1$ mixed modes, included to emphasize the trend with frequency. The shaded region represents the range $\numax \pm \sigma_\mathrm{env}$, with $\numax$ from Table~\ref{tab:bkg2} indicated by the dashed vertical line. \textit{Bottom panel}: the normalized fraction of resolved mixed modes with respect to unresolved ones, $f_\mathrm{res}$ (black dots), defined by Eq.~(\ref{eq:fraction_resolved}). The frequency position of each point is the average frequency of the resolved dipole mixed modes falling in each radial order (or that of the unresolved mixed modes if no resolved mixed modes are present). The horizontal dotted line represents the limit of resolved-dominated regime, as defined in Sect.~\ref{sec:fwhm}, while the horizontal dashed red line marks the average $f_\mathrm{res}$ given by Eq.~(\ref{eq:average_fraction}).}
    \label{fig:9267654fwhm}
\end{figure}

\begin{figure}
   \centering
   \includegraphics[width=9.0cm]{KIC009267654_amplitude.png}
      \caption{Mode amplitudes for KIC~9267654 as a function of the corresponding oscillation frequencies. Amplitude measurements as defined by Eq.~(\ref{eq:resolved_profile}) for each angular degree ($\ell = 0$ blue squares, $\ell = 2$ green triangles, $\ell = 3$ yellow stars, and resolved $\ell = 1$ mixed modes red circles). Open symbols represent modes with detection probability under the suggested threshold (see Sect.~\ref{sec:test}). The 68\,\% credible intervals for the amplitudes as derived by \diamonds\,\,are shown for each data point. The solid red line represents a polynomial fit to the amplitudes of the $\ell = 1$ mixed modes, included to emphasize the trend with frequency. The dashed vertical line indicates the $\numax$ value listed in Table~\ref{tab:bkg2}.}
    \label{fig:9267654amplitude}
\end{figure}
\clearpage


\begin{figure}
   \centering
   \includegraphics[width=9.0cm]{KIC009475697_fwhm.png}
      \caption{Mode linewidths for KIC~9475697 as a function of the corresponding oscillation frequencies. \textit{Top panel}: linewidth measurements as defined by Eq.~(\ref{eq:resolved_profile}) for each angular degree ($\ell = 0$ blue squares, $\ell = 2$ green triangles, $\ell = 3$ yellow stars, and resolved $\ell = 1$ mixed modes red circles). Open symbols represent modes with detection probability under the suggested threshold (see Sect.~\ref{sec:test}). The 68\,\% credible intervals for the linewidths as derived by \diamonds\,\,are shown for each data point. The red solid line represents a polynomial fit to the linewidths of the $\ell = 1$ mixed modes, included to emphasize the trend with frequency. The shaded region represents the range $\numax \pm \sigma_\mathrm{env}$, with $\numax$ from Table~\ref{tab:bkg2} indicated by the dashed vertical line. \textit{Bottom panel}: the normalized fraction of resolved mixed modes with respect to unresolved ones, $f_\mathrm{res}$ (black dots), defined by Eq.~(\ref{eq:fraction_resolved}). The frequency position of each point is the average frequency of the resolved dipole mixed modes falling in each radial order (or that of the unresolved mixed modes if no resolved mixed modes are present). The horizontal dotted line represents the limit of resolved-dominated regime, as defined in Sect.~\ref{sec:fwhm}, while the horizontal dashed red line marks the average $f_\mathrm{res}$ given by Eq.~(\ref{eq:average_fraction}).}
    \label{fig:9475697fwhm}
\end{figure}

\begin{figure}
   \centering
   \includegraphics[width=9.0cm]{KIC009475697_amplitude.png}
      \caption{Mode amplitudes for KIC~9475697 as a function of the corresponding oscillation frequencies. Amplitude measurements as defined by Eq.~(\ref{eq:resolved_profile}) for each angular degree ($\ell = 0$ blue squares, $\ell = 2$ green triangles, $\ell = 3$ yellow stars, and resolved $\ell = 1$ mixed modes red circles). Open symbols represent modes with detection probability under the suggested threshold (see Sect.~\ref{sec:test}). The 68\,\% credible intervals for the amplitudes as derived by \diamonds\,\,are shown for each data point. The solid red line represents a polynomial fit to the amplitudes of the $\ell = 1$ mixed modes, included to emphasize the trend with frequency. The dashed vertical line indicates the $\numax$ value listed in Table~\ref{tab:bkg2}.}
    \label{fig:9475697amplitude}
\end{figure}
\clearpage


\begin{figure}
   \centering
   \includegraphics[width=9.0cm]{KIC009882316_fwhm.png}
      \caption{Mode linewidths for KIC~9882316 as a function of the corresponding oscillation frequencies. \textit{Top panel}: linewidth measurements as defined by Eq.~(\ref{eq:resolved_profile}) for each angular degree ($\ell = 0$ blue squares, $\ell = 2$ green triangles, $\ell = 3$ yellow stars, and resolved $\ell = 1$ mixed modes red circles). Open symbols represent modes with detection probability under the suggested threshold (see Sect.~\ref{sec:test}). The 68\,\% credible intervals for the linewidths as derived by \diamonds\,\,are shown for each data point. The red solid line represents a polynomial fit to the linewidths of the $\ell = 1$ mixed modes, included to emphasize the trend with frequency. The shaded region represents the range $\numax \pm \sigma_\mathrm{env}$, with $\numax$ from Table~\ref{tab:bkg2} indicated by the dashed vertical line. \textit{Bottom panel}: the normalized fraction of resolved mixed modes with respect to unresolved ones, $f_\mathrm{res}$ (black dots), defined by Eq.~(\ref{eq:fraction_resolved}). The frequency position of each point is the average frequency of the resolved dipole mixed modes falling in each radial order (or that of the unresolved mixed modes if no resolved mixed modes are present). The horizontal dotted line represents the limit of resolved-dominated regime, as defined in Sect.~\ref{sec:fwhm}, while the horizontal dashed red line marks the average $f_\mathrm{res}$ given by Eq.~(\ref{eq:average_fraction}).}
    \label{fig:9882316fwhm}
\end{figure}


\begin{figure}
   \centering
   \includegraphics[width=9.0cm]{KIC009882316_amplitude.png}
      \caption{Mode amplitudes for KIC~9882316 as a function of the corresponding oscillation frequencies. Amplitude measurements as defined by Eq.~(\ref{eq:resolved_profile}) for each angular degree ($\ell = 0$ blue squares, $\ell = 2$ green triangles, $\ell = 3$ yellow stars, and resolved $\ell = 1$ mixed modes red circles). Open symbols represent modes with detection probability under the suggested threshold (see Sect.~\ref{sec:test}). The 68\,\% credible intervals for the amplitudes as derived by \diamonds\,\,are shown for each data point. The solid red line represents a polynomial fit to the amplitudes of the $\ell = 1$ mixed modes, included to emphasize the trend with frequency. The dashed vertical line indicates the $\numax$ value listed in Table~\ref{tab:bkg2}.}
    \label{fig:9882316amplitude}
\end{figure}
\clearpage


\begin{figure}
   \centering
   \includegraphics[width=9.0cm]{KIC010123207_fwhm.png}
      \caption{Mode linewidths for KIC~10123207 as a function of the corresponding oscillation frequencies. \textit{Top panel}: linewidth measurements as defined by Eq.~(\ref{eq:resolved_profile}) for each angular degree ($\ell = 0$ blue squares, $\ell = 2$ green triangles, $\ell = 3$ yellow stars, and resolved $\ell = 1$ mixed modes red circles). Open symbols represent modes with detection probability under the suggested threshold (see Sect.~\ref{sec:test}). The 68\,\% credible intervals for the linewidths as derived by \diamonds\,\,are shown for each data point. The red solid line represents a polynomial fit to the linewidths of the $\ell = 1$ mixed modes, included to emphasize the trend with frequency. The shaded region represents the range $\numax \pm \sigma_\mathrm{env}$, with $\numax$ from Table~\ref{tab:bkg2} indicated by the dashed vertical line. \textit{Bottom panel}: the normalized fraction of resolved mixed modes with respect to unresolved ones, $f_\mathrm{res}$ (black dots), defined by Eq.~(\ref{eq:fraction_resolved}). The frequency position of each point is the average frequency of the resolved dipole mixed modes falling in each radial order (or that of the unresolved mixed modes if no resolved mixed modes are present). The horizontal dotted line represents the limit of resolved-dominated regime, as defined in Sect.~\ref{sec:fwhm}, while the horizontal dashed red line marks the average $f_\mathrm{res}$ given by Eq.~(\ref{eq:average_fraction}).}
    \label{fig:10123207fwhm}
\end{figure}


\begin{figure}
   \centering
   \includegraphics[width=9.0cm]{KIC010123207_amplitude.png}
      \caption{Mode amplitudes for KIC~10123207 as a function of the corresponding oscillation frequencies. Amplitude measurements as defined by Eq.~(\ref{eq:resolved_profile}) for each angular degree ($\ell = 0$ blue squares, $\ell = 2$ green triangles, $\ell = 3$ yellow stars, and resolved $\ell = 1$ mixed modes red circles). Open symbols represent modes with detection probability under the suggested threshold (see Sect.~\ref{sec:test}). The 68\,\% credible intervals for the amplitudes as derived by \diamonds\,\,are shown for each data point. The solid red line represents a polynomial fit to the amplitudes of the $\ell = 1$ mixed modes, included to emphasize the trend with frequency. The dashed vertical line indicates the $\numax$ value listed in Table~\ref{tab:bkg2}.}
    \label{fig:10123207amplitude}
\end{figure}
\clearpage


\begin{figure}
   \centering
   \includegraphics[width=9.0cm]{KIC010200377_fwhm.png}
      \caption{Mode linewidths for KIC~10200377 as a function of the corresponding oscillation frequencies. \textit{Top panel}: linewidth measurements as defined by Eq.~(\ref{eq:resolved_profile}) for each angular degree ($\ell = 0$ blue squares, $\ell = 2$ green triangles, $\ell = 3$ yellow stars, and resolved $\ell = 1$ mixed modes red circles). Open symbols represent modes with detection probability under the suggested threshold (see Sect.~\ref{sec:test}). The 68\,\% credible intervals for the linewidths as derived by \diamonds\,\,are shown for each data point. The red solid line represents a polynomial fit to the linewidths of the $\ell = 1$ mixed modes, included to emphasize the trend with frequency. The shaded region represents the range $\numax \pm \sigma_\mathrm{env}$, with $\numax$ from Table~\ref{tab:bkg2} indicated by the dashed vertical line. \textit{Bottom panel}: the normalized fraction of resolved mixed modes with respect to unresolved ones, $f_\mathrm{res}$ (black dots), defined by Eq.~(\ref{eq:fraction_resolved}). The frequency position of each point is the average frequency of the resolved dipole mixed modes falling in each radial order (or that of the unresolved mixed modes if no resolved mixed modes are present). The horizontal dotted line represents the limit of resolved-dominated regime, as defined in Sect.~\ref{sec:fwhm}, while the horizontal dashed red line marks the average $f_\mathrm{res}$ given by Eq.~(\ref{eq:average_fraction}).}
    \label{fig:10200377fwhm}
\end{figure}


\begin{figure}
   \centering
   \includegraphics[width=9.0cm]{KIC010200377_amplitude.png}
      \caption{Mode amplitudes for KIC~10200377 as a function of the corresponding oscillation frequencies. Amplitude measurements as defined by Eq.~(\ref{eq:resolved_profile}) for each angular degree ($\ell = 0$ blue squares, $\ell = 2$ green triangles, $\ell = 3$ yellow stars, and resolved $\ell = 1$ mixed modes red circles). Open symbols represent modes with detection probability under the suggested threshold (see Sect.~\ref{sec:test}). The 68\,\% credible intervals for the amplitudes as derived by \diamonds\,\,are shown for each data point. The solid red line represents a polynomial fit to the amplitudes of the $\ell = 1$ mixed modes, included to emphasize the trend with frequency. The dashed vertical line indicates the $\numax$ value listed in Table~\ref{tab:bkg2}.}
    \label{fig:10200377amplitude}
\end{figure}
\clearpage


\begin{figure}
   \centering
   \includegraphics[width=9.0cm]{KIC010257278_fwhm.png}
      \caption{Mode linewidths for KIC~10257278 as a function of the corresponding oscillation frequencies. \textit{Top panel}: linewidth measurements as defined by Eq.~(\ref{eq:resolved_profile}) for each angular degree ($\ell = 0$ blue squares, $\ell = 2$ green triangles, $\ell = 3$ yellow stars, and resolved $\ell = 1$ mixed modes red circles). Open symbols represent modes with detection probability under the suggested threshold (see Sect.~\ref{sec:test}). The 68\,\% credible intervals for the linewidths as derived by \diamonds\,\,are shown for each data point. The red solid line represents a polynomial fit to the linewidths of the $\ell = 1$ mixed modes, included to emphasize the trend with frequency. The shaded region represents the range $\numax \pm \sigma_\mathrm{env}$, with $\numax$ from Table~\ref{tab:bkg2} indicated by the dashed vertical line. \textit{Bottom panel}: the normalized fraction of resolved mixed modes with respect to unresolved ones, $f_\mathrm{res}$ (black dots), defined by Eq.~(\ref{eq:fraction_resolved}). The frequency position of each point is the average frequency of the resolved dipole mixed modes falling in each radial order (or that of the unresolved mixed modes if no resolved mixed modes are present). The horizontal dotted line represents the limit of resolved-dominated regime, as defined in Sect.~\ref{sec:fwhm}, while the horizontal dashed red line marks the average $f_\mathrm{res}$ given by Eq.~(\ref{eq:average_fraction}).}
    \label{fig:10257278fwhm}
\end{figure}


\begin{figure}
   \centering
   \includegraphics[width=9.0cm]{KIC010257278_amplitude.png}
      \caption{Mode amplitudes for KIC~10257278 as a function of the corresponding oscillation frequencies. Amplitude measurements as defined by Eq.~(\ref{eq:resolved_profile}) for each angular degree ($\ell = 0$ blue squares, $\ell = 2$ green triangles, $\ell = 3$ yellow stars, and resolved $\ell = 1$ mixed modes red circles). Open symbols represent modes with detection probability under the suggested threshold (see Sect.~\ref{sec:test}). The 68\,\% credible intervals for the amplitudes as derived by \diamonds\,\,are shown for each data point. The solid red line represents a polynomial fit to the amplitudes of the $\ell = 1$ mixed modes, included to emphasize the trend with frequency. The dashed vertical line indicates the $\numax$ value listed in Table~\ref{tab:bkg2}.}
    \label{fig:10257278amplitude}
\end{figure}
\clearpage


\begin{figure}
   \centering
   \includegraphics[width=9.0cm]{KIC011353313_fwhm.png}
      \caption{Mode linewidths for KIC~11353313 as a function of the corresponding oscillation frequencies. \textit{Top panel}: linewidth measurements as defined by Eq.~(\ref{eq:resolved_profile}) for each angular degree ($\ell = 0$ blue squares, $\ell = 2$ green triangles, $\ell = 3$ yellow stars, and resolved $\ell = 1$ mixed modes red circles). Open symbols represent modes with detection probability under the suggested threshold (see Sect.~\ref{sec:test}). The 68\,\% credible intervals for the linewidths as derived by \diamonds\,\,are shown for each data point. The red solid line represents a polynomial fit to the linewidths of the $\ell = 1$ mixed modes, included to emphasize the trend with frequency. The shaded region represents the range $\numax \pm \sigma_\mathrm{env}$, with $\numax$ from Table~\ref{tab:bkg2} indicated by the dashed vertical line. \textit{Bottom panel}: the normalized fraction of resolved mixed modes with respect to unresolved ones, $f_\mathrm{res}$ (black dots), defined by Eq.~(\ref{eq:fraction_resolved}). The frequency position of each point is the average frequency of the resolved dipole mixed modes falling in each radial order (or that of the unresolved mixed modes if no resolved mixed modes are present). The horizontal dotted line represents the limit of resolved-dominated regime, as defined in Sect.~\ref{sec:fwhm}, while the horizontal dashed red line marks the average $f_\mathrm{res}$ given by Eq.~(\ref{eq:average_fraction}).}
    \label{fig:11353313fwhm}
\end{figure}


\begin{figure}
   \centering
   \includegraphics[width=9.0cm]{KIC011353313_amplitude.png}
      \caption{Mode amplitudes for KIC~11353313 as a function of the corresponding oscillation frequencies. Amplitude measurements as defined by Eq.~(\ref{eq:resolved_profile}) for each angular degree ($\ell = 0$ blue squares, $\ell = 2$ green triangles, $\ell = 3$ yellow stars, and resolved $\ell = 1$ mixed modes red circles). Open symbols represent modes with detection probability under the suggested threshold (see Sect.~\ref{sec:test}). The 68\,\% credible intervals for the amplitudes as derived by \diamonds\,\,are shown for each data point. The solid red line represents a polynomial fit to the amplitudes of the $\ell = 1$ mixed modes, included to emphasize the trend with frequency. The dashed vertical line indicates the $\numax$ value listed in Table~\ref{tab:bkg2}.}
    \label{fig:11353313amplitude}
\end{figure}
\clearpage


\begin{figure}
   \centering
   \includegraphics[width=9.0cm]{KIC011913545_fwhm.png}
      \caption{Mode linewidths for KIC~11913545 as a function of the corresponding oscillation frequencies. \textit{Top panel}: linewidth measurements as defined by Eq.~(\ref{eq:resolved_profile}) for each angular degree ($\ell = 0$ blue squares, $\ell = 2$ green triangles, $\ell = 3$ yellow stars, and resolved $\ell = 1$ mixed modes red circles). Open symbols represent modes with detection probability under the suggested threshold (see Sect.~\ref{sec:test}). The 68\,\% credible intervals for the linewidths as derived by \diamonds\,\,are shown for each data point. The red solid line represents a polynomial fit to the linewidths of the $\ell = 1$ mixed modes, included to emphasize the trend with frequency. The shaded region represents the range $\numax \pm \sigma_\mathrm{env}$, with $\numax$ from Table~\ref{tab:bkg2} indicated by the dashed vertical line. \textit{Bottom panel}: the normalized fraction of resolved mixed modes with respect to unresolved ones, $f_\mathrm{res}$ (black dots), defined by Eq.~(\ref{eq:fraction_resolved}). The frequency position of each point is the average frequency of the resolved dipole mixed modes falling in each radial order (or that of the unresolved mixed modes if no resolved mixed modes are present). The horizontal dotted line represents the limit of resolved-dominated regime, as defined in Sect.~\ref{sec:fwhm}, while the horizontal dashed red line marks the average $f_\mathrm{res}$ given by Eq.~(\ref{eq:average_fraction}).}
    \label{fig:11913545fwhm}
\end{figure}

\begin{figure}
   \centering
   \includegraphics[width=9.0cm]{KIC011913545_amplitude.png}
      \caption{Mode amplitudes for KIC~11913545 as a function of the corresponding oscillation frequencies. Amplitude measurements as defined by Eq.~(\ref{eq:resolved_profile}) for each angular degree ($\ell = 0$ blue squares, $\ell = 2$ green triangles, $\ell = 3$ yellow stars, and resolved $\ell = 1$ mixed modes red circles). Open symbols represent modes with detection probability under the suggested threshold (see Sect.~\ref{sec:test}). The 68\,\% credible intervals for the amplitudes as derived by \diamonds\,\,are shown for each data point. The solid red line represents a polynomial fit to the amplitudes of the $\ell = 1$ mixed modes, included to emphasize the trend with frequency. The dashed vertical line indicates the $\numax$ value listed in Table~\ref{tab:bkg2}.}
    \label{fig:11913545amplitude}
\end{figure}
\clearpage


\begin{figure}
   \centering
   \includegraphics[width=9.0cm]{KIC011968334_fwhm.png}
      \caption{Mode linewidths for KIC~11968334 as a function of the corresponding oscillation frequencies. \textit{Top panel}: linewidth measurements as defined by Eq.~(\ref{eq:resolved_profile}) for each angular degree ($\ell = 0$ blue squares, $\ell = 2$ green triangles, $\ell = 3$ yellow stars, and resolved $\ell = 1$ mixed modes red circles). Open symbols represent modes with detection probability under the suggested threshold (see Sect.~\ref{sec:test}). The 68\,\% credible intervals for the linewidths as derived by \diamonds\,\,are shown for each data point. The red solid line represents a polynomial fit to the linewidths of the $\ell = 1$ mixed modes, included to emphasize the trend with frequency. The shaded region represents the range $\numax \pm \sigma_\mathrm{env}$, with $\numax$ from Table~\ref{tab:bkg2} indicated by the dashed vertical line. \textit{Bottom panel}: the normalized fraction of resolved mixed modes with respect to unresolved ones, $f_\mathrm{res}$ (black dots), defined by Eq.~(\ref{eq:fraction_resolved}). The frequency position of each point is the average frequency of the resolved dipole mixed modes falling in each radial order (or that of the unresolved mixed modes if no resolved mixed modes are present). The horizontal dotted line represents the limit of resolved-dominated regime, as defined in Sect.~\ref{sec:fwhm}, while the horizontal dashed red line marks the average $f_\mathrm{res}$ given by Eq.~(\ref{eq:average_fraction}).}
    \label{fig:11968334fwhm}
\end{figure}

\begin{figure}
   \centering
   \includegraphics[width=9.0cm]{KIC011968334_amplitude.png}
      \caption{Mode amplitudes for KIC~11968334 as a function of the corresponding oscillation frequencies. Amplitude measurements as defined by Eq.~(\ref{eq:resolved_profile}) for each angular degree ($\ell = 0$ blue squares, $\ell = 2$ green triangles, $\ell = 3$ yellow stars, and resolved $\ell = 1$ mixed modes red circles). Open symbols represent modes with detection probability under the suggested threshold (see Sect.~\ref{sec:test}). The 68\,\% credible intervals for the amplitudes as derived by \diamonds\,\,are shown for each data point. The solid red line represents a polynomial fit to the amplitudes of the $\ell = 1$ mixed modes, included to emphasize the trend with frequency. The dashed vertical line indicates the $\numax$ value listed in Table~\ref{tab:bkg2}.}
    \label{fig:11968334amplitude}
\end{figure}
\clearpage