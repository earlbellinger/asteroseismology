%% This is emulateapj reformatting of the AASTEX sample document
%%
%\documentclass[iop,apj,twocolappendix]{emulateapj}
\documentclass[manuscript]{aastex}

\newcommand{\vdag}{(v)^\dagger}
\newcommand{\myemail}{bellinger@mps.mpg.de}

\usepackage{subfig}
\usepackage{graphicx}	% Including figure files
\usepackage{amsmath}	% Advanced maths commands
\usepackage{amssymb}	% Extra maths symbols
\usepackage{bm}		% Bold maths symbols, including upright Greek

\usepackage{mathrsfs}
%\usepackage{savesym}
%\savesymbol{tablenum}
\usepackage{siunitx}
%\restoresymbol{SIX}{tablenum}

\usepackage{bookmark}

%\usepackage{todonotes}
\usepackage{physics}
\usepackage{natbib}

%\usepackage[none]{hyphenat}
\usepackage{tikz}
\usetikzlibrary{arrows,positioning,shapes,decorations.markings}

\newcommand{\Dnu}{\Delta\nu}
\newcommand{\dnu}{\delta\nu}
\newcommand{\Mo}{\rm{M}_\odot}
\newcommand{\Lo}{\rm{L}_\odot}
\newcommand{\Ro}{\rm{R}_\odot}

%André
%Biggie
%Coolio
%Dre
%Eazy-E
%Funkmaster
%Gza
%Havoc
%Ice Cube
%J Dilla
%Kanye
%Lil' Kim

%% You can insert a short comment on the title page using the command below.

\slugcomment{}

\shorttitle{Stellar parameters in an instant with machine learning}
\shortauthors{Bellinger \& Angelou et al.}

\begin{document}

\title{Fundamental Parameters of Main-Sequence Stars in an Instant with Machine Learning}

\author{Earl P. Bellinger\altaffilmark{1,2,3}, George C. Angelou\altaffilmark{1,2}, Saskia Hekker\altaffilmark{1,2}, Sarbani Basu\altaffilmark{4}, Warrick H. Ball\altaffilmark{5}, and Elisabeth Guggenberger\altaffilmark{1,2}}
\affil{\altaffilmark{1} Max-Planck-Institut f\"{u}r Sonnensystemforschung, Justus-von-Liebig-Weg 3, 37077 G\"{o}ttingen, Germany\\
\altaffilmark{2} Stellar Astrophysics Centre, Department of Physics and Astronomy, Aarhus University, Ny Munkegade 120, DK-8000 Aarhus C, Denmark \\
\altaffilmark{3} Institut f\"ur Informatik, Georg-August-Universit\"at G\"ottingen, Goldschmidtstrasse 7, 37077 G\"ottingen, Germany \\
\altaffilmark{4} Department of Astronomy, Yale University, New Haven, CT 06520, USA \\
\altaffilmark{5} Institut f\"ur Astrophysik G\"ottingen, Friedrich-Hund-Platz 1, 37077 G\"ottingen, Germany}


\begin{abstract}
We develop machine learning methods for instantly estimating fundamental stellar parameters such as the age, mass, radius, and chemical composition of main-sequence solar-like stars from classical and asteroseismic observations. We first demonstrate these methods on a Hare-and-Hound exercise and then apply them to the Sun, 16 Cyg A \& B, and thirty-two \emph{Kepler} objects-of-interest. We find that our estimates and their associated uncertainties are comparable to the results of other methods, but with the additional benefit of needing practically zero computation time. We provide an open-source implementation for the community to use in inferring the attributes of observed stars.\footnote{The source code for all analyses and for all figures appearing in this manuscript can be found electronically at \url{https://github.com/earlbellinger/asteroseismology}.}
\end{abstract}


\keywords{methods: statistical --- stars: abundances --- stars: fundamental parameters --- stars: low-mass --- stars: oscillations --- stars: solar-type}


%%%%%%%%%%%%%%%%%%%%%%%%%%%%%%%%%%%%%%%%%%%%%%%%%%
%%%%%%%%%%%%%%%%% BODY OF PAPER %%%%%%%%%%%%%%%%%%
\section{Introduction}

Asteroseismology provides the opportunity to constrain the ages of stars through accurate inferences of their interior structures.  This in turn drives a wide range of applications in astrophysics, such as characterizing extrasolar planetary systems \citep[e.g.][]{2015ApJ...799..170C,2015MNRAS.452.2127S}, assessing galactic chemical evolution \citep[e.g.][]{2015ASSP...39..111C}, and performing ensemble studies of the Galaxy \citep[e.g.][]{2011Sci...332..213C, 2013MNRAS.429..423M, 2014ApJS..210....1C}. However, seismic ages cannot be measured directly. Instead, they depend on indirect determinations via stellar model calculations. 

To determine the age of an observed star via seismic analysis, models that best match the available observables are sought via mathematical optimization \citep{1994ApJ...427.1013B}. Several search strategies have been employed, including exploration through a pre-computed grid of models (i.e.~grid-based modelling, see \citealt{2011ApJ...730...63G, 2014ApJS..210....1C}); or \emph{in-situ} optimization such as genetic algorithms \citep{2014ApJS..214...27M}, Markov chain Monte Carlo \citep{2012MNRAS.427.1847B}, or downhill simplex \citep{2013ApJS..208....4P} to name but a few (see e.g.~\citealt{2015MNRAS.452.2127S} for an extended discussion). %\footnote{See \citet{2015MNRAS.452.2127S} for a description of the search strategies and techniques in use.} 
%Both grid-based modelling and \emph{in-situ} optimization are computationally expensive, however, and furthermore depend on the choices in physics used to construct the model as well as the uncertainties therein \citep{2014A&A...569A..21L}. 
Utilizing the precision and long temporal observations from the \emph{Kepler} spacecraft and CoRoT satellite, these methods have constrained stellar ages of field stars to within 10\% of their main-sequence lifetime \citep{2015MNRAS.452.2127S}. 

\emph{In-situ} optimization methods are computationally expensive, however, and grid-based modelling methods that rely on interpolation require dense grids \citep{2009ApJ...699..373M}. These methods also depend on the choices in physics used to construct their models as well as the uncertainties therein \citep{2014A&A...569A..21L}. In addition, these methods are based on $\chi^2$-reduction and therefore assume a linear relation between the free parameters of their search and the observations as well as Gaussian distributions for each error function, which can cause uncertainties to be underestimated \citep{2010arXiv1012.3754A}.

The relationships that exist between the observed properties and model parameters are non-linear and difficult to invert. Here we will show that through the use of machine learning, it is possible for these relationships to be characterized %in a semi-analytical manner 
and used to construct a statistical model that is able to relate observations of stars to their internal properties. These learned relationships can then be utilized to process entire catalogs with a cost of only seconds per star. 

In this work, we consider the constrained multiple-regression problem of inferring fundamental stellar parameters from observations. We construct a random forest of decision tree regressors to learn the relationships connecting observable quantities to zero-age main-sequence (ZAMS) properties and current-age evolutionary attributes. The method presented here has strong advantages over existing approaches. First, random forests can be trained and used in only seconds and hence provide massive speedups over other methods. This ability to rapidly process entire catalogs will be crucial in the era of TESS \citep{2015JATIS...1a4003R} and PLATO \citep{2014ExA....38..249R} when seismic data for millions of stars will become available. Secondly, random forests perform non-linear and non-parametric regression -- as opposed to linear interpolation -- which frees us from the strong assumptions laden to other methods and therefore allows us to more faithfully appraise the uncertainties of predicted quantities. %Secondly, iterative searches through the high-dimension parameter space are sensitive to the resolution in their parameter ranges. While iterative searches may identify the best match from a set of \emph{computed} models, they inherently assume that a linear relation exists between observations and model attributes and interpolate within that linear approximation. Interpolating in this fashion, as opposed to the statistical regression applied here, carries assumptions remain unquantified and unpropagated into error estimates.
Thirdly, our method allows us to investigate wide ranges in and combinations of non-canonical stellar physics, whereas other approaches typically have their physics ``locked in,'' using for example the solar mixing length or a fixed amount of core overshooting. This is especially important in the case of atomic diffusion, which is essential when modelling the Sun \citep{1994MNRAS.269.1137B}, but is usually turned off for stars with M/M$_\odot > 1.4$ because it leads to the unobserved consequence of a hydrogen-only surface \citep{2002A&A...390..611M}. Instead of turning it off \emph{ad-hoc} as is commonly done, our method is able to determine empirically the efficiency of atomic diffusion required to reproduce observations. %This is especially important in the case of atomic diffusion, which is essential in modelling the Sun, but is usually turned off in other approaches for stars with M/M$_\odot > 1.4$ because it totally depletes the surface of heavy elements, which is not observed \citep{2002A&A...390..611M}. 
And finally, the method presented here provides the opportunity to extract insights from the statistical regression that is being performed, which contrasts the blind optimization processes of other methods that provide an answer but do not indicate the elements that were important in doing so. 

We apply our method to main-sequence solar-like oscillators, where the detection of many modes of oscillation  offer tight constraints on the stellar interior. We explore various model physics by simulating stars varied not only in their initial mass and chemical composition, but also in their efficiency of convection, extent of core overshooting, and strength of gravitational settling. We validate our technique by inferring the parameters of a Hare-and-Hound exercise, the Sun, and the well-studied stars 16 Cyg A and B. %Notably, in the case of 16 Cyg A and B, the predictions for their respective radii and their corresponding uncertainties match the values measured via interferometry. 
We conclude by processing a catalog of \emph{Kepler} objects-of-interest (hereafter \emph{KAGES}) and compare to the recent results from grid-based modelling, \emph{in-situ} optimization, interferometry, and asteroseismic glitch analyses. 


%%%%%%%%%%%%%%%%%%%%%%%%%%%%%%%%%%%%%%%%%
%%% Grid %%%%%%%%%%%%%%%%%%%%%%%%%%%%%%%%
%%%%%%%%%%%%%%%%%%%%%%%%%%%%%%%%%%%%%%%%%
\section{Method} \label{sec:Method} 
Training a machine learning algorithm for the purpose of characterizing observed stars requires a matrix of evolutionary models from which the machine can learn the processes that relate observable quantities to model properties. In order to construct such a matrix, models along evolutionary sequences are extracted and homogenized to yield the same types of information as the stars being observed. Once the network has learned the non-linear relationships present in the simulations, one can feed the algorithm a catalogue of observational data. This includes, but is not limited to, combinations of 
temperatures, metallicities, frequency separations, surface gravities, luminosities, or radii. %temperatures, metallicities, and frequency separations as well as -- if available -- surface gravities, luminosities, and/or radii. 
A trained machine can then infer current stellar properties such as ages, masses, core-hydrogen and surface-helium abundances. If luminosities, surface gravities, or radii are not supplied, then they can be predicted as well. In addition, the machine also infers ZAMS parameters, including initial mass and initial chemical compositions as well as the mixing length parameter, overshoot coefficient, and diffusion factor needed to reproduce observations, which are explained in more detail below. The determined distributions of these values can later be used as Bayesian priors when constructing more detailed models of the observed stars. 


\subsection{Model generation}
\label{sec:models}
We use the open-source 1D stellar evolution code \emph{Modules for Experiments in Stellar Astrophysics} \citep[MESA,][]{Paxton2011} to generate 100,000 main-sequence stellar models across 1,000 solar-like evolutionary tracks varied in initial mass M, helium Y$_0$, metallicity Z$_0$, mixing length parameter $\alpha_{\text{MLT}}$, overshoot coefficient $\alpha_{\text{ov}}$, and atomic diffusion factor D. The diffusion factor serves to amplify or diminish the effects of diffusion, where a value of zero would turn it off and a value of two would double all coefficients. The initial conditions are varied in the ranges M $\in [0.7, 1.6]$ M$_\odot$, Y$_0$ $\in [0.22, 0.33]$, Z$_0$ $\in [0.0004, 0.04]$ (varied logarithmically), $\alpha_{\text{MLT}}$ $\in [1.5, 2.5]$, $\alpha_{\text{ov}}$ $\in [10^{-4}, 0.5]$ (varied logarithmically), and D $\in [10^{-6}, 10]$ (varied logarithmically). We furthermore put a cut-off of 10$^{-3}$ and 10$^{-5}$ on $\alpha_{\text{ov}}$ and D, respectively, below which we consider zero and turn off. The initial parameters of each track at the ZAMS are chosen in a quasi-random fashion so as to populate the initial-condition hyperspace as rapidly as possible (see Appendix \ref{sec:grid} for more details). The initial conditions considered are projected onto two dimensions in pair-wise scatterplots and histograms in Figure \ref{fig:inputs} with points color-coded by initial hydrogen abundance X$_0$. Note that each variable is varied independently with respect to the the initial hydrogen with the exception of Y$_0$ and Z$_0$, which are required to fulfill baryonic conservation, i.e., X+Y+Z=1. 

We use MESA version r8118 with the Helmholtz-formulated equation of state that allows for radiation pressure and interpolates within the 2005 update of the OPAL EOS tables \citep{2002ApJ...576.1064R}. We assume a \citet{1998SSRv...85..161G} solar composition for our initial abundances and opacity tables. Since we restrict our study to the main sequence, we use an eight-isotope nuclear network consisting of $^1$H, $^3$He, $^4$He, $^{12}$C, $^{14}$N, $^{16}$O, $^{20}$Ne, and $^{24}$Mg. We set a constant $f_0 = 0.001$ to determine the radius $r_0 = H_p \cdot f_0$ inside the convective zone at which convection is switched to overshooting, where $H_p$ is the pressure scale height. All pre-main-sequence (PMS) models are calculated with a simple photospheric approximation, after which an Eddington T-$\tau$ atmosphere is appended on at ZAMS. We end the PMS at the point where the nuclear luminosity represents 99.9\% of the total luminosity, and additionally consider as PMS any models that still decrease in total luminosity past that point. %Explicitly calculating the diffusion rate of each isotope can become computationally expensive. 
We calculate atomic diffusion during the main sequence using four diffusion class representatives: $^1$H, $^4$He, $^{16}$O, and $^{56}$Fe.\footnote{The atomic number of each representative isotope is used to calculate the diffusion rate of the other isotopes allocated to that group; see \citealt{Paxton2011}.} 
Following their most recent measurements, we correct the defaults in MESA of the gravitational constant ($G=6.67408\times 10^{-8}$ \si{\per\g\cm\cubed\per\square\s}; \citealt{2015arXiv150707956M}), the gravitational mass of the Sun ($M_\odot = 1.988475\times 10^{33}$ \si{\g} $= \mu G^{-1} = 1.32712440042\times 10^{11}$ \si{\km\per\s} $G^{-1}$; \citealt{pitjeva2015determination}), and the solar radius ($R_\odot = 6.95568\times 10^{10}$ \si{\cm}; \citealt{2008ApJ...675L..53H}). 

Each track is evolved using at most one-million-year time-steps from ZAMS to either an age of $\tau=15$ Gyr, until it loses its convective envelope, or until terminal-age main sequence (TAMS), which we define as having a fractional core-hydrogen abundance (X$_{\text{c}}$) below $10^{-3}$. We implement adaptive remeshing by recomputing with a finer resolution any track having discontinuities in its surface abundances, which occurs as a result of models with efficient diffusion requiring finer spatial and temporal resolution that those without (see Appendix \ref{sec:remeshing} for details). In order to prevent bias towards any particular run, we select the same number of models from each evolutionary track (see Appendix \ref{sec:selection} for details). Running stellar physics codes in a batch mode like this requires care, so we manually inspect Hertzsprung-Russell, Kippenhahn, and Christensen-Dalsgaard diagrams of all evolutionary tracks to ensure that proper convergence has been achieved. %A great number of complications arise in creating a grid of MESA models simultaneously varied in many dimensions; see the discussion in Appendix \ref{sec:mesa} for a discussion of problems and solutions for undertaking such an exercise. 

\begin{figure*}
    \centering
    \includegraphics[width=\linewidth,keepaspectratio]{figs/inputs.png}
    \caption{Scatterplot matrix (lower panel) and density plots (diagonal) of evolutionary track initial conditions considered. Mass (M), initial helium (Y$_0$), initial metallicity (Z$_0$), mixing length parameter ($\alpha_{\text{MLT}}$), overshoot ($\alpha_{\text{ov}}$), and diffusion factor (D) were varied in a quasi-random fashion to obtain a low-discrepancy grid of model tracks. Points are colored by their initial hydrogen X$_0=1-$Y$_0-$Z$_0$, with blue being low X$_0$ ($\sim 62\%$) and black being high X$_0$ ($\sim 78\%$). The space is densely populated with evolutionary tracks of maximally-different initial conditions. }
    \label{fig:inputs}
\end{figure*}

%%%%%%%%%%%%%%%%%%%%%%%%%%%%%%%
%%% Seismology %%%%%%%%%%%%%%%%
%%%%%%%%%%%%%%%%%%%%%%%%%%%%%%%
\subsection{Seismological calculations}
\label{sec:seis}
We use the ADIPLS pulsation package \citep{2008Ap&SS.316..113C} to post-process all p-mode oscillations up to spherical degree $\ell=3$ and below the acoustic cut-off frequency. We define a frequency separation $S$ as the difference between a frequency $\nu$ of spherical degree $\ell$ and radial order $n$ and another frequency, that is:
\begin{equation} 
  S_{(\ell_1, \ell_2)}(n_1, n_2) \equiv \nu_{\ell_1}(n_1) - \nu_{\ell_2}(n_2).
\end{equation}
The large frequency separation is then
\begin{equation} 
  \Delta\nu_\ell(n) \equiv S_{(\ell, \ell)}(n, n-1)
\end{equation}
and the small frequency separation is
\begin{equation}
  \delta\nu_{(\ell, \ell+2)}(n) \equiv S_{(\ell, \ell+2)}(n, n-1).
\end{equation}
The ratios between the large and small frequency separations (Equation \ref{eqn:LSratio}), and also between the large frequency separation and five-point-averaged frequencies (Equation \ref{eqn:rnl}), have been shown to be less sensitive to the surface term than the aforementioned separations and are therefore valuable asteroseismic diagnostics of stellar interiors \citep{2003A&A...411..215R}. They are defined as
\begin{equation} 
  \mathrm{r}_{(\ell,\ell+2)}(n) \equiv \frac{\delta\nu_\ell(n)}{\Delta\nu_{(1-\ell)}(n+\ell)} \label{eqn:LSratio}
\end{equation}
%and
\begin{equation} 
  \mathrm{r}_{(\ell, 1-\ell)}(n) \equiv \frac{\mathrm{dd}_{(\ell,1-\ell)}(n)}{\Delta\nu_{(1-\ell)}(n+\ell)} \label{eqn:rnl}
  %r_{0, 1}(n) \equiv \frac{1}{8\Delta\nu_{1}(n)} [&\nu_0(n-1) - 4\nu_1(n-1) + 6\nu_0(n) \notag\\
  %& - 4\nu_1(n) + \nu_0(n+1)]
\end{equation}
where
\begin{align} 
  \mathrm{dd}_{0,1} \equiv \frac{1}{8} \big[&\nu_0(n-1) - 4\nu_1(n-1) \notag\\
                                 &+6\nu_0(n) - 4\nu_1(n) + \nu_0(n+1)\big]\\ 
  \mathrm{dd}_{1,0} \equiv -\frac{1}{8} \big[&\nu_1(n-1) - 4\nu_0(n) \notag\\
                                 &+6\nu_1(n) - 4\nu_0(n+1) + \nu_1(n+1)\big].
\end{align}
%where
%\begin{align}
%  dd_{0,1}(n) = \frac{1}{8} [ &\nu_0(n-1) - 4\nu_1(n-1) + 6\nu_0(n) \notag\\
%  & - 4\nu_1(n) + \nu_0(n+1) ]
%\end{align}
Since the set of radial orders that are observable differs from star to star, we collect global statistics on $\delta\nu_{0,2}$, $\delta\nu_{1,3}$, $r_{0,2}$, $r_{1,3}$, $r_{0,1}$, and $r_{1,0}$. %We omit $\Delta\nu$ because it has considerable nonlinear structure to it, and also because it as well has been shown to be highly sensitive to the surface term (ibid). 
We mimic the range of observable frequencies in our models by weighting all frequencies by their position in a Gaussian envelope centered at the predicted frequency of maximum oscillation power $\nu_{\max}$ and having full-width at half-maximum of $0.66\cdot\nu_{\max}{}^{0.88}$ as per the prescription given by \citet{2012A&A...537A..30M}. We then calculate the weighted median of each variable, which we denote with angled parenthesis (e.g.~$\langle\delta\nu_{0,2}\rangle$).\footnote{We also investigated using weighted linear regression to obtain slopes (e.g.~$\langle\dv{\delta\nu_{0,2}}{\nu}\rangle$), but we found that these values did not enhance our analysis.} We elect to use the median rather than the mean because it has a breakdown point of 50\%, meaning that half of the frequencies can be replaced by noise and our result will not differ (for a discussion of breakdown points, see \citealt{hampel1971general}, who attributed them to Gauss). This approach allows us to predict the fundamental parameters of any solar-like oscillator irrespective of which radial orders are observed for that star. Illustrations of these techniques applied to the frequencies of a model in our grid are shown in Figure \ref{fig:ratios}. 

\begin{figure*}
    \centering
    \includegraphics[width=0.5\linewidth,keepaspectratio]{figs/freqs/solar-test-Dnu0.pdf}\hfill
    \includegraphics[width=0.5\linewidth,keepaspectratio]{figs/freqs/solar-test-dnu02.pdf}\\
    \includegraphics[width=0.5\linewidth,keepaspectratio]{figs/freqs/solar-test-r_avg01.pdf}\hfill
    \includegraphics[width=0.5\linewidth,keepaspectratio]{figs/freqs/solar-test-r_sep02.pdf}%
    \caption{The large and small frequency separations $\Delta\nu_0$ and $\delta\nu_{0,2}$ and frequency ratios $r_{0,1}$ and $r_{0,2}$ of a stellar model. The vertical dotted line indicates $\nu_{\max}$. A $\nu_{\max}$-weighted linear fit is indicated with a dashed diagonal line to guide the eye. Point sizes and colors are proportional to the applied weighting, with blue points having a large influence and red points having little. }%
    \label{fig:ratios}
\end{figure*}

\subsection{Training the Random Forest} \label{sec:forest}
We train a random forest regressor on our matrix of evolutionary models to discover the relations that facilitate inference of fundamental stellar parameters from observable quantities. The topology of our random forest regressor can be seen in Figure \ref{fig:rf}. There are several good textbooks that discuss random forests; see for example Chapter 15 of Elements of Statistical Learning \citep{hastie2005elements}. We choose random forests over any of the many other nonlinear regression routines (e.g.~Gaussian processes, symbolic regression, neural networks, support vector regression, etc.) for three reasons. First, random forests perform \emph{constrained} regression; that is, they only make predictions within the boundaries of the supplied training data \citep[see e.g.~section 9.2.1 of][]{hastie2005elements}. This is in contrast to other methods like neural networks, which perform unconstrained regression and are therefore not prevented from predicting non-physical quantities such as negative masses or from violating conservation requirements. Secondly, due to the decision rule process explained below, random forests are insensitive to the scale of the data. Unless care is taken, other regression methods will artificially weight some properties like temperature as being more important than, say, luminosity, solely because temperatures are written using larger numbers (e.g.~5777 vs.~1, see for example section 11.5.3 of \citealt{hastie2005elements} for a discussion). And finally, random forests provide the opportunity to extract insight about the actual regression being performed by examining the importances assigned to each observable quantity, which we explore in detail below. 

\begin{figure}
    \centering
    %\documentclass{standalone}
%\usepackage{tikz}
%\usetikzlibrary{arrows,positioning,shapes,decorations.markings}

%\begin{document}

\definecolor{mDarkBrown}{HTML}{604c38}
\definecolor{mDarkTeal}{HTML}{23373b}
\definecolor{DodgerBlue}{HTML}{1E90FF}
\definecolor{DeepPurple}{HTML}{800080}
\definecolor{mLightBrown}{HTML}{EB811B}
\definecolor{mMediumBrown}{HTML}{C87A2F}

\begin{tikzpicture}
    [   shorten >=1pt,
        ->,
        draw=black!50, 
        node distance=7cm, 
        every node/.style={font=\normalsize}
    ]
    \tikzstyle{every pin edge}=[<-,shorten <=1pt];
    \tikzstyle{neuron}=[circle,fill=black!25,minimum size=35pt,inner sep=0pt];
    \tikzstyle{input neuron}=[neuron, fill=mMediumBrown!50!black, text=white];
    \tikzstyle{output neuron}=[neuron, fill=DodgerBlue!50!black, text=white];
    \tikzstyle{hidden neuron}=[neuron, fill=black, text=white];
    \tikzstyle{annot}=[text width=10em, text centered];
    
    \node[input neuron, pin=left:Temperature] (I-1) at (0,-1) {};
    \node[input neuron, pin=left:Metallicity] (I-2) at (0,-2.5) {};
    \node[input neuron, pin=left:{\shortstack[r]{Large frequency\\separation}}] (I-3) at (0,-4) {};
    \node[input neuron, pin=left:{\shortstack[r]{Small frequency\\separations}}] (I-4) at (0,-5.5) {};
    \node[input neuron, pin=left:{\shortstack[r]{Frequency\\ratios}}] (I-5) at (0,-7) {};
    \node[input neuron, pin=left:Luminosity] (I-6) at (0,-8.5) {};
    \node[input neuron, pin=left:{\shortstack[r]{Surface\\gravity}}] (I-7) at (0,-10) {};
    \node[input neuron, pin=left:Radius] (I-8) at (0,-11.5) {};

    \foreach \name / \y in {1,...,9}
        \path[yshift=1.25cm]
            node[hidden neuron] (H-\name) at (3.5cm,{-\y*1.5 cm}) {$\bigwedge$};
    
    \node[output neuron,pin={[pin edge={->}]right:Age}] (O-1) at (7cm, -1 cm) {};
    \node[output neuron,pin={[pin edge={->}]right:Mass}] (O-2) at (7cm, -2.5 cm) {};
    \node[output neuron,pin={[pin edge={->}]right:\shortstack[l]{Initial\\Helium}}] (O-3) at (7cm, -4 cm) {};
    \node[output neuron,pin={[pin edge={->}]right:\shortstack[l]{Initial\\Metallicity}}] (O-4) at (7cm, -5.5 cm) {};
    \node[output neuron,pin={[pin edge={->}]right:{\shortstack[l]{Mixing\\length}}}] (O-5) at (7cm, -7 cm) {};
    \node[output neuron,pin={[pin edge={->}]right:Overshoot}] (O-6) at (7cm, -8.5 cm) {};
    \node[output neuron,pin={[pin edge={->}]right:Diffusion}] (O-7) at (7cm, -10 cm) {};
    \node[output neuron,pin={[pin edge={->}]right:{\shortstack[l]{Other\\information}}}] (O-8) at (7cm, -11.5 cm) {};
    
    \foreach \source in {1,...,8}
        \foreach \dest in {1,...,9}
            \path (I-\source) edge (H-\dest);
            
    \foreach \source in {1,...,9}
        \foreach \dest in {1,...,8}
            \path (H-\source) edge (O-\dest);
    
    \node[input neuron] (I-1) at (0,-1) {\textbf{T$_{\text{eff}}$}};
    \node[input neuron] (I-2) at (0,-2.5) {\textbf{[Fe/H]}};
    \node[input neuron] (I-3) at (0,-4) {$\boldsymbol{\langle\Delta\nu_0\rangle}$};
    \node[input neuron] (I-4) at (0,-5.5) {$\boldsymbol{\langle\delta\nu\rangle}$};
    \node[input neuron] (I-5) at (0,-7) {$\boldsymbol\langle$\textbf{r}$\boldsymbol\rangle$};
    \node[input neuron] (I-6) at (0,-8.5) {\textbf{L}};
    \node[input neuron] (I-7) at (0,-10) {$\boldsymbol{\log}$ \textbf{g}};
    \node[input neuron] (I-8) at (0,-11.5) {\textbf{R}};
    
    \foreach \name / \y in {1,...,9}
        \path[yshift=1.25cm]
            node[hidden neuron] (H-\name) at (3.5cm,{-\y*1.5 cm}) {$\bigwedge$};
    
    \node[output neuron] (O-1) at (7cm, -1 cm) {$\boldsymbol\tau$};
    \node[output neuron] (O-2) at (7cm, -2.5 cm) {\textbf{M}};
    \node[output neuron] (O-3) at (7cm, -4 cm) {\textbf{Y}$_{\boldsymbol{0}}$};
    \node[output neuron] (O-4) at (7cm, -5.5 cm) {\textbf{Z}$_{\boldsymbol{0}}$};
    \node[output neuron] (O-5) at (7cm, -7 cm) {$\boldsymbol\alpha_{\textbf{\text{MLT}}}$};
    \node[output neuron] (O-6) at (7cm, -8.5 cm) {$\boldsymbol\alpha_{\textbf{\text{ov}}}$};
    \node[output neuron] (O-7) at (7cm, -10 cm) {\textbf{D}};
    \node[output neuron] (O-8) at (7cm, -11.5 cm) {\textbf{\&}};
    
    \node[annot, below of=H-9, node distance=1cm] (hl) {Decision Trees};
    \node[annot, left of=hl, node distance=3.5cm] {Observations};
    \node[annot, right of=hl, node distance=3.5cm] {Predictions};
    
\end{tikzpicture}

%\end{document}
    \caption{A schematic representation of a random forest regressor for inferring fundamental stellar parameters. Classical observables like temperature and asteroseismic observables like $\delta\nu_{0,2}$ are input on the left side. These quantities are then fed through to some number of hidden decision trees, which each independently predict attributes like age and mass. The predictions are then averaged and output on the right side. Frequency separations and ratios include separate nodes for $\nu_{\max}$-centered weighted medians of $\delta\nu_{0,2}$, $r_{0,2}$, $r_{1,0}$, $r_{0,1}$, and, if they are measured, $\delta\nu_{1,3}$ and $r_{1,3}$.\\
    Surface gravities, luminosities, radii, and octupole modes are not always available (e.g.~with the KAGES stars); in their absence, these quantities can be predicted instead of being supplied. In this case, those nodes can be moved over to the ``prediction'' side instead of being on the ``observations'' side. Also, in addition to potentially unobserved inputs like stellar radii, other interesting model properties can be predicted as well, such as core-hydrogen mass fraction and surface helium. }
    \label{fig:rf}
\end{figure}

A random forest is an ensemble regressor, meaning that it is composed of many individual components that each perform regression, and the forest subsequently averages over the results from each \citep{breiman2001random}. The components of the ensemble are decision trees, each of which learns a set of decision rules for relating the observations to the model parameters. Each decision tree is supplied with a random subset of the evolutionary models and a random subset of the observable quantities, a process known as statistical bagging \citep[see section 8.7 of][]{hastie2005elements}, which prevents the random forest from becoming overfit to the training data. We moreover use a variant on random forests known as \emph{extremely} randomized trees \citep{geurts2006extremely}, which further randomizes the attribute and cut-point choices used when creating decision rules. %Training each decision tree in this fashion, a process known as statistical bagging, has been shown to prevent overfitting to the training data \textbf{cite}. 
%Each decision tree ultimately learns a high-dimensional step function, and if the grid is much finer than the uncertainties on the observations, then supplying Monte-Carlo perturbed inputs to the random forest some large number of times (we choose 10,000) results in smooth prediction densities as will be seen in \S \ref{sec:results}. 
%By efficiently sampling our hypercube we can train out network on a fraction of the stellar models required from grid-based modelling or \emp{in situ} optimization. 

The decision trees use information theory to decide which rule is the best choice for inferring quantities like age and mass from the supplied information \citep[see chapter 9 of][]{hastie2005elements}. At every stage, the rule that creates the largest decrease in mean squared error (MSE) is crafted. A rule may be, for example, ``all models with L $<0.4$ L$_\odot$ have M $<$ M$_\odot$.'' %The information theoretic criterion used for creating decision rules is the Gini (im)purity, which quantifies the extent to which a candidate decision rule (e.g.~select all models with L $<0.4$ L$_\odot$) partitions the data into similar classes (e.g.~all models with L $<0.4$ L$_\odot$ have M $<$ M$_\odot$). 
The rules are refined until every data point that was supplied to that particular tree is fully explained by a sequence of decisions. This process presents an opportunity for not only rapidly inferring stellar parameters from observations, but also for understanding the relationships that exist in the data, as each decision tree explicitly ranks the relative importance and therefore inferential power of each observable quantity. Figure \ref{fig:importances} shows the relative importances of each type of observation in inferring stellar attributes. Figure \ref{fig:importance-covariances} furthermore shows the covariances of these importances. Taken together, these two plots illustrate what combination of variables provide the best connections for determining stellar parameters. 

\begin{figure*}
    \centering
    \includegraphics[width=\linewidth, keepaspectratio]{figs/importances-perturb.pdf}
    \caption{Box-and-whisker plots of relative importances for each observable feature in inferring fundamental stellar parameters as measured by a random forest regressor grown from a grid of evolutionary models. %This forest contains 1024 trees, with each tree determining its own relative importances for all input variables. 
    The boxes display the first (16\%) and third (84\%) quartile of feature importances over all trees; the white line indicates the median; the color indicates the mean, with blue being low and red being high; and the whiskers extend to the most extreme values. The boxes are sorted by the median value. %It can be seen that the asteroseismic frequency ratios and surface gravity are the most frequently used diagnostics by the machine learning algorithm for inferring stellar attributes, whereas the large frequency separation is the least used one. 
    \textbf{note: these results potentially subject to change when new simulations finish running}%the median slopes of the small frequency separation and ratios are the least used ones. 
    %This can be understood by virtue of the fact that $\Delta\nu$ is essentially encoded in the other variables. %slopes of all the measured separations and ratios are extremely similar (see Figure \ref{fig:ratios}), so knowledge of one makes the others redundant. Likewise, the relative down-weighting of measured radii can be understood by the fact that its value can be inferred by the combination of L and T$_{\text{eff}}$, so it adds no new information if those two are already present.  %This relative down-weighting of the small frequency separations can be understood by virtue of the fact that the information they bear are more or less encapsulated by the better-probing $\langle r_{\ell, \ell+1}\rangle$. Likewise, 
     }
    \label{fig:importances}
\end{figure*}

\begin{figure*}
    \centering
    \includegraphics[width=\linewidth, keepaspectratio, trim={0cm 0 0cm 0},clip]{figs/cov-perturb.pdf}
    \caption{Standardized covariances of random forest feature importances, with the standardized variance of each feature importance shown along the diagonal. Blue points on the off-diagonal indicate that when the abscissa variable is selected, the ordinate variable is redundant; and conversely, red points indicate that when the abscissa variable is selected, the ordinate variable becomes more useful for successful forecasting. For example, the redness of the top left cell between the asteroseismic frequency ratio $\langle r_{1,3} \rangle$ (abscissa) and the surface gravity log g (ordinate) indicates that while this ratio is the most useful quantity for making inferences, a tree will often supplement it with information about the surface -- which by construction the ratios lack -- to complete an analysis. Likewise, the large covariance between T$_{\text{eff}}$ and [Fe/H] is due to the fact that the effective temperature of a star is not only a function of its mass but also its composition. 
    \textbf{note: these results potentially subject to change when new simulations finish running}}
    \label{fig:importance-covariances}
\end{figure*}

While each tree is trained on only a subset of the data, all trees are tested on all of the data that they did not see. This provides an ``out-of-bag'' accuracy score representing how well the forest will perform when predicting on observations that it has not seen yet \citep[see section 3.1 of][]{breiman2001random}. %The uncertainty on a prediction is a combination of the uncertainty on each individual prediction, call it $\sigma_s$, and the uncertainty inherent to the predictor making the inference, $\sigma_p$. Since these are uncorrelated, these quantities are additive; that is, $\sigma = \sigma_s+\sigma_p$. In our case, the out-of-bag accuracy score is nearly 100\%, so we may safely assume that $\sigma_p=0$, i.e., all the uncertainty in prediction stems from observation. Naturally, there is most likely a systematic bias introduced by assuming a particular set of physics; however, it is impossible to quantify and propagate such uncertainties. 
We show in Figure \ref{fig:rftraining} the accuracy of the random forest regressor as a function of the number of decision trees in the forest. There it can be seen that with less than 50 trees in the forest and only seconds of training time, the method is able to provide excellent predictive accuracy. 

\begin{figure*}
    \centering
    \includegraphics[width=\linewidth,keepaspectratio]{figs/oob-without.pdf}\\
    \includegraphics[width=\linewidth,keepaspectratio]{figs/time-without.pdf}\\
    \caption{Training accuracies (top) and times (bottom) of a random forest regressor plotted as a function of the number of trees in the forest. In less than one minute's time and using only a small number of trees, a machine learning algorithm can determine the relations that permit inference of stellar parameters from observations. It can also be seen that the training time is linear in the amount of decision trees used to fit the data. After the forest has about 16 trees, a regressor using a limited set of information (blue and red) does just as well as the regressor using all of the information (black). The performance of the regressor does not improve much after having grown about 32 trees. }
    \label{fig:rftraining}
\end{figure*}

It is often the case for many stars that radii, luminosities, surface gravities, and oscillation modes with spherical degree $\ell=3$ are not available. For example, the KAGES data set lacks all of this information, and the hare-and-hound exercise data lack all excepting luminosities. We test the impact of this information being absent by predicting those quantities instead of using them as constraints. We find that measuring these variables is dispensible: the information they hold is contained within the other observable quantities, and therefore they (as well as ages, masses, etc.) can be predicted to the same degree of accuracy that they were measured using these nonlinear inference techniques on temperatures, metallicities, and lower degree modes. This fact is demonstrated in Figure \ref{fig:rftraining} as well. It can be seen that the same accuracy can be achieved when predicting luminosities, surface gravities, and radii as was when they were used as measured inputs. We furthermore show the relative observational importances determined by these forests in Figure \ref{fig:importances2}. Remarkably, when $\ell=3$ modes and especially when luminosities are omitted, temperatures become the most important quantity to have observed. 
%As the nodes are learning statistical relationships between the inputs and outputs the network inherits the uncertainties in stellar modelling used for the training. 

\begin{figure*}
    \centering
    \includegraphics[width=0.5\linewidth, keepaspectratio]{figs/importances-hares.pdf}\hfill
    \includegraphics[width=0.5\linewidth, keepaspectratio]{figs/importances-kages.pdf}
    \caption{Box-and-whisker plots of relative importances of each observable feature in measuring fundamental stellar parameters for the Hare-and-Hound exercise data (left), where luminosities are available; and the \emph{Kepler} objects-of-interest (right), where they are not. Octupole ($\ell=3$) modes have not been measured in any of these stars, so $\delta\nu_{1,3}$ and $r_{1,3}$ from evolutionary modelling are not supplied to these random forests. }
    \label{fig:importances2}
\end{figure*}


%%%%%%%%%%%%%%%%%%%%%%%%%%%%%%%%%%%%%%%%%
%%% Results %%%%%%%%%%%%%%%%%%%%%%%%%%%%%
%%%%%%%%%%%%%%%%%%%%%%%%%%%%%%%%%%%%%%%%%
\section{Results}
We perform three tests of our method. We begin with a Hare-and-Hound simulation exercise to show that we can can reliably recover inputs with known truth values. We then move to the Sun and the solar-like stars of 16 Cygni, which have been well-studied; and we conclude by demonstrating our method on thirty-two \emph{Kepler} objects-of-interest. In each case, we train our random forest regressor with 1024 decision trees on the subset of observational data that is available to the catalogue being processed. In order to propagate uncertainties into the predictions, we perturb all measurements 10,000 times. We account for the covariance between asteroseismic separations and ratios by recalculating them upon each perturbation. 


%%%%%%%%%%%%%%%%%%%%%%%%%%%%%%%%%%%%%%%%%
%%% Hare and Hound %%%%%%%%%%%%%%%%%%%%%%
%%%%%%%%%%%%%%%%%%%%%%%%%%%%%%%%%%%%%%%%%
\subsection{Hare and Hound} 
Sarbani Basu prepared twelve models varied in mass, initial chemical composition, and mixing length parameter with only some models having overshooting and only some models having atomic diffusion included. The models were evolved using the Yale rotating stellar evolution code \citep[YREC,][]{2008Ap&SS.316...31D}, which is a different evolution code than the one that was used to inform the random forest. Effective temperatures, luminosities, [Fe/H] and $\nu_{\max}$ values as well as $\ell=0,1,2$ frequencies were obtained from each model. George Angelou perturbed the ``observations'' of these models according to the scheme devised by \citet{spaceinn}. The true values and the perturbed observations can be seen in Appendix \ref{sec:hare-and-hound}. The perturbed observations and their uncertainties were given to Earl Bellinger, who used the machine learning methods described herein to recover them without being given access to the true values. The predicted quantities are plotted against their true values in Figure \ref{fig:hare-comparison}. Here it can be seen that the method is able to recover the true model values within uncertainties even when they have been perturbed by noise. 

\begin{figure*}
    \centering
    \includegraphics[width=0.5\linewidth,keepaspectratio]{figs/comparison/basu-Mass.pdf}\hfill
    \includegraphics[width=0.5\linewidth,keepaspectratio]{figs/comparison/basu-Radius.pdf}\\
    \includegraphics[width=0.5\linewidth,keepaspectratio]{figs/comparison/basu-Overshoot.pdf}\hfill
    \includegraphics[width=0.5\linewidth,keepaspectratio]{figs/comparison/basu-Diffusion.pdf}\\
    \includegraphics[width=0.5\linewidth,keepaspectratio]{figs/comparison/basu-Age.pdf}
    \caption{asdf \textbf{todo: write stuff} \label{fig:hare-comparison}}
\end{figure*}


%%%%%%%%%%%%%%%%%%%%%%%%%%%%%%%%%%%%%%%%%
%%% The Sun & 16 Cygni %%%%%%%%%%%%%%%%%%
%%%%%%%%%%%%%%%%%%%%%%%%%%%%%%%%%%%%%%%%%
\subsection{The Sun \& 16 Cygni}
In order to be confident about our predictions on \emph{Kepler} data, we first degrade the frequencies of the Quiet Sun that were obtained by the Birmingham Solar-Oscillations Network \citep[BiSON,][]{2014MNRAS.439.2025D} to the level of information that is achievable by the spacecraft. We also degrade the Sun's uncertainties of classical observations by applying 16 Cyg B's uncertainties of effective temperature, luminosity, surface gravity, metallicity, $\nu_{\max}$, radius, and radial velocity. Finally, we perturb each central value with random Gaussian noise according to its uncertainty to reflect the fact that the central value of an uncertain observation is not the true value. 

Effective temperatures, surface gravities, and metallicities of 16 Cyg A and B were obtained from \citet{2009A&A...508L..17R}; radii and luminosities from \citet{2013MNRAS.433.1262W}; and frequencies from \citet{2015MNRAS.446.2959D}. We obtained the radial velocity measurements of 16 Cyg A and B from \citet{2002ApJS..141..503N} and corrected frequencies for Doppler shifting as per the prescription in \citet{2014MNRAS.445L..94D}.\footnote{We tried with and without line-of-sight corrections and found that it did not make a difference to the predicted quantities or their uncertainties.} The initial parameters as predicted by machine learning can be seen in Table \ref{tab:results}, and the predicted current-age parameters can be seen in Table \ref{tab:results-ca}. These results support the hypothesis that 16 Cyg A and B were formed at the same time with the same initial composition. We also show in Figure \ref{fig:corner} the densities and correlations for the predicted age, mass, initial composition, mixing length parameter, overshoot coefficient, and diffusion factor for the degraded Sun. Initial condition distributions like these are useful as strong priors when performing more detailed stellar analyses. 

\begin{deluxetable}{cccccccc}
\tabletypesize{\scriptsize}
\tablecaption{Means and standard deviations for fundamental stellar parameters -- masses, initial chemical compositions, mixing lengths, diffusion factors, and overshoot coefficients -- of 16 Cyg A, 16 Cyg B, and the Sun as a Star inferred via machine learning from their respective temperatures, metallicities, luminosities, surface gravities, radii, and asteroseismic observables $\langle \Delta\nu_0 \rangle$, $\langle \delta\nu_{0,2} \rangle$, $\langle \delta\nu_{1,3} \rangle$, $\langle r_{0,2} \rangle$, $\langle r_{1,3} \rangle$, $\langle r_{0,1} \rangle$, and $\langle r_{1,0} \rangle$. %Values of the undegraded solar data predicted from this set of information are also shown for reference. 
\label{tab:results}}
\tablewidth{0pt}
\tablehead{\colhead{Name} & \colhead{M$/$M$_\odot$} & \colhead{Y$_0$} & \colhead{Z$_0$} & \colhead{$\alpha_{\mathrm{MLT}}$} & \colhead{D} & \colhead{$\alpha_{\mathrm{ov}}$}}\startdata
16CygA & 1.07 $\pm$ 0.024 & 0.27 $\pm$ 0.009 & 0.0221 $\pm$ 0.001 & 1.87 $\pm$ 0.051 & 0.292 $\pm$ 0.14 & 0.205 $\pm$ 0.022 \\
16CygB & 1 $\pm$ 0.01 & 0.275 $\pm$ 0.0025 & 0.0202 $\pm$ 0.0008 & 1.79 $\pm$ 0.057 & 0.417 $\pm$ 0.099 & 0.157 $\pm$ 0.026 \\
Degraded Sun & 1.01 $\pm$ 0.0071 & 0.27 $\pm$ 0.0032 & 0.02 $\pm$ 0.00079 & 1.92 $\pm$ 0.063 & 0.727 $\pm$ 0.14 & 0.154 $\pm$ 0.012 \\
%Sun & 1.01 $\pm$ 0.0062 & 0.273 $\pm$ 0.0019 & 0.0206 $\pm$ 0.0013 & 1.9 $\pm$ 0.012 & 0.725 $\pm$ 0.06 & 0.149 $\pm$ 0.0048 \\
\enddata
\end{deluxetable}

\begin{deluxetable}{cccc}
\tabletypesize{\scriptsize}
%\rotate
\tablecaption{Means and standard deviations for current-age stellar attributes -- age, surface-helium and core-hydrogen abundances -- of 16 Cyg A, 16 Cyg B, and the Sun as a Star inferred via machine learning from their respective temperatures, metallicities, luminosities, surface gravities, radii, and asteroseismic observables $\langle \Delta\nu_0 \rangle$, $\langle \delta\nu_{0,2} \rangle$, $\langle \delta\nu_{1,3} \rangle$, $\langle r_{0,2} \rangle$, $\langle r_{1,3} \rangle$, $\langle r_{0,1} \rangle$, and $\langle r_{1,0} \rangle$. %Values of the undegraded solar data predicted from this set of information are also shown for reference. 
\label{tab:results-ca}}
\tablewidth{0pt}
\tablehead{\colhead{Name} & \colhead{$\tau/$Gyr} & \colhead{X$_{\text{c}}$/M$_*$} & \colhead{Y$_{\text{surf}}}}\startdata
16CygA & 7.07 $\pm$ 0.4 & 0.0931 $\pm$ 0.019 & 0.264 $\pm$ 0.0095 \\
16CygB & 7.2 $\pm$ 0.27 & 0.15 $\pm$ 0.019 & 0.264 $\pm$ 0.003 \\
Degraded Sun & 4.5 $\pm$ 0.15 & 0.366 $\pm$ 0.018 & 0.256 $\pm$ 0.0018 \\
%Sun & 4.4 $\pm$ 0.081 & 0.374 $\pm$ 0.0038 & 0.259 $\pm$ 0.0023 \\
\enddata
\end{deluxetable}

\begin{figure*}
    \centering
    \includegraphics[width=\linewidth,keepaspectratio]{figs/Tagesstern_init-corner.pdf}
    \caption{Predictions from machine learning of initial stellar parameters for degraded solar data. The diagonal shows histograms of mass M, initial helium Y$_0$, initial metallicity $Z_0$, mixing length parameter $\alpha_{\text{MLT}}$, diffusion factor D, and overshoot $\alpha_{\text{ov}}$. The lower panel shows contour plots for each pair of these variables. }
    \label{fig:corner}
\end{figure*}

We additionally predict the radii and luminosities of 16 Cyg A and B instead of using them as constraints. Figure \ref{fig:interferometry} shows our inferred radii, luminosities, surface-helium, and core-hydrogen abundances of 16 Cyg A and B plotted against the values determined by interferometry \citep{2013MNRAS.433.1262W} and an asteroseismic estimate \citep{2014ApJ...790..138V}. Here again we find excellent agreement between our method and the measured values. 

\begin{figure*}
    \centering
    %\includegraphics[width=\linewidth,keepaspectratio]{figs/16cyg-all.png}
    \includegraphics[width=0.5\linewidth, keepaspectratio]{figs/hists/cyg-radius.pdf}\hfill
    \includegraphics[width=0.5\linewidth, keepaspectratio]{figs/hists/cyg-L.pdf}\\
    \includegraphics[width=0.5\linewidth, keepaspectratio]{figs/hists/cyg-Y_surf.pdf}\hfill
    \includegraphics[width=0.5\linewidth, keepaspectratio]{figs/hists/cyg-X_c.pdf}\\
    \caption{Probability densities showing predictions of 16 Cyg A (red) and B (blue) from machine learning of radii (top left) and luminosities (top right) as well as surface-helium (bottom left) and core-hydrogen (bottom right) abundances. Relative uncertainties ($\epsilon = \frac{\sigma}{\mu}\cdot 100$, with mean $\mu$ and standard deviation $\sigma$) are shown beside of each plot. Predictions and $2\sigma$ uncertainties from interferometric (``int'') measurements and asteroseismic (``ast'') estimates are shown with arrows.}
    \label{fig:interferometry}
\end{figure*}

Detailed modelling of 16 Cyg A and B have been performed  by \citet{2015ApJ...811L..37M} using the Asteroseismic Modeling Portal (AMP). They calculated their results using a fixed diffusion factor, but without heavy element diffusion (i.e.~only helium diffusion) and with no overshooting. In order to account for systematic uncertainties, they multiplied the spectroscopic uncertainties of 16 Cyg A and B by an arbitrary constant $c=3$. Therefore, in order to make a fair comparison between the results of our method and theirs, we generate a new matrix of evolutionary models with those same conditions and also increase the uncertainties on [Fe/H] by a factor of $c$. In Figure \ref{fig:16Cyg-hist}, we show probability densities of the predicted parameters of 16 Cyg A and B that we obtain using machine learning in comparison to the results obtained by AMP. We find good agreement of values and uncertainties. To perform their analysis, AMP required 7,712.99 hours and 7,467.28 hours of computational time to model 16 Cyg A and B respectively using the world's 10th fastest supercomputer, the Texas Advanced Computing Center Stampede \citep{TOP500}. Here we have obtained comparable results in less than one minute. We note however that detailed optimization codes like AMP still remain important for obtaining structural models of stars as well as for performing post-main sequence modelling, where our method has not yet been extended. 

\begin{figure*}
    \centering
    %\includegraphics[width=\linewidth,keepaspectratio]{figs/16cyg-all.png}
    \includegraphics[width=0.5\linewidth, keepaspectratio]{figs/hists/cyg-age.pdf}\hfill
    \includegraphics[width=0.5\linewidth, keepaspectratio]{figs/hists/cyg-M.pdf}\\
    \includegraphics[width=0.5\linewidth, keepaspectratio]{figs/hists/cyg-Y.pdf}\hfill
    \includegraphics[width=0.5\linewidth, keepaspectratio]{figs/hists/cyg-Z.pdf}
    %\includegraphics[width=0.5\linewidth, keepaspectratio]{figs/hists/cyg-alpha.pdf}\\
    %\includegraphics[width=0.5\linewidth, keepaspectratio]{figs/hists/cyg-overshoot.pdf}\hfill
    %\includegraphics[width=0.5\linewidth, keepaspectratio]{figs/hists/cyg-diffusion.pdf}\\
    
    \caption{Probability densities showing predictions from machine learning of fundamental stellar parameters for 16 Cyg A (red) and B (blue) against predictions from AMP modelling. Relative uncertainties are shown beside each plot. Predictions and $2\sigma$ uncertainties from AMP modelling are shown with arrows.}
    \label{fig:16Cyg-hist}
\end{figure*}


%%%%%%%%%%%%%%%%%%%%%%%%%%%%%%%%%%%%%%%%%
%%% Kepler Objects of Interest %%%%%%%%%%
%%%%%%%%%%%%%%%%%%%%%%%%%%%%%%%%%%%%%%%%%
\subsection{\emph{Kepler} Objects of Interest}
We obtain classical observations and frequencies of the KAGES targets from \citet{2015MNRAS.452.2127S}. We use line-of-sight radial velocity corrections when available, which was only the case with KIC 6278762 \citep{2002AJ....124.1144L}, KIC 10666592 \citep{2013A&A...554A..84M}, and KIC 3632418 \citep{2006AstL...32..759G}. We use the random forest whose feature importances were shown in Figure \ref{fig:importances2} to predict the fundamental properties of these stars, the results of which can be seen in Table \ref{tab:results-kages}. Finally, Figure \ref{fig:us-vs-them} shows the fundamental parameters obtained from our method plotted against those obtained by KAGES. We find good agreement across all stars.

\begin{deluxetable}{ccccccc}
\tabletypesize{\scriptsize}
\tablecaption{Means and standard deviations for initial conditions -- mass, chemical composition, mixing length, diffusion factor, and overshoot coefficient -- of the KAGES data set inferred via machine learning from their respective temperatures, metallicities, and asteroseismic observables $\langle \Delta\nu_0 \rangle$, $\langle \delta\nu_{0,2} \rangle$, $\langle r_{0,2} \rangle$, $\langle r_{0,1} \rangle$, and $\langle r_{1,0} \rangle$. \label{tab:results-kages}}
\tablewidth{0pt}
\tablehead{\colhead{KIC} & \colhead{M$/$M$_\odot$} & \colhead{Y$_0$} & \colhead{Z$_0$} & \colhead{$\alpha_{\mathrm{MLT}}$} & \colhead{D} & \colhead{$\alpha_{\mathrm{ov}}$}}\startdata
3425851 & 1.15 $\pm$ 0.034 & 0.283 $\pm$ 0.0097 & 0.0148 $\pm$ 0.0026 & 2.03 $\pm$ 0.16 & 0.145 $\pm$ 0.13 & 0.15 $\pm$ 0.045 \\
3544595 & 0.911 $\pm$ 0.026 & 0.274 $\pm$ 0.0086 & 0.0144 $\pm$ 0.0022 & 1.94 $\pm$ 0.072 & 1.18 $\pm$ 0.54 & 0.255 $\pm$ 0.045 \\
3632418 & 1.35 $\pm$ 0.046 & 0.278 $\pm$ 0.012 & 0.0173 $\pm$ 0.0029 & 2.04 $\pm$ 0.16 & 0.582 $\pm$ 0.4 & 0.276 $\pm$ 0.075 \\
4141376 & 1 $\pm$ 0.048 & 0.281 $\pm$ 0.014 & 0.0118 $\pm$ 0.0024 & 1.86 $\pm$ 0.096 & 1.02 $\pm$ 0.85 & 0.172 $\pm$ 0.049 \\
4143755 & 0.949 $\pm$ 0.022 & 0.27 $\pm$ 0.003 & 0.00979 $\pm$ 0.0018 & 1.64 $\pm$ 0.019 & 2.34 $\pm$ 0.94 & 0.354 $\pm$ 0.023 \\
4349452 & 1.19 $\pm$ 0.044 & 0.273 $\pm$ 0.011 & 0.0177 $\pm$ 0.0036 & 1.86 $\pm$ 0.13 & 0.589 $\pm$ 0.52 & 0.173 $\pm$ 0.046 \\
4914423 & 1.12 $\pm$ 0.037 & 0.28 $\pm$ 0.009 & 0.0201 $\pm$ 0.0036 & 1.78 $\pm$ 0.068 & 0.434 $\pm$ 0.31 & 0.22 $\pm$ 0.037 \\
5094751 & 1.07 $\pm$ 0.043 & 0.275 $\pm$ 0.0084 & 0.0152 $\pm$ 0.0033 & 1.78 $\pm$ 0.11 & 0.318 $\pm$ 0.21 & 0.225 $\pm$ 0.043 \\
5866724 & 1.24 $\pm$ 0.049 & 0.278 $\pm$ 0.0085 & 0.0226 $\pm$ 0.0037 & 1.88 $\pm$ 0.12 & 0.699 $\pm$ 0.58 & 0.202 $\pm$ 0.045 \\
6278762 & 0.778 $\pm$ 0.012 & 0.258 $\pm$ 0.0062 & 0.0143 $\pm$ 0.0021 & 2.06 $\pm$ 0.1 & 0.995 $\pm$ 0.33 & 0.235 $\pm$ 0.044 \\
6521045 & 1.12 $\pm$ 0.033 & 0.275 $\pm$ 0.0077 & 0.0196 $\pm$ 0.0034 & 1.79 $\pm$ 0.049 & 0.641 $\pm$ 0.31 & 0.241 $\pm$ 0.033 \\
7670943 & 1.29 $\pm$ 0.062 & 0.282 $\pm$ 0.0083 & 0.0208 $\pm$ 0.0041 & 2.04 $\pm$ 0.13 & 0.31 $\pm$ 0.25 & 0.178 $\pm$ 0.047 \\
8077137 & 1.18 $\pm$ 0.054 & 0.274 $\pm$ 0.0075 & 0.0143 $\pm$ 0.0024 & 1.78 $\pm$ 0.11 & 0.371 $\pm$ 0.37 & 0.272 $\pm$ 0.048 \\
8292840 & 1.12 $\pm$ 0.062 & 0.264 $\pm$ 0.0064 & 0.0124 $\pm$ 0.0024 & 1.8 $\pm$ 0.088 & 0.874 $\pm$ 0.83 & 0.227 $\pm$ 0.057 \\
8349582 & 1.13 $\pm$ 0.025 & 0.283 $\pm$ 0.0053 & 0.0289 $\pm$ 0.0029 & 1.82 $\pm$ 0.081 & 0.404 $\pm$ 0.23 & 0.249 $\pm$ 0.022 \\
8478994 & 0.795 $\pm$ 0.021 & 0.277 $\pm$ 0.0092 & 0.00993 $\pm$ 0.0013 & 1.92 $\pm$ 0.1 & 1.56 $\pm$ 0.98 & 0.277 $\pm$ 0.068 \\
8494142 & 1.4 $\pm$ 0.044 & 0.286 $\pm$ 0.013 & 0.0251 $\pm$ 0.0035 & 1.72 $\pm$ 0.067 & 0.46 $\pm$ 0.39 & 0.163 $\pm$ 0.045 \\
8554498 & 1.35 $\pm$ 0.056 & 0.282 $\pm$ 0.0063 & 0.0253 $\pm$ 0.0026 & 1.76 $\pm$ 0.042 & 0.405 $\pm$ 0.23 & 0.198 $\pm$ 0.044 \\
8866102 & 1.22 $\pm$ 0.055 & 0.278 $\pm$ 0.0093 & 0.019 $\pm$ 0.0037 & 1.86 $\pm$ 0.1 & 0.495 $\pm$ 0.44 & 0.173 $\pm$ 0.041 \\
9414417 & 1.33 $\pm$ 0.042 & 0.269 $\pm$ 0.0092 & 0.015 $\pm$ 0.0022 & 1.84 $\pm$ 0.13 & 0.509 $\pm$ 0.4 & 0.286 $\pm$ 0.047 \\
9592705 & 1.41 $\pm$ 0.03 & 0.296 $\pm$ 0.0094 & 0.0238 $\pm$ 0.003 & 1.69 $\pm$ 0.071 & 0.143 $\pm$ 0.11 & 0.193 $\pm$ 0.025 \\
9955598 & 0.931 $\pm$ 0.017 & 0.269 $\pm$ 0.0099 & 0.0212 $\pm$ 0.003 & 1.91 $\pm$ 0.089 & 0.772 $\pm$ 0.49 & 0.205 $\pm$ 0.054 \\
10514430 & 1.08 $\pm$ 0.045 & 0.277 $\pm$ 0.0077 & 0.0152 $\pm$ 0.0027 & 1.72 $\pm$ 0.043 & 0.843 $\pm$ 0.4 & 0.312 $\pm$ 0.04 \\
10586004 & 1.22 $\pm$ 0.067 & 0.279 $\pm$ 0.0054 & 0.0262 $\pm$ 0.0039 & 1.79 $\pm$ 0.089 & 0.394 $\pm$ 0.28 & 0.294 $\pm$ 0.049 \\
10666592 & 1.44 $\pm$ 0.031 & 0.303 $\pm$ 0.0093 & 0.0254 $\pm$ 0.0025 & 1.75 $\pm$ 0.14 & 0.0705 $\pm$ 0.088 & 0.196 $\pm$ 0.043 \\
10963065 & 1.04 $\pm$ 0.044 & 0.275 $\pm$ 0.0076 & 0.0127 $\pm$ 0.0027 & 1.8 $\pm$ 0.079 & 0.585 $\pm$ 0.35 & 0.213 $\pm$ 0.032 \\
11133306 & 1.11 $\pm$ 0.042 & 0.268 $\pm$ 0.011 & 0.0196 $\pm$ 0.0033 & 1.86 $\pm$ 0.073 & 0.989 $\pm$ 0.6 & 0.169 $\pm$ 0.036 \\
11295426 & 1.1 $\pm$ 0.032 & 0.268 $\pm$ 0.013 & 0.0231 $\pm$ 0.0021 & 1.82 $\pm$ 0.07 & 0.414 $\pm$ 0.37 & 0.184 $\pm$ 0.032 \\
11401755 & 1.1 $\pm$ 0.019 & 0.274 $\pm$ 0.0047 & 0.0115 $\pm$ 0.0012 & 1.76 $\pm$ 0.028 & 0.936 $\pm$ 0.46 & 0.321 $\pm$ 0.033 \\
11807274 & 1.28 $\pm$ 0.062 & 0.273 $\pm$ 0.0072 & 0.0195 $\pm$ 0.0033 & 1.81 $\pm$ 0.08 & 0.344 $\pm$ 0.3 & 0.187 $\pm$ 0.036 \\
11853905 & 1.15 $\pm$ 0.052 & 0.281 $\pm$ 0.0083 & 0.0224 $\pm$ 0.0035 & 1.83 $\pm$ 0.088 & 0.866 $\pm$ 0.54 & 0.278 $\pm$ 0.039 \\
11904151 & 0.917 $\pm$ 0.031 & 0.271 $\pm$ 0.0079 & 0.0149 $\pm$ 0.0026 & 1.85 $\pm$ 0.064 & 0.958 $\pm$ 0.45 & 0.25 $\pm$ 0.04 \\
\enddata
\end{deluxetable}

\begin{deluxetable}{ccccccc}
\tabletypesize{\scriptsize}
\tablecaption{Means and standard deviations for current-age conditions -- age, core-hydrogen abundance, surface gravity, luminosity, radius, and surface-helium abundance -- of the KAGES data set inferred via machine learning from their respective temperatures, metallicities, and asteroseismic observables $\langle \Delta\nu_0 \rangle$, $\langle \delta\nu_{0,2} \rangle$, $\langle r_{0,2} \rangle$, $\langle r_{0,1} \rangle$, and $\langle r_{1,0} \rangle$. \label{tab:results-kages-curr}}
\tablewidth{0pt}
\tablehead{\colhead{KIC} & \colhead{$\tau/$Gyr} & \colhead{X$_{\text{c}}$/M$_*$} & \colhead{log g (cgs)} & \colhead{L$/$L$_\odot$} & \colhead{R$/$R$_\odot$} & \colhead{Y$_{\text{surf}}$}}\startdata
3425851 & 3.96 $\pm$ 0.55 & 0.182 $\pm$ 0.058 & 4.23 $\pm$ 0.0086 & 2.7 $\pm$ 0.14 & 1.36 $\pm$ 0.016 & 0.277 $\pm$ 0.012 \\
3544595 & 6.47 $\pm$ 1.2 & 0.335 $\pm$ 0.08 & 4.47 $\pm$ 0.017 & 0.848 $\pm$ 0.061 & 0.924 $\pm$ 0.015 & 0.249 $\pm$ 0.013 \\
3632418 & 3.15 $\pm$ 0.28 & 0.139 $\pm$ 0.042 & 4.02 $\pm$ 0.0071 & 5.05 $\pm$ 0.2 & 1.89 $\pm$ 0.023 & 0.256 $\pm$ 0.019 \\
4141376 & 3.27 $\pm$ 0.69 & 0.404 $\pm$ 0.076 & 4.42 $\pm$ 0.011 & 1.38 $\pm$ 0.098 & 1.03 $\pm$ 0.022 & 0.262 $\pm$ 0.016 \\
4143755 & 9.41 $\pm$ 0.52 & 0.0448 $\pm$ 0.012 & 4.11 $\pm$ 0.0077 & 2 $\pm$ 0.1 & 1.42 $\pm$ 0.012 & 0.216 $\pm$ 0.021 \\
4349452 & 3.04 $\pm$ 0.68 & 0.353 $\pm$ 0.072 & 4.28 $\pm$ 0.011 & 2.44 $\pm$ 0.13 & 1.31 $\pm$ 0.017 & 0.257 $\pm$ 0.014 \\
4914423 & 6.35 $\pm$ 0.81 & 0.069 $\pm$ 0.026 & 4.16 $\pm$ 0.0062 & 2.3 $\pm$ 0.16 & 1.46 $\pm$ 0.023 & 0.269 $\pm$ 0.013 \\
5094751 & 6.42 $\pm$ 0.83 & 0.11 $\pm$ 0.024 & 4.21 $\pm$ 0.0071 & 2.11 $\pm$ 0.15 & 1.35 $\pm$ 0.023 & 0.266 $\pm$ 0.01 \\
5866724 & 3.31 $\pm$ 0.64 & 0.317 $\pm$ 0.084 & 4.24 $\pm$ 0.012 & 2.65 $\pm$ 0.12 & 1.4 $\pm$ 0.021 & 0.261 $\pm$ 0.013 \\
6278762 & 10.1 $\pm$ 0.97 & 0.391 $\pm$ 0.026 & 4.57 $\pm$ 0.0057 & 0.351 $\pm$ 0.022 & 0.756 $\pm$ 0.0049 & 0.237 $\pm$ 0.011 \\
6521045 & 6.76 $\pm$ 0.7 & 0.0323 $\pm$ 0.01 & 4.13 $\pm$ 0.0036 & 2.44 $\pm$ 0.16 & 1.52 $\pm$ 0.025 & 0.261 $\pm$ 0.011 \\
7670943 & 2.52 $\pm$ 0.63 & 0.349 $\pm$ 0.08 & 4.23 $\pm$ 0.0086 & 3.29 $\pm$ 0.28 & 1.44 $\pm$ 0.027 & 0.269 $\pm$ 0.014 \\
8077137 & 4.87 $\pm$ 0.77 & 0.102 $\pm$ 0.053 & 4.07 $\pm$ 0.01 & 3.54 $\pm$ 0.27 & 1.65 $\pm$ 0.037 & 0.26 $\pm$ 0.014 \\
8292840 & 4.55 $\pm$ 1.4 & 0.26 $\pm$ 0.11 & 4.25 $\pm$ 0.018 & 2.42 $\pm$ 0.2 & 1.32 $\pm$ 0.023 & 0.236 $\pm$ 0.021 \\
8349582 & 7.55 $\pm$ 0.58 & 0.0461 $\pm$ 0.016 & 4.17 $\pm$ 0.0058 & 2.04 $\pm$ 0.11 & 1.46 $\pm$ 0.012 & 0.273 $\pm$ 0.0082 \\
8478994 & 6.2 $\pm$ 2.2 & 0.496 $\pm$ 0.053 & 4.57 $\pm$ 0.0099 & 0.481 $\pm$ 0.034 & 0.767 $\pm$ 0.0076 & 0.258 $\pm$ 0.014 \\
8494142 & 2.75 $\pm$ 0.44 & 0.204 $\pm$ 0.06 & 4.04 $\pm$ 0.0092 & 4.66 $\pm$ 0.25 & 1.87 $\pm$ 0.028 & 0.26 $\pm$ 0.028 \\
8554498 & 3.78 $\pm$ 0.59 & 0.109 $\pm$ 0.048 & 4.03 $\pm$ 0.0084 & 4.15 $\pm$ 0.23 & 1.86 $\pm$ 0.034 & 0.268 $\pm$ 0.013 \\
8866102 & 2.69 $\pm$ 0.65 & 0.369 $\pm$ 0.087 & 4.27 $\pm$ 0.011 & 2.68 $\pm$ 0.15 & 1.35 $\pm$ 0.02 & 0.266 $\pm$ 0.015 \\
9414417 & 3.32 $\pm$ 0.3 & 0.172 $\pm$ 0.028 & 4.01 $\pm$ 0.0062 & 4.91 $\pm$ 0.29 & 1.88 $\pm$ 0.027 & 0.245 $\pm$ 0.018 \\
9592705 & 2.76 $\pm$ 0.25 & 0.13 $\pm$ 0.034 & 3.97 $\pm$ 0.0072 & 5.59 $\pm$ 0.3 & 2.04 $\pm$ 0.022 & 0.288 $\pm$ 0.012 \\
9955598 & 6.33 $\pm$ 0.81 & 0.386 $\pm$ 0.033 & 4.5 $\pm$ 0.006 & 0.657 $\pm$ 0.043 & 0.901 $\pm$ 0.0097 & 0.255 $\pm$ 0.014 \\
10514430 & 7.2 $\pm$ 0.85 & 0.0399 $\pm$ 0.014 & 4.08 $\pm$ 0.0074 & 2.67 $\pm$ 0.21 & 1.57 $\pm$ 0.031 & 0.255 $\pm$ 0.014 \\
10586004 & 5.91 $\pm$ 1 & 0.085 $\pm$ 0.049 & 4.08 $\pm$ 0.017 & 2.93 $\pm$ 0.22 & 1.67 $\pm$ 0.046 & 0.268 $\pm$ 0.0083 \\
10666592 & 2.25 $\pm$ 0.14 & 0.218 $\pm$ 0.034 & 4.01 $\pm$ 0.0062 & 5.53 $\pm$ 0.34 & 1.95 $\pm$ 0.022 & 0.3 $\pm$ 0.012 \\
10963065 & 5.29 $\pm$ 0.7 & 0.217 $\pm$ 0.039 & 4.29 $\pm$ 0.0066 & 1.91 $\pm$ 0.12 & 1.21 $\pm$ 0.024 & 0.256 $\pm$ 0.013 \\
11133306 & 4.73 $\pm$ 0.65 & 0.259 $\pm$ 0.05 & 4.32 $\pm$ 0.0096 & 1.72 $\pm$ 0.081 & 1.21 $\pm$ 0.019 & 0.247 $\pm$ 0.02 \\
11295426 & 6.83 $\pm$ 0.68 & 0.0936 $\pm$ 0.032 & 4.28 $\pm$ 0.0079 & 1.64 $\pm$ 0.1 & 1.26 $\pm$ 0.016 & 0.258 $\pm$ 0.018 \\
11401755 & 5.97 $\pm$ 0.44 & 0.0467 $\pm$ 0.014 & 4.04 $\pm$ 0.0056 & 3.23 $\pm$ 0.16 & 1.66 $\pm$ 0.018 & 0.245 $\pm$ 0.016 \\
11807274 & 3.44 $\pm$ 0.69 & 0.236 $\pm$ 0.078 & 4.15 $\pm$ 0.012 & 3.5 $\pm$ 0.26 & 1.58 $\pm$ 0.032 & 0.257 $\pm$ 0.015 \\
11853905 & 6.65 $\pm$ 0.8 & 0.0472 $\pm$ 0.021 & 4.11 $\pm$ 0.0064 & 2.53 $\pm$ 0.19 & 1.57 $\pm$ 0.037 & 0.262 $\pm$ 0.011 \\
11904151 & 10.1 $\pm$ 1.3 & 0.1 $\pm$ 0.038 & 4.35 $\pm$ 0.0098 & 1.09 $\pm$ 0.059 & 1.06 $\pm$ 0.015 & 0.245 $\pm$ 0.013 \\
\enddata
\end{deluxetable}

\begin{figure*}
    \centering
    \includegraphics[width=0.5\linewidth,keepaspectratio]{figs/comparison/kages-Mass.pdf}\hfill
    \includegraphics[width=0.5\linewidth,keepaspectratio]{figs/comparison/kages-Radius.pdf}\\
    \includegraphics[width=0.5\linewidth,keepaspectratio]{figs/comparison/kages-Luminosity.pdf}\hfill
    \includegraphics[width=0.5\linewidth,keepaspectratio]{figs/comparison/kages-logg.pdf}\\
    \includegraphics[width=0.5\linewidth,keepaspectratio]{figs/comparison/kages-Age.pdf}
    \caption{Suggested KAGES masses, radii, luminosities, surface gravities, and ages of \emph{Kepler} objects-of-interest plotted against the predictions made here by machine learning (ML). Medians, 16\% quantiles, and 84\% quantiles are shown for each point. A dashed line of agreement is shown in both plots to guide the eye. The color of each point indicates roughly the sigma of disagreement between both methods, with black points indicating agreement and redder points indicating disagreement. }
    \label{fig:us-vs-them}
\end{figure*}


%%%%%%%%%%%%%%%%%%%%%%%%%%%%%%%%%%%%%%%%%
%%% Discussion %%%%%%%%%%%%%%%%%%%%%%%%%%
%%%%%%%%%%%%%%%%%%%%%%%%%%%%%%%%%%%%%%%%%
\section{Discussion}
The amount of time it takes to predict the fundamental characteristics of a star can be decomposed into two parts: the amount of time it takes to calculate Monte Carlo perturbations to the observations, and the amount of time it takes to make a prediction on each perturbed set of observations. Hence we have
\begin{equation}
    t = n(t_p + t_r)
\end{equation}
where $t$ is the total time, $n$ is the number of perturbations, $t_p$ is the time it takes to perform a single perturbation, and $t_r$ is the random forest regression time. We typically see times of $t_p \simeq 0.009 \si{\s}$ and $t_r \simeq 0.00007 \si{\s}$. Since we use $n=$10,000 in this work, we find in general a time of around a minute per star. Since each star can be processed independently in parallel, a computing cluster could process a catalog containing millions of objects in less than a day's time. Since $t_p >> t_r$, the calculation depends almost entirely on the time it takes to perturb the observations.\footnote{We note that our perturbation code uses an interpreted language (R), so it has room even still for many orders-of-magnitude of speed-up.} There is also a one-time cost of generating the matrix of training data, which took us roughly a day to run. Unless one wants to consider a different range of parameters or different input physics, however, the matrix does not need to be calculated again; this same matrix can be used for all future observations. 

Previously, \citet{pulone1997age} developed a neural network for predicting stellar age based on the star's position in the Hertzsprung-Russell diagram. More recently, \citet{2016arXiv160200902V} have worked on incorporating seismic information into that analysis as we have done here. Our method provides several advantages over both of these. Firstly, the random forests that we use perform constrained regression, meaning that the values we predict for quantities like age and mass will always be non-negative and within the bounds of the training data, which is not true of the neural networks-based approach that they take. Indeed, this can be readily seen by the fact that half of the values \citet{2016arXiv160200902V} predict are outside the bounds of their input grid. Secondly, we use averaged frequency separations which allows us to make predictions on any main-sequence solar-like star irrespective of which radial orders have been observed. In contrast, they consider only radial orders $n=16\ldots 19$, and therefore their network would need to be re-trained for any new star that is observed with a different (or even a sub-) set of radial orders than those. This leads to the next point: we have shown that our random forests are very fast to train, and can be retrained in only seconds for stars that are missing, for example, luminosity information. In contrast, deep neural networks are computationally intensive to train, taking days or even weeks to converge depending on the breadth of network topologies considered in the cross-validation. Their approach therefore provides no additional time benefit over the standard method of \emph{in-situ} optimization over an evolution path, and produces less precise values (see Section 3 of \citealt{2016arXiv160200902V}). Finally, our grid is varied in six parameters -- M, Y$_0$, Z$_0$, $\alpha_{\text{MLT}}$, $\alpha_{\text{ov}}$, and D -- whereas their grid considers a much more limited range of stellar physics. In particular, their grid does not include diffusion or overshooting calculations, and they consider only masses up to 1.1 M$_\odot$. As mentioned, including these higher mass ranges and more intricate physical calculations involves many challenges that we discussed at length in Section 2 and in the Appendices.

%approached this problem by constructing a feed-forward neural network with three hidden layers to calculate stellar age, mass, initial composition and mixing length from frequencies with radial orders $n=16\ldots 19$. Since the radial orders are fixed, their method only works on stars with those particular radial orders observed (i.e.~currently only the Sun, 16 Cyg A, and 16 Cyg B). Their network would need to be re-trained for any new star that is observed with a different (or even a sub-) set of radial orders than those. Deep neural networks are computationally intensive to train, taking days or even weeks to converge depending on the breadth of network topologies considered in the cross-validation. This approach therefore provides no additional benefit in terms of time over the classical approach of \emph{in-situ} optimization over an evolution path, and has been seen to produce less precise values than the traditional method.% since it considers only a subset of the available information. 
%Moreover, as mentioned earlier, neural networks solve an unconstrained rather than a constrained regression problem and can therefore yield non-physical solutions. 

%Random forests of decision tree regressors that are trained to learn the relationships between fundamental parameters and averaged frequency separations and ratios avoid such complications. Since we consider $\nu_{\max}$-weighted median values of observables like the small frequency separation, we are not limited to any particular set of observed radial orders. As such, we are able to process all KAGES stars and regress the fundamental parameters for all of these objects. Furthermore, decision tree regressors learn the boundaries of the parameter space, and therefore will not predict non-physical values like a negative age. In addition, random forests do not need to cross-validate their internal architectures because each tree is constructed independently of each other tree, and additionally provide the opportunity to glean insight about the regression being performed, as we have shown. Finally, the training times of a random forest provides orders-of-magnitude improvement over a deep neural network, discovering the patterns in the inputs in only seconds. %In an unconstrained setting, neural networks could be preferential because of their ability to learn posterior distributions. However, being that inference of stellar parameters is a highly constrained regression problem, random forests are better suited for the task. 

\section{Conclusions}
Here we have considered the constrained multiple-regression problem of inferring fundamental stellar parameters from observations. We created a grid of evolutionary tracks varied in mass, chemical composition, mixing length parameter, diffusion factor, and overshooting coefficient. We evolved them in time along the main sequence and collected classical properties as well as global statistics on the modes of oscillations from models along each evolutionary path. We trained a machine learning algorithm to discern the patterns that relate observations to stellar properties throughout all of the evolutionary tracks. We then applied this method to hare-and-hound exercise data, the Sun, 16 Cyg A and B, and thirty-two stars observed by \emph{Kepler}. We obtained in an instant precise initial stellar conditions as well as current-age values, which are vital for ensemble studies of the Galaxy. The retrodicted initial conditions like the mixing length parameter and overshoot coefficient can furthermore be used as strong priors when performing more detailed stellar modelling, e.g.~when obtaining a reference model for an inversion. Remarkably, we were able to empirically determine the value of the diffusion factor and hence the efficiency of diffusion required to reproduce the observations instead of inhibiting it \emph{ad hoc} for higher-mass stars as is often done. 

We note that these estimates represent a set of choices in stellar physics, for which such a bias is impossible to rightfully propagate. Nevertheless, varying quantities that are usually kept fixed, such as the mixing length parameter, the diffusion factor, and the overshoot coefficient, takes us a step in the right direction. And finally, the fact that quantities such as stellar radii and luminosities -- quantities that have been measured accurately, not just precisely -- can be so faithfully reproduced by this method, gives a degree of confidence in its efficacy. 

The method we have presented here is applicable to solar-like stars on the main sequence. We intend to extend this study to later stages of evolution. Such a task presents novel challenges, however, because the global statistics we have considered here, such as the small frequency separation, are less applicable in those later stages due to their mixed modes of oscillation. Therefore, new approaches and advances be needed for those stars. 

\acknowledgments The research leading to the presented results has received funding from the European Research Council under the European Community's Seventh Framework Programme (FP7/2007-2013) / ERC grant agreement no 338251 (StellarAges). This research was undertaken in the context of the International Max Planck Research School. 

Analysis in this manuscript was performed with python3 libraries scikit-learn \citep{scikit-learn}, NumPy \citep{van2011numpy}, and pandas \citep{mckinney2010data}; the R software package \citep{R}; and the R libraries magicaxis \citep{magicaxis}, RColorBrewer \citep{RColorBrewer}, parallelMap \citep{parallelMap}, data.table \citep{data.table}, lpSolve \citep{lpSolve}, ggplot2 \citep{ggplot2}, GGally \citep{GGally}, scales \citep{scales}, and matrixStats \citep{matrixStats}. 

\appendix

%%%%%%%%%%%%%%%%%%%%%%%%%%%%%%%
%%% Grid strategy %%%%%%%%%%%%%
%%%%%%%%%%%%%%%%%%%%%%%%%%%%%%%
\section{Initial grid strategy}
\label{sec:grid}
The initial conditions of a stellar model can be viewed as a 6-orthotope with dimensions M, Y$_0$, Z$_0$, $\alpha_{\text{MLT}}$, $\alpha_{\text{ov}}$, and D. In most experiments, only a subset of these dimensions are varied with the others either kept fixed at the solar value or turned off completely. Here we construct a grid of stellar models with all quantities varied simultaneously. Instead of varying the quantities in a linear fashion, however, we opt to construct a quasi-random grid of evolutionary tracks. 

A linear grid subdivides all dimensions in which initial quantities can vary into equal parts and creates a track of models for every combination of these subdivisions. Although in the limit such a strategy will eventually fill the hypercube of initial conditions, it does so very slowly. It is furthermore suboptimal in the sense that linear grids maximize redundant information, as each varied quantity is tried with the exact same values of all other parameters that have been considered already. In a high-dimensional setting, if any of the parameters are irrelevant to the task of the computation, then the majority of the tracks in a linear grid will not contribute any new information.

A refinement on this approach is to create a grid of models with \emph{randomly} varied initial conditions. Such a strategy fills the space more rapidly, and furthermore solves the problem of redundant information. However, this approach suffers from a different problem: since the points are generated at random, they tend to ``clump up'' at random as well. This results in random gaps in the parameter space, which are obviously undesirable. 

Therefore, in order to select points that do not stack, do not clump, and also fill the space as rapidly as possible, we generate Sobol numbers \citep{sobol1967distribution} in the unit 6-cube and map them to the parameter ranges of each quantity that we want to vary. Sobol numbers are a sequence of $m$-dimensional vectors $x_1 \ldots x_n$ in the unit hypercube $I^m$ constructed such that the integral of a real function $f$ in that space is equivalent in the limit to that function evaluated on those numbers, that is,
\begin{equation}
    \int_{I^m} f = \lim_{n \to \infty} \frac{1}{n}\sum_{i=1}^n f(x_i)
\end{equation}
with the sequence being chosen such that the convergence is achieved as quickly as possible. By doing this, we both minimize redundant information and furthermore sample the hyperspace of possible stars as uniformly as possible. Figure \ref{fig:grids} visualizes the different methods of generating multidimensional grids: linear, random, and the quasi-random strategy that we took; and Figure \ref{fig:inputs} shows 1- and 2D projection plots of the initial model conditions for all of the evolutionary tracks in our grid. 

\begin{figure*}
    \centering
    \subfloat[Linear]{{\includegraphics[width=0.33\textwidth,keepaspectratio]{figs/grid-linear.png}}}%
    %\hfill
    \subfloat[Random]{{\includegraphics[width=0.33\textwidth,keepaspectratio]{figs/grid-random.png}}}%
    %\hfill
    \subfloat[Quasi-random]{{\includegraphics[width=0.33\textwidth,keepaspectratio]{figs/grid-quasirandom.png}}}%
    \caption{Results of different methods for generating multidimensional grids portrayed via a unit cube projected onto a unit square. Linear (left), random (middle), and quasi-random (right) grids are generated in three dimensions, with color depicting the third dimension, i.e., the distance between the reader and the screen. %Linear grids exhaust each dimension uniformly, but with all points stacked on top of each other, so the unit cube is filled very slowly and each point bears redundant information, which is why all of the (stacked) points appear black. Random grids fill the unit cube more rapidly, but points tend to clump up and leave large gaps in the parameter space. Quasi-random grids achieve the best of both worlds and fill the unit cube most rapidly by generating points that are maximally distant along all dimensions. 
    From top to bottom, all three methods are shown with 100, 400, and 2000 points generated, respectively. }%
    \label{fig:grids}
\end{figure*}

%%%%%%%%%%%%%%%%%%%%%%%%%%%%%%%
%%% Remeshing %%%%%%%%%%%%%%%%%
%%%%%%%%%%%%%%%%%%%%%%%%%%%%%%%
\section{Adaptive Remeshing}
\label{sec:remeshing}

When performing element diffusion calculations in MESA, the surface abundance of each isotope is considered as an average over the outermost cells of the model. The number of cells $N$ is chosen such that the mass of the surface is more than $dq$ times the mass of the $(N+1)^{\text{th}}$ cell. Occasionally, this approach can lead to a situation where surface abundances change dramatically and discontinuously in a single time-step. These abundance discontinuities then propagate as discontinuities in effective temperatures, surface gravities, and radii. An example of such a difficulty can be seen in Figure \ref{fig:discontinuity}. 

\begin{figure}
    \centering
    \subfloat[First iteration]{{\includegraphics[width=0.5\linewidth,keepaspectratio]{figs/discontinuity-1.pdf}}}\\
    \subfloat[Second iteration]{{\includegraphics[width=0.5\linewidth,keepaspectratio]{figs/discontinuity-2.pdf}}}\\
    \subfloat[Third iteration]{{\includegraphics[width=0.5\linewidth,keepaspectratio]{figs/discontinuity-3.pdf}}}%
    \caption{Surface abundance discontinuity detection and iterative remeshing for an evolutionary track in our grid. The detected discontinuities are encircled in red. The third iteration has no discontinuities and so this track is considered to have converged. }
    \label{fig:discontinuity}
\end{figure}

Instead of being a physical reality, these effects arise only when there is insufficient mesh resolution in the outermost layers of the model. We therefore seek to detect these cases and re-run any such evolutionary track with finer mesh resolution. We consider a track an outlier if its surface hydrogen abundance changes by more than 0.01 in a single time-step. We iteratively re-run any track with outliers detected using a finer mesh resolution, and, if necessary, smaller time-steps, until convergence is reached. The process and a resolved track can also be seen in Figure \ref{fig:discontinuity}. 

%To address this problem, we must devise a discontinuity detection scheme. %A function is smooth if its first derivative is continuous. %A jump discontinuity in a function causes an extreme value to occur in that function's first derivative. 
%Hence, we first fit cubic splines to the surface hydrogen abundance $\text{X}_{\text{surf}}$ as a function of the star's age $\tau$. We then twice differentiate the spline along a million equally-spaced points in stellar age and consider whether any second derivatives are outliers. We define an outlier using robust statistics as any point whose absolute difference with the median is more than a million times the median absolute deviation (MAD) of the local derivatives, that is, if

%\begin{equation}
%    \abs{
%        \dv[2]{\text{X}_{\text{surf}}}{\tau} - 
%        \text{median}\qty( \dv[2]{\text{X}_{\text{surf}}}{\tau} ) 
%    } > 10^6 \cdot \text{MAD}\qty( \dv[2]{\text{X}_{\text{surf}}}{\tau} ) 
%\end{equation}
%where
%\begin{equation}
%    \text{MAD} \equiv \text{median}_{i}\left(\ \left| \text{X}_{i} - \text{median}_{j} (\text{X}_{j}) \right|\ \right).
%\end{equation}


%For the purposes of calculating gravitational settling on the near-surface layer, the surface is considered as not only the outermost cell, but also by accumulating the outermost cells of 

%%%%%%%%%%%%%%%%%%%%%%%%%%%%%%%
%%% Model selection %%%%%%%%%%%
%%%%%%%%%%%%%%%%%%%%%%%%%%%%%%%
\section{Model selection}
\label{sec:selection}
In order to prevent statistical bias towards the evolutionary tracks that generate the most models, i.e.~the ones that require the most careful calculations and therefore use smaller time-steps, or those that live on the main sequence for a longer amount of time; we select $N=100$ models from each evolutionary track such that the models are as evenly-spaced in core-hydrogen abundance as possible. 

Starting from the original vector of length $m$ of core-hydrogen abundances $\vec X$, we find the subset of length $m$ that is closest to the optimal spacing $\vec B$, where
\begin{equation}
  \vec B \equiv \qty[
    X_T, 
    \ldots,
    %\min\qty(\vec H)+\frac{\max\qty(\vec H)-\min\qty(\vec H)}{99},
    \frac{(m-i)\cdot X_T + X_Z}{m-1}, 
    \ldots, 
    X_Z
  ]
\end{equation}
with $X_Z$ being the core-hydrogen abundance at zero-age (ZAMS) and $X_T$ being that at the end of the star's main-sequence lifetime (TAMS). To obtain the closest possible vector to $\vec B$ from our data $\vec X$, we solve a transportation problem using integer optimization \citep{23145595}. First we set up a cost matrix $\boldsymbol{C}$ consisting of absolute differences between the original abundances $\vec X$ and the ideal abundances $\vec B$:
\begin{equation}
  \boldsymbol C \equiv 
  \begin{bmatrix}
    \abs{B_1-X_1} & \abs{B_1-X_2} & \dots & \abs{B_1-X_n} \\ 
    \abs{B_2-X_1} & \abs{B_2-X_2} & \dots & \abs{B_2-X_n} \\ 
    \vdots & \vdots & \ddots & \vdots\\ 
    \abs{B_m-X_1} & \abs{B_m-X_2} & \dots & \abs{B_m-X_n}
  \end{bmatrix}.
\end{equation}
We then require that exactly $m$ values are selected from $\vec X$, and that each value is selected no more than one time. Simply selecting the closest data point to each ideally-separated point will not work because this could result in the same point being selected twice; and selecting the second closest point in that situation does not remedy it because a different result would be had if the points were chosen in a different order. 

We call the optimal solution matrix by $\hat{\boldsymbol{S}}$, and find it by minimising the cost matrix subject to the following constraints:
\begin{align}
  \hat{\boldsymbol{S}} = \underset{\boldsymbol S}{\arg\min} \; & \sum_{ij} S_{ij} C_{ij} \notag\\
  \text{subject to } & \sum_j S_{ij} \leq 1 \; \text{ for all } i=1\ldots N \notag\\
  \text{and } & \sum_i S_{ij} = 1 \; \text{ for all } j=1\ldots M.
  \label{eq:optimal-spacing}
\end{align}
The indices of $\vec X$ that are most near to being equidistantly-spaced are then found by looking at which columns of $\boldsymbol S$ contain ones, and we are done. The solution is visualized in Figure \ref{fig:nearly-even}.

\begin{figure*}
    \centering
    \includegraphics[width=\linewidth, keepaspectratio]{figs/nearly-even.pdf}
    \caption{ A visaulization of the model selection process performed on each evolutionary track in order to obtain the same number of models from each track. The blue crosses show all of the models along the evolutionary track as they vary from ZAMS to TAMS in core-hydrogen abundance and the red crosses show the models selected from this track. The models were chosen via linear transport such that they satisfy Equation \ref{eq:optimal-spacing}. For reference, an equidistant spacing is shown with black points. }% 
    \label{fig:nearly-even}
\end{figure*}

%%%%%%%%%%%%%%%%%%%%%%%%%%%%%%%
%%% Hare-and-Hound %%%%%%%%%%%%
%%%%%%%%%%%%%%%%%%%%%%%%%%%%%%%
\section{Hare-and-Hound}
\label{sec:hare-and-hound}
Table \ref{tab:hnh-true} lists the true values of the Hare-and-Hound exercise performed here, and Table \ref{tab:hnh-perturb} lists the perturbed inputs that were supplied to the machine learning algorithm. 

\begin{deluxetable}{ccccccccccc}
\tabletypesize{\scriptsize}
%\rotate
\tablecaption{True values for the Hare-and-Hound exercise. \label{tab:hnh-true}}
\tablewidth{0pt}
\tablehead{\colhead{Model} & \colhead{R/R$_\odot$} & \colhead{M/M$_\odot$} & \colhead{$\tau$} & \colhead{T$_{\text{eff}}$} & \colhead{L/L$_\odot$} & \colhead{[Fe/H]} & \colhead{Y$_0$} & \colhead{$\nu_{\max}$} & \colhead{$\alpha_{\text{ov}}$} & \colhead{D}}\startdata
0 & 1.705 & 1.303 & 3.725 & 6297.96 & 4.11 & 0.03 & 0.2520 & 1313.67 & 0 & 0 \\
1 & 1.388 & 1.279 & 2.608 & 5861.38 & 2.04 & 0.26 & 0.2577 & 2020.34 & 0 & 0 \\
2 & 1.068 & 0.951 & 6.587 & 5876.25 & 1.22 & 0.04 & 0.3057 & 2534.29 & 0 & 0 \\
3 & 1.126 & 1.066 & 2.242 & 6453.57 & 1.98 & -0.36 & 0.2678 & 2429.83 & 0 & 0 \\
4 & 1.497 & 1.406 & 1.202 & 6506.26 & 3.61 & 0.14 & 0.2629 & 1808.52 & 0 & 0 \\
5 & 1.331 & 1.163 & 4.979 & 6081.35 & 2.18 & 0.03 & 0.2499 & 1955.72 & 0 & 0 \\
6 & 0.953 & 0.983 & 2.757 & 5721.37 & 0.87 & -0.06 & 0.2683 & 3345.56 & 0 & 0 \\
7 & 1.137 & 1.101 & 2.205 & 6378.23 & 1.92 & -0.31 & 0.2504 & 2483.83 & 0 & 0 \\
8 & 1.696 & 1.333 & 2.792 & 6382.22 & 4.29 & -0.07 & 0.2555 & 1348.83 & 0 & 0 \\
9 & 0.810 & 0.769 & 9.705 & 5919.70 & 0.72 & -0.83 & 0.2493 & 3563.09 & 0 & 0 \\
10 & 1.399 & 1.164 & 6.263 & 5916.71 & 2.15 & 0.00 & 0.2480 & 1799.10 & 0.2 & 1 \\
11 & 1.233 & 1.158 & 2.176 & 6228.02 & 2.05 & 0.11 & 0.2796 & 2247.53 & 0.2 & 1 \\
\enddata
\end{deluxetable}

\begin{deluxetable}{ccccc}
\tabletypesize{\scriptsize}
%\rotate
\tablecaption{Supplied (perturbed) inputs for the Hare-and-Hound exercise. \label{tab:hnh-perturb}}
\tablewidth{0pt}
\tablehead{\colhead{Model} & \colhead{T$_{\text{eff}}$} & \colhead{L/L$_\odot$} & \colhead{[Fe/H]} & \colhead{$\nu_{\max}$}}\startdata
0 & 6236.96 $\pm$ 85 & 4.23 $\pm$ 0.12 & -0.03 $\pm$ 0.09 & 1397.87 $\pm$ 66 \\
1 & 5806.39 $\pm$ 85 & 2.1 $\pm$ 0.06 & 0.16 $\pm$ 0.09 & 2031.82 $\pm$ 100 \\
2 & 5884.56 $\pm$ 85 & 1.23 $\pm$ 0.04 & -0.05 $\pm$ 0.09 & 2625.73 $\pm$ 127 \\
3 & 6421.65 $\pm$ 85 & 1.99 $\pm$ 0.06 & -0.36 $\pm$ 0.09 & 2475.43 $\pm$ 124 \\
4 & 6525.9 $\pm$ 85 & 3.73 $\pm$ 0.11 & 0.14 $\pm$ 0.09 & 1752.33 $\pm$ 89 \\
5 & 6117.96 $\pm$ 85 & 2.18 $\pm$ 0.06 & 0.04 $\pm$ 0.09 & 1890.79 $\pm$ 101 \\
6 & 5740.71 $\pm$ 85 & 0.82 $\pm$ 0.03 & 0.06 $\pm$ 0.09 & 3486.37 $\pm$ 165 \\
7 & 6288.63 $\pm$ 85 & 1.95 $\pm$ 0.06 & -0.28 $\pm$ 0.09 & 2438.7 $\pm$ 124 \\
8 & 6351.2 $\pm$ 85 & 4.28 $\pm$ 0.13 & -0.12 $\pm$ 0.09 & 1294.24 $\pm$ 67 \\
9 & 5997.95 $\pm$ 85 & 0.7 $\pm$ 0.02 & -0.85 $\pm$ 0.09 & 3292.27 $\pm$ 179 \\
10 & 5899.27 $\pm$ 85 & 2.17 $\pm$ 0.06 & -0.031 $\pm$ 0.09 & 1931.63 $\pm$ 101 \\
11 & 6251.49 $\pm$ 85 & 1.99 $\pm$ 0.06 & 0.126 $\pm$ 0.09 & 2356.977 $\pm$ 101 \\
\enddata
\end{deluxetable}


\bibliographystyle{apj.bst}
\bibliography{astero}


\end{document}

