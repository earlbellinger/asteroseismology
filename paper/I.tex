%% This is emulateapj reformatting of the AASTEX sample document
%%
\documentclass[iop,apj,twocolappendix]{emulateapj}
%\documentclass[manuscript]{aastex}

\newcommand{\vdag}{(v)^\dagger}
\newcommand{\myemail}{bellinger@mps.mpg.de}

\usepackage{graphicx}	% Including figure files
\usepackage{subfig}     % For subcaptions 
\usepackage{amsmath}	% Advanced maths commands
\usepackage{amssymb}	% Extra maths symbols
\usepackage{bm}		    % Bold maths symbols, including upright Greek

\usepackage{mathrsfs}
\usepackage{siunitx}    % SI units 

\usepackage{bookmark}   % Hyperlinking to sections etc 

\usepackage{physics}    % Calculus notation 
\usepackage{natbib}     % Bibliography

\usepackage{lineno}
%\linenumbers

\usepackage{tikz}       % Drawing the RF figure 
\usetikzlibrary{arrows,positioning,shapes,decorations.markings}

\shorttitle{Stellar parameters in an instant with machine learning}
\shortauthors{Bellinger \& Angelou et al.}

\begin{document}

\title{Fundamental Parameters of Main-Sequence Stars in an Instant with Machine Learning}

\author{Earl P. Bellinger\altaffilmark{1,2,3}, George C. Angelou\altaffilmark{1,2}, Saskia Hekker\altaffilmark{1,2}, Sarbani Basu\altaffilmark{4}, Warrick H. Ball\altaffilmark{5}, and Elisabeth Guggenberger\altaffilmark{1,2}}
\affil{\altaffilmark{1} Max-Planck-Institut f\"{u}r Sonnensystemforschung, Justus-von-Liebig-Weg 3, 37077 G\"{o}ttingen, Germany\\
\altaffilmark{2} Stellar Astrophysics Centre, Department of Physics and Astronomy, Aarhus University, Ny Munkegade 120, DK-8000 Aarhus C, Denmark \\
\altaffilmark{3} Institut f\"ur Informatik, Georg-August-Universit\"at G\"ottingen, Goldschmidtstrasse 7, 37077 G\"ottingen, Germany \\
\altaffilmark{4} Department of Astronomy, Yale University, New Haven, CT 06520, USA \\
\altaffilmark{5} Institut f\"ur Astrophysik G\"ottingen, Friedrich-Hund-Platz 1, 37077 G\"ottingen, Germany}

\begin{abstract}
Owing to the remarkable photometric precision that is being provided by modern space observatories, stellar and planetary systems other than our own are now beginning to be characterized en masse for the first time. These characterizations are pivotal for endeavors such as searching for Earth-like planets and solar twins, understanding the processes that comprise stellar evolution, and tracing the dynamics of our Galaxy. This volume of data that is becoming available, however, brings with it the need to process this information accurately and rapidly. While existing methods are able to constrain stellar properties from these observations, they require substantial computational efforts to do so. 

We develop machine learning methods for rapidly estimating fundamental stellar parameters such as the ages, masses, and radii of main-sequence solar-like stars from classical and asteroseismic observations. We first demonstrate these methods on a hare-and-hound exercise and then apply them to the Sun, 16 Cyg A \& B, and thirty-three planet-hosting candidates that have been observed by the Kepler spacecraft. We find that our estimates and their associated uncertainties are comparable to the results of other methods, but with the additional benefit of needing practically zero computation time. Our method is open source and freely available for the community to use.\footnote{The source code for all analyses and for all figures appearing in this manuscript can be found electronically at \url{https://github.com/earlbellinger/asteroseismology}.}
\end{abstract}


\keywords{methods: statistical --- stars: abundances --- stars: fundamental parameters --- stars: low-mass --- stars: oscillations --- stars: solar-type}


%%%%%%%%%%%%%%%%%%%%%%%%%%%%%%%%%%%%%%%%%%%%%%%%%%
%%%%%%%%%%%%%%%%% BODY OF PAPER %%%%%%%%%%%%%%%%%%
\section{Introduction}
In recent years, dedicated photometric space missions have delivered dramatic improvements in the observations of Sun-like stars. These improvements come not only in terms of their accuracy, but also in their frequency of observation, which enables the direct measurement of dynamical stellar phenomena. Such measurements place strong constraints on the ages, masses, and chemical compositions of these stars, which in turn drives a wide range of applications in astrophysics, such as testing theories of stellar evolution, characterizing extrasolar planetary systems \citep[e.g.][]{2015ApJ...799..170C,2015MNRAS.452.2127S}, assessing galactic chemical evolution \citep[e.g.][]{2015ASSP...39..111C}, and performing ensemble studies of the Galaxy \citep[e.g.][]{2011Sci...332..213C, 2013MNRAS.429..423M, 2014ApJS..210....1C}. 

The drive to increase photometric accuracy has in part been driven by the goal of measuring oscillation modes in stars that are like our Sun. This is due to the fact that asteroseismology, the study of these oscillations, provides the opportunity to constrain the ages of stars through accurate inferences of their interior structures. However, seismic ages cannot be measured directly. Instead, they depend on indirect determinations via stellar model calculations. 

To determine the age of a star via seismic analysis, optimization procedures seek the model that best matches the available observations \citep{1994ApJ...427.1013B}. 
Several search strategies have been employed, including exploration through a pre-computed grid of models (i.e.\ grid-based modelling, hereafter GBM; see \citealt{2011ApJ...730...63G, 2014ApJS..210....1C}); or \emph{in situ} optimization (hereafter ISO) such as genetic algorithms \citep{2014ApJS..214...27M}, Markov-chain Monte Carlo \citep{2012MNRAS.427.1847B}, or the downhill simplex algorithm \citep{2013ApJS..208....4P}; see e.g.\ \citealt{2015MNRAS.452.2127S} for an extended discussion. Utilizing the precision and long temporal observations from the Kepler and CoRoT space telescopes, these methods have constrained stellar ages of field stars to within 10\% of their main-sequence lifetime \citep{2015MNRAS.452.2127S}. 

ISO and GBM are computationally intensive, however, both requiring the calculation of a large number of stellar models (see \citealt{2009ApJ...699..373M} for a discussion).
ISO-based methods require that new stellar tracks are calculated for every target as they cannot predict \emph{a priori} what combination of stellar parameters the optimizer will need for each successive iteration. They furthermore converge to local minima and therefore in theory should be run multiple times from different starting points to ensure global coverage. %There is no way to be certain that the global minima has been located and these methods should in theory be run multiple times to ensure convergence. 
%This can somewhat be mitigated by storing each computed track, in essence combining optimization with a grid based search. 
%Adaptive time-stepping combined with minimization can direct resolution to the domain around any minima ensuring regions of interest are well sampled. There is however no way to be certain that the global minima has been located and these methods should in theory be run multiple times to ensure convergence. 
GBM by means of interpolation in a high-dimensional parameter space, on the other hand, is sensitive to the resolution of each parameter. This then necessitates very fine grids that cover the whole search domain. Additional dimensions such as efficiency parameters (e.g.\@, overshooting or mixing length parameters) significantly impact on the number of models needed and hence the search times for these methods. 
%Furthermore, any method that fits stellar models by $\chi^2$ techniques assume a linear relation between the observations and the free parameters of their search, and also assume normally-distributed error functions. Such assumptions can cause uncertainties to be underestimated \citep{2010arXiv1012.3754A}.   



%Such methods furthermore depend on the choices in physics used to construct models and the uncertainties therein \citep{2014A&A...569A..21L} and require new models to be run every time 
 %Adaptive time-stepping and minimization means that large parts of the parameter space need not be considered and thus resolution is optimal in the necessary search domain. To obtain the same resolution in the region of interest as ISO,  GBM require dense grids in temperal resolution (successive stellar models for every track) and in terms of sampling of input parameters. Both methods require that models be rerun when varying efficiency parameters such as (OS and mixing length parameters).  In addition, those methods that fit stellar models by chi sq techniques assume a linear relation between the observations and the free parameters of their search, and also assume normally-distributed error functions. Such assumptions can cause uncertainties to be underestimated \citep{2010arXiv1012.3754A}.%Interpolation in a high-dimensional parameter space is also sensitive to the resolution of each parameter, which then requires very fine grids. 

These concessions have been made because the relationships connecting observable properties of stars to their internal attributes are non-linear and difficult to invert. Here we will show that through the use of machine learning, it is possible to avoid these difficulties by capturing those relations statistically and using them to construct a model that is capable of relating observations of stars to their structural and evolutionary properties. These learned relationships can then be utilized to process entire catalogs with a cost of only seconds per star. 

To date, only a modest number of solar-like stars have had their frequencies resolved, allowing each of them to be modelled in detail using expensive techniques. In the forthcoming era of TESS \citep{2015JATIS...1a4003R} and PLATO \citep{2014ExA....38..249R}, however, seismic data for millions of stars will become available, and it will not be possible to dedicate large amounts of supercomputing time on every star. Furthermore, for most stars, it may only be possible to resolve global asteroseismic quantities rather than individual frequencies. Therefore, the ability to rapidly process large volumes of these global quantities will be paramount. 

In this work, we consider the constrained multiple-regression problem of inferring fundamental stellar properties from observable parameters. We construct a random forest of decision tree regressors to learn the relationships connecting observable quantities to zero-age main-sequence (ZAMS) histories and current-age structural and chemical attributes. We validate our technique by inferring the parameters of a hare-and-hound exercise, the Sun, and the well-studied stars 16 Cyg A and B. Finally, we conclude by applying our method on a catalog of Kepler objects-of-interest (hereafter KOI, \citealt{2016MNRAS.456.2183D}).

The method presented here has many advantages over existing approaches. First, random forests can be trained and used in only seconds and hence provide substantial speed-ups over other methods. 
Secondly, random forests perform non-linear and non-parametric regression, which %-- as opposed to linear interpolation -- which frees us from the strong assumptions laden to other methods and therefore 
allows us to appraise the uncertainties of predicted quantities in a more rigorous manner. 
Thirdly, our method allows us to investigate wide ranges and combinations of stellar parameters, whereas other approaches typically use for example the solar mixing length or a fixed amount of core overshooting, which additionally results in an underestimates of uncertainty. This is especially important in the case of atomic diffusion, which is essential when modelling the Sun \citep[see e.g.][]{1994MNRAS.269.1137B}, but is usually turned off for stars with M/M$_\odot > 1.4$ because it leads to the unobserved consequence of a hydrogen-only surface \citep{2002A&A...390..611M}. Instead of turning it off \emph{ad-hoc} as is commonly done, our method empirically determines the efficiency of atomic diffusion that is required to reproduce observations on a star-by-star basis. 
And finally, the method presented here provides the opportunity to extract insights from the statistical regression that is being performed. This contrasts the blind optimization processes of other methods that provide an answer but do not indicate the elements that were important in doing so. 

We have currently developed this method for main-sequence solar-like stars, whose modes of oscillation are known to offer tight constraints on the stellar interior. We explore various model physics by considering stellar evolutionary tracks that are varied not only in their initial mass and chemical composition, but also in their efficiency of convection, extent of core overshooting, and strength of gravitational settling. We compare our results to the recent findings from GBM \citep{2015MNRAS.452.2127S}, ISO \citep{2015ApJ...811L..37M}, interferometry \citep{2013MNRAS.433.1262W}, and asteroseismic glitch analyses \citep{2014ApJ...790..138V} and find that we obtain similar estimates but with orders-of-magnitude speed-ups in calculation times. 


%%%%%%%%%%%%%%%%%%%%%%%%%%%%%%%%%%%%%%%%%
%%% Grid %%%%%%%%%%%%%%%%%%%%%%%%%%%%%%%%
%%%%%%%%%%%%%%%%%%%%%%%%%%%%%%%%%%%%%%%%%
\section{Method} \label{sec:Method} 
Training a machine learning algorithm for the purpose of characterizing observed stars requires a matrix of evolutionary models from which the machine can learn the processes that relate observable quantities to model properties. In order to construct such a matrix, models along evolutionary sequences are extracted and homogenized to yield the same types of information as the stars being observed. Once the network has learned the non-linear relationships present in the simulations, one can feed the algorithm a catalogue of observational data and obtain estimates of the desired model properties for that star. 

The types of observations used to inform the algorithm include, but are not limited to, combinations of temperatures, metallicities, frequency separations, surface gravities, luminosities, and/or radii. A trained machine can then infer stellar properties calculated from stellar models such as ages, masses, core-hydrogen and surface-helium abundances. If luminosities, surface gravities, or radii are not supplied, then they may be predicted as well. In addition, the machine can also infer evolutionary parameters such as the initial stellar mass and initial chemical compositions as well as the mixing length parameter, overshoot coefficient, and diffusion factor needed to reproduce observations, which are explained in detail below. %The determined distributions of these values can later be used as strong priors when constructing more detailed models of those stars. 


\subsection{Model generation}
\label{sec:models}
We use the open-source 1D stellar evolution code \emph{Modules for Experiments in Stellar Astrophysics} \citep[MESA,][]{Paxton2011} to generate main-sequence stellar models from solar-like evolutionary tracks varied in initial mass M, helium Y$_0$, metallicity Z$_0$, mixing length parameter $\alpha_{\text{MLT}}$, overshoot coefficient $\alpha_{\text{ov}}$, and atomic diffusion factor D. The diffusion factor serves to amplify or diminish the effects of diffusion, where a value of zero would turn it off and a value of two would double all coefficients. The initial conditions are varied in the ranges M $\in [0.7, 1.6]$ M$_\odot$, Y$_0$ $\in [0.22, 0.34]$, Z$_0$ $\in [10^{-5}, 10^{-1}]$ (varied logarithmically), $\alpha_{\text{MLT}}$ $\in [1.5, 2.5]$, $\alpha_{\text{ov}}$ $\in [10^{-4}, 1]$ (varied logarithmically), and D $\in [10^{-6}, 10^2]$ (varied logarithmically). We furthermore put a cut-off of 10$^{-3}$ and 10$^{-5}$ on $\alpha_{\text{ov}}$ and D, respectively, below which we consider them to be zero and turn them off. The initial parameters of each track at the ZAMS are chosen in a quasi-random fashion so as to populate the initial-condition hyperspace as homogenously and rapidly as possible (shown in Figure \ref{fig:inputs}; see Appendix \ref{sec:grid} for more details). Note that the initial hydrogen abundance X$_0$ is also varied independently with respect to each parameter with the exception of Y$_0$ and Z$_0$, with which it is required to fulfill baryonic conservation; i.e., X+Y+Z=1. 

We use MESA version r8118 with the Helmholtz-formulated equation of state that allows for radiation pressure and interpolates within the 2005 update of the OPAL EOS tables \citep{2002ApJ...576.1064R}. We assume a \citet{1998SSRv...85..161G} solar composition for our initial abundances and opacity tables. Since we restrict our study to the main sequence, we use an eight-isotope nuclear network consisting of $^1$H, $^3$He, $^4$He, $^{12}$C, $^{14}$N, $^{16}$O, $^{20}$Ne, and $^{24}$Mg. We use a step function for overshooting and set an $f_0 = \alpha_{\text{ov}}/5$ to determine the radius $r_0 = H_p \cdot f_0$ inside the convective zone at which convection switches to overshooting, where $H_p$ is the pressure scale height. %We opt for a classical formulation of overshooting rather than more recent implementations with exponential decay.
All pre-main-sequence (PMS) models are calculated with a simple photospheric approximation, after which an Eddington T-$\tau$ atmosphere is appended on the models at ZAMS. We call ZAMS the point at which the nuclear luminosity of the models make up 99.9\% of the total luminosity. We calculate atomic diffusion during the main sequence using five diffusion class representatives: $^1$H, $^3$He, $^4$He, $^{16}$O, and $^{56}$Fe.\footnote{The atomic number of each representative isotope is used to calculate the diffusion rate of the other isotopes allocated to that group; see \citealt{Paxton2011}.} 
Following their most recent measurements, we correct the defaults in MESA of the gravitational constant ($G=6.67408\times 10^{-8}$ \si{\per\g\cm\cubed\per\square\s}; \citealt{2015arXiv150707956M}), the gravitational mass of the Sun (M$_\odot = 1.988475\times 10^{33}$ \si{\g} $= \mu G^{-1} = 1.32712440042\times 10^{11}$ \si{\km\per\s} $G^{-1}$, where $\mu$ is the standard gravitational parameter; \citealt{pitjeva2015determination}), and the solar radius (R$_\odot = 6.95568\times 10^{10}$ \si{\cm}; \citealt{2008ApJ...675L..53H}). 

Each track is evolved from ZAMS to either an age of $\tau=16$ Gyr or until terminal-age main sequence (TAMS), which we define as having a fractional core-hydrogen abundance (X$_{\text{c}}$) below $10^{-3}$. Evolutionary tracks with efficient heavy-element settling can develop discontinuities in their surface abundances if they lack sufficient model resolution. We implement adaptive remeshing by recomputing any track with abundance discontinuities in its surface layers using finer spatial and temporal resolutions (see Appendix \ref{sec:selection} for details). Running stellar physics codes in a batch mode like this requires care, so we manually inspect the evolutionary tracks to ensure that proper convergence has been achieved. %Hertzsprung-Russell, Kippenhahn, and Christensen-Dalsgaard diagrams of the evolutionary tracks to ensure that proper convergence has been achieved. 



%This occurs as a result of models having efficient diffusion requiring finer spatial and temporal resolution than those models without efficient diffusion (see Appendix \ref{sec:remeshing} for details). In order to prevent biases towards any particular run, we select the same number of models from each evolutionary track (see Appendix \ref{sec:selection} for details). Running stellar physics codes in a batch mode like this requires care, so we manually inspect the evolutionary tracks to ensure that proper convergence has been achieved. %Hertzsprung-Russell, Kippenhahn, and Christensen-Dalsgaard diagrams of the evolutionary tracks to ensure that proper convergence has been achieved. 

\begin{figure*}
    \centering
    \includegraphics[width=\textwidth,keepaspectratio]{figs/grids/inputs.png}
    \caption{Scatterplot matrix (lower panels) and density plots (diagonal) of evolutionary track initial conditions considered. Mass (M), initial helium (Y$_0$), initial metallicity (Z$_0$), mixing length parameter ($\alpha_{\text{MLT}}$), overshoot ($\alpha_{\text{ov}}$), and diffusion factor (D) were varied in a quasi-random fashion to obtain a low-discrepancy grid of model tracks. Points are colored by their initial hydrogen X$_0=1-$Y$_0-$Z$_0$, with blue being low X$_0$ ($\sim 62\%$) and black being high X$_0$ ($\sim 78\%$). The parameter space is densely populated with evolutionary tracks of maximally-different initial conditions. }
    \label{fig:inputs}
\end{figure*}


%%%%%%%%%%%%%%%%%%%%%%%%%%%%%%%
%%% Seismology %%%%%%%%%%%%%%%%
%%%%%%%%%%%%%%%%%%%%%%%%%%%%%%%
\subsection{Calculation of Seismic Parameters}
\label{sec:seis}
We use the ADIPLS pulsation package \citep{2008Ap&SS.316..113C} to compute p-mode oscillations up to spherical degree $\ell=3$ below the acoustic cut-off frequency. We use on average around 4,000 points per stellar model and therefore have adequate resolution to calculate frequencies without remeshing. We denote any frequency separation $S$ as the difference between a frequency $\nu$ of spherical degree $\ell$ and radial order $n$ and another frequency, that is:
\begin{equation} 
  S_{(\ell_1, \ell_2)}(n_1, n_2) \equiv \nu_{\ell_1}(n_1) - \nu_{\ell_2}(n_2).
\end{equation}
The large frequency separation is then
\begin{equation} 
  \Delta\nu_\ell(n) \equiv S_{(\ell, \ell)}(n, n-1)
\end{equation}
and the small frequency separation is
\begin{equation}
  \delta\nu_{(\ell, \ell+2)}(n) \equiv S_{(\ell, \ell+2)}(n, n-1).
\end{equation}
Near-surface layers of a star are model, and frequency-dependent effects are introduced in their approximation \citep[see e.g.][]{1999A&A...351..689R}. The ratios between the large and small frequency separations (Equation \ref{eqn:LSratio}), and also between the large frequency separation and five-point-averaged frequencies (Equation \ref{eqn:rnl}) have been shown to be less sensitive to the poorly-modelled surface term than the aforementioned separations and are therefore valuable asteroseismic diagnostics of stellar interiors \citep{2003A&A...411..215R}. They are defined as
\begin{equation} 
  \mathrm{r}_{(\ell,\ell+2)}(n) \equiv \frac{\delta\nu_\ell(n)}{\Delta\nu_{(1-\ell)}(n+\ell)} \label{eqn:LSratio}
\end{equation}
%and
\begin{equation} 
  \mathrm{r}_{(\ell, 1-\ell)}(n) \equiv \frac{\mathrm{dd}_{(\ell,1-\ell)}(n)}{\Delta\nu_{(1-\ell)}(n+\ell)} \label{eqn:rnl}
\end{equation}
where
\begin{align} 
  \mathrm{dd}_{0,1} \equiv \frac{1}{8} \big[&\nu_0(n-1) - 4\nu_1(n-1) \notag\\
                                 &+6\nu_0(n) - 4\nu_1(n) + \nu_0(n+1)\big]\\ 
  \mathrm{dd}_{1,0} \equiv -\frac{1}{8} \big[&\nu_1(n-1) - 4\nu_0(n) \notag\\
                                 &+6\nu_1(n) - 4\nu_0(n+1) + \nu_1(n+1)\big].
\end{align}
Since the set of radial orders that are observable differs from star to star, we collect global statistics on $\Delta\nu_0$, $\delta\nu_{0,2}$, $\delta\nu_{1,3}$, $r_{0,2}$, $r_{1,3}$, $r_{0,1}$, and $r_{1,0}$. We mimic the range of observable frequencies in our models by weighting all frequencies by their position in a Gaussian envelope centered at the predicted frequency of maximum oscillation power $\nu_{\max}$ and having full-width at half-maximum of $0.66\cdot\nu_{\max}{}^{0.88}$ as per the prescription given by \citet{2012A&A...537A..30M}. We then calculate the weighted median of each variable, which we denote with angled parenthesis (e.g.\ $\langle r_{0,2}\rangle$). We choose the median rather than the mean because it is a robust statistic with a high breakdown point, meaning that it is much less sensitive to the presence of outliers (for a discussion of breakdown points, see \citealt{hampel1971general}, who attributed them to Gauss). This approach allows us to predict the fundamental parameters of any solar-like oscillator with multiple observed modes irrespective of which exact radial orders have been detected. Illustrations of the methods used to derive the frequency separations and ratios of a stellar model are shown in Figure \ref{fig:ratios}. 

\begin{figure*}
    \centering
    \includegraphics[width=0.5\textwidth,keepaspectratio]{figs/separations/solar-test-Dnu0.pdf}\hfill
    \includegraphics[width=0.5\textwidth,keepaspectratio]{figs/separations/solar-test-dnu02.pdf}\\
    \includegraphics[width=0.5\textwidth,keepaspectratio]{figs/separations/solar-test-r_avg01.pdf}\hfill
    \includegraphics[width=0.5\textwidth,keepaspectratio]{figs/separations/solar-test-r_sep02.pdf}%
    \caption{The large and small frequency separations $\Delta\nu_0$ and $\delta\nu_{0,2}$ (top) and frequency ratios $r_{0,1}$ and $r_{0,2}$ (bottom) of a stellar model shown as a function of frequency. The vertical dotted line indicates $\nu_{\max}$. A $\nu_{\max}$-weighted linear fit is indicated with a dashed diagonal line to guide the eye. Point sizes and colors are proportional to the applied weighting, with blue points having a large influence and red points having little. }%
    \label{fig:ratios}
\end{figure*}

\subsection{Training the Random Forest} \label{sec:forest}
We train a random forest regressor on our matrix of evolutionary models to discover the relations that facilitate inference of fundamental stellar parameters from observable quantities. A schematic representation of the topology of our random forest regressor can be seen in Figure \ref{fig:rf}. There are several good textbooks that discuss random forests; see for example Chapter 15 of Elements of Statistical Learning \citep{hastie2005elements}. We choose random forests over any of the many other non-linear regression routines (e.g.\ Gaussian processes, symbolic regression, neural networks, support vector regression, etc.) for several reasons. First, random forests perform \emph{constrained} regression; that is, they only make predictions within the boundaries of the supplied training data \citep[see e.g.\ section 9.2.1 of][]{hastie2005elements}. This is in contrast to other methods like neural networks, which ordinarily perform unconstrained regression and are therefore not prevented from predicting non-physical quantities such as negative masses or from violating conservation requirements. Secondly, due to the decision rule process explained below, random forests are insensitive to the scale of the data. Unless care is taken, other regression methods will artificially weight some properties like temperature as being more important than, say, luminosity, solely because temperatures are written using larger numbers (e.g.\ 5777 vs.\ 1, see for example section 11.5.3 of \citealt{hastie2005elements} for a discussion). Thirdly, random forests take only seconds to train, which can be a large benefit if different stars have different types of observations available (e.g.\ some stars have luminosity information available whereas others do not; a different forest must be trained for each). And finally, random forests provide the opportunity to extract insight about the actual regression being performed by examining the importances assigned to each observable quantity, which we explore in detail below. 

\begin{figure}
    \centering
    \input{figs/random_forest2.tex}
    \caption{A schematic representation of a random forest regressor for inferring fundamental stellar parameters. Classical observables such as temperature and asteroseismic observables like $\delta\nu_{0,2}$ are input on the left side. These quantities are then fed through to some number of hidden decision trees, which each independently predict attributes like age and mass. The predictions are then averaged and output on the right side. %Frequency separations and ratios include separate nodes for $\nu_{\max}$-centered weighted medians of $\delta\nu_{0,2}$, $r_{0,2}$, $r_{1,0}$, $r_{0,1}$, and, if they are measured, $\delta\nu_{1,3}$ and $r_{1,3}$.\\
    All inputs and outputs are optional. For example, surface gravities, luminosities, and radii are not always available (e.g.\ with the KOI stars); so in their absence, these quantities can be predicted instead of being supplied. In this case, those nodes can be moved over to the ``prediction'' side instead of being on the ``observations'' side. Also, in addition to potentially unobserved inputs like stellar radii, other interesting model properties can be predicted as well, such as core-hydrogen mass fraction and surface helium. }
    \label{fig:rf}
\end{figure}


\subsubsection{Feature importances}
\label{sec:importances}
A random forest is an ensemble regressor, meaning that it is composed of many individual components that each perform regression, and the forest subsequently averages over the results from each component \citep{breiman2001random}. The components of the ensemble are decision trees, each of which learns a set of decision rules for relating the observations to the model parameters. Each decision tree is supplied with a random subset of the evolutionary models and a random subset of the observable quantities, a process known as statistical bagging \citep[see section 8.7 of][]{hastie2005elements}, which prevents the random forest from becoming overfit to the training data. We moreover use a variant on random forests known as \emph{extremely} randomized trees \citep{geurts2006extremely}, which further randomizes the attribute and cut-point choices used when creating decision rules. 

The decision trees use information theory to decide which rule is the best choice for inferring quantities like age and mass from the supplied information \citep[see chapter 9 of][]{hastie2005elements}. At every stage, the rule that creates the largest decrease in mean squared error (MSE) is crafted. A rule may be, for example, ``all models with L $<0.4$ L$_\odot$ have M $<$ M$_\odot$.'' The rules are refined until every data point that was supplied to that particular tree is fully explained by a sequence of decisions. This process presents an opportunity for both rapidly inferring stellar parameters from observations and also understanding the relationships that exist in the data as well, as each decision tree explicitly ranks the relative importance and therefore inferential power of each observable quantity. Figure \ref{fig:importances} shows the relative importances of each type of observation in inferring stellar attributes. There it can be seen that the attributes used most often to construct decision rules are metallicity and temperature, which are each significantly more important attributes than the rest in resolving stellar parameters. Figure \ref{fig:importance-covariances} furthermore shows the covariances of these importances, which serves to demonstrate which combinations of variables are most useful for constraining stellar parameters. It reveals, for example, that constructing a rule based on [Fe/H] makes knowledge of other surface attributes such as the radius or surface gravity less important for constraining stellar attributes. 
%For example, the redness of the top left cell between the asteroseismic frequency ratio $\langle r_{1,3} \rangle$ (abscissa) and the surface gravity log g (ordinate) indicates that while this ratio is the most useful quantity for making inferences, a tree will often supplement it with information about the surface -- which by construction the ratios lack -- to complete an analysis. Likewise, the large covariance between T$_{\text{eff}}$ and [Fe/H] is due to the fact that the effective temperature of a star is not only a function of its mass but also its composition.%Taken together, these two plots illustrate what combination of variables provide the best connections for determining stellar parameters. 

\begin{figure}
    \centering
    \includegraphics[width=0.5\textwidth, keepaspectratio]{figs/importances/importances-perturb.pdf}
    \caption{Box-and-whisker plot of relative importances for each observable feature in inferring fundamental stellar parameters as measured by a random forest regressor grown from a grid of evolutionary models. The boxes display the first (16\%) and third (84\%) quartile of feature importances over all trees, the middle line indicates the median, and the whiskers extend to the most extreme values. The boxes are sorted by the median value.}
    \label{fig:importances}
\end{figure}

\begin{figure*}
    \centering
    \includegraphics[width=\textwidth, keepaspectratio, trim={0cm 0 0cm 0},clip]{figs/importances/cov-perturb.pdf}
    \caption{Standardized covariances of random forest feature importances, with the variables shown in order of median importance. Blue points on the off-diagonal indicate that when the abscissa variable is selected, the ordinate variable is redundant; and conversely, red points indicate that when the abscissa variable is selected, the ordinate variable becomes more useful for successful forecasting.}
    \label{fig:importance-covariances}
\end{figure*}

For many stars, information such as radii, luminosities, surface gravities, and/or oscillation modes with spherical degree $\ell=3$ are not available from observations. For example, the KOI data set lacks all of this information, and the hare-and-hound exercise data lack all of these except luminosities. We therefore must train random forests that predict those quantities instead of using them as inputs. We show the relative importances determined by these forests in Figure \ref{fig:importances2}. Remarkably, when $\ell=3$ modes and luminosities are omitted, effective temperatures jump in importance and tie with [Fe/H] as the most important quantity to have observed. 

\begin{figure*}
    \centering
    \includegraphics[width=0.5\textwidth,keepaspectratio]{figs/importances/importances-hares.pdf}\hfill
    \includegraphics[width=0.5\textwidth, keepaspectratio]{figs/importances/importances-kages.pdf}
    \caption{Box-and-whisker plots of relative importances of each observable feature in measuring fundamental stellar parameters for the hare-and-hound exercise data (left), where luminosities are available; and the Kepler objects-of-interest (right), where they are not. Octupole ($\ell=3$) modes have not been measured in any of these stars, so $\langle\delta\nu_{1,3}\rangle$ and $\langle r_{1,3}\rangle$ from evolutionary modelling are not supplied to these random forests. }
    \label{fig:importances2}
\end{figure*}


\subsubsection{Uncertainty}
\label{sec:uncertainties}
There are three separate sources of uncertainty in predicting stellar parameters. The first is the systematic uncertainty in the physics used to model stars. These uncertainties are unknown, however, and hence cannot be propagated. The second is the uncertainty belonging to the observations of the star. In order to propagate measurement uncertainties $\sigma$ into the predictions, we perturb all measured quantities $n=10,000$ times with normal noise having zero mean and standard deviation $\sigma$. We account for the covariance between asteroseismic separations and ratios by recalculating them upon each perturbation. 

The final source is model uncertainty. Fundamentally, each parameter can only be constrained to the extent that observations are able to bear information pertaining to that parameter. Even if observations were error-free, there still may exist a limit to which information gleaned from the surface may tell us about the physical attributes and evolutionary history of a star. We quantify those limits via cross-validation: we train the random forest on only a subset of the simulated evolutionary tracks and make predictions on a held-out validation set. We randomly select a different subset of the tracks 25 times to serve as different held-out validation sets and obtain averaged accuracy scores. 

We calculate accuracies using two scores. The first is the explained variance score V$_{\text{e}}$:
\begin{equation}
  \text{V}_{\text{e}} = 1 - \frac{\text{Var}\{ y - \hat y \}}{\text{Var}\{ y \}}
\end{equation}
where $y$ is the value we want to predict from the validation set (e.g.\ stellar mass), $\hat y$ is the predicted value from the random forest, and Var is the variance, i.e.\ the square of the standard deviation. This score tells us the extent to which the regressor has reduced the variance in the parameter it is predicting. The value ranges from negative infinity, which would be obtained by a pathologically bad predictor; to one for a perfect predictor, which occurs if all of the values are predicted with zero error. 

The second score we consider is the distance in units of model uncertainty $\hat \sigma$ between the true value $y$ and the predicted value $\hat y$. The model uncertainty can be obtained by taking the standard deviation of predictions across all of the decision trees in the forest. We show in Figure \ref{fig:evaluation-tracks} these accuracies as a function of the number of evolutionary tracks used in the training of the random forest. By both scores, it can be seen that when supplied with enough simulations, the random forest reduces the variance in each parameter and is able to make precise inferences. We also consider the number of trees in the forest and the number of models per evolutionary track; see Appendix \ref{sec:evaluation} for an extended discussion. 

\begin{figure*}
    \centering
    \includegraphics[width=0.5\textwidth,keepaspectratio]{figs/evaluation/legend.png}\\
    \includegraphics[width=0.5\textwidth,keepaspectratio]{figs/evaluation/num_tracks-ev.pdf}\hfill
    \includegraphics[width=0.5\textwidth,keepaspectratio]{figs/evaluation/num_tracks-sigma.pdf}
    \caption{Explained variance (left) and accuracy per precision score (right) of each stellar parameter as a function of the number of evolutionary tracks used in training the random forest. } 
    \label{fig:evaluation-tracks}
\end{figure*}

The explained variance score reveals that most attributes have very little model uncertainty when being inferred by a well-trained random forest. For example, essentially all of the uncertainty in predicting stellar radii and luminosities will stem from observational error. However, for some model properties -- most notably the mixing length parameter -- there is still a great deal of model uncertainty. Indeed, prior to about 500 evolutionary tracks, the difference between the absolute and predicted mixing lengths actually have a greater variance than the raw mixing lengths themselves. Likewise, the diffusion factor is difficult to constrain because a star can achieve the same present-day [Fe/H] by having a large initial non-hydrogen abundance and a large diffusion factor as it can by having the same initial [Fe/H] but with no diffusion. These difficult-to-constrain attributes will therefore be predicted with substantial error bars regardless of the precision of the observations. On the other hand, the accuracy per precision score shows that the regressor is well-calibrated: in the average case, the values are predicted within one $\hat\sigma$ as desired. In other words, the uncertainties for all of the parameters -- even for the ones that cannot be constrained very well -- are still appropriately gauged by the regressor. 


%%%%%%%%%%%%%%%%%%%%%%%%%%%%%%%%%%%%%%%%%
%%% Results %%%%%%%%%%%%%%%%%%%%%%%%%%%%%
%%%%%%%%%%%%%%%%%%%%%%%%%%%%%%%%%%%%%%%%%
\section{Results}
We perform three tests of our method. We begin with a hare-and-hound simulation exercise to show that we can can reliably recover parameters with known truth values. We then move to the Sun and the solar-like stars of 16 Cygni, which have been the subjects of many investigations; and we conclude by applying our method to thirty-three Kepler objects-of-interest. In each case, we train our random forest regressor on the subset of observational data that is available to the stars being processed. In the case of the Sun and 16 Cygni, we know very accurately their radii, luminosities, and surface gravities. For other stars, however, we predict this information rather than supplying it as an input. 


%%%%%%%%%%%%%%%%%%%%%%%%%%%%%%%%%%%%%%%%%
%%% Hare and Hound %%%%%%%%%%%%%%%%%%%%%%
%%%%%%%%%%%%%%%%%%%%%%%%%%%%%%%%%%%%%%%%%
\subsection{Hare and Hound} 
We performed a blind hare-and-hound exercise to evaluate the performance of our predictor. Author S.B.\ prepared twelve models varied in mass, initial chemical composition, and mixing length parameter with only some models having overshooting and only some models having atomic diffusion included. The models were evolved using the Yale rotating stellar evolution code \citep[YREC,][]{2008Ap&SS.316...31D}, which is a different evolution code than the one that was used to train the random forest. Effective temperatures, luminosities, [Fe/H] and $\nu_{\max}$ values as well as $\ell=0,1,2$ frequencies were obtained from each model. Author G.A.\  perturbed the ``observations'' of these models according to the scheme devised by \citet{spaceinn}. The true values and the perturbed observations can be seen in Appendix \ref{sec:hare-and-hound}. The perturbed observations and their uncertainties were given to author E.P.B.\ , who used the machine learning methods described herein to recover the stellar parameters of the models without being given access to the true values. The predicted ages, masses, and radii for these models are plotted against their true values in Figure \ref{fig:hare-comparison}. Here it can be seen that the method is able to recover the true model values within uncertainties even when they have been perturbed by noise. 

\begin{figure}
    \centering
    \includegraphics[width=0.5\textwidth,keepaspectratio]{figs/comparison/basu-Age-rel.pdf}\\
    \includegraphics[width=0.5\textwidth,keepaspectratio]{figs/comparison/basu-Mass-rel.pdf}\\
    \includegraphics[width=0.5\textwidth,keepaspectratio]{figs/comparison/basu-Radius-rel.pdf}
    \caption{Relative differences between the predicted and true values for age, mass, and radius as a function of the true values in a hare-and-hound simulation exercise.%Predicted values of age, mass, and radius plotted against the true values in a hare-and-hound simulation exercise. 
    \label{fig:hare-comparison}}
\end{figure}


%%%%%%%%%%%%%%%%%%%%%%%%%%%%%%%%%%%%%%%%%
%%% The Sun & 16 Cygni %%%%%%%%%%%%%%%%%%
%%%%%%%%%%%%%%%%%%%%%%%%%%%%%%%%%%%%%%%%%
\subsection{The Sun and 16 Cygni}
In order to be confident about our predictions for Kepler data, we first degrade the frequencies of the quiet Sun, i.e.\ the Sun with ``an intermediate level of solar activity,'' that were obtained by the Birmingham Solar-Oscillations Network \citep[BiSON,][]{2014MNRAS.439.2025D} to the level of information that is achievable by the spacecraft. We also degrade the Sun's uncertainties of classical observations by applying 16 Cyg B's uncertainties of effective temperature, luminosity, surface gravity, metallicity, $\nu_{\max}$, radius, and radial velocity. Finally, we perturb each value with random Gaussian noise according to its uncertainty to reflect the fact that the measured value of an uncertain observation is not \emph{per se} the true value. We show in Figure \ref{fig:corner} the densities for the predicted mass, initial composition, mixing length parameter, overshoot coefficient, and diffusion factor needed for fitting an evolutionary model to degraded data of the Sun, and also the predicted age, core-hydrogen abundance, and surface-helium abundance of the Sun. Relative uncertainties $\epsilon=100\cdot\sigma/\mu$ are also indicated. Initial condition distributions like these are useful as strong priors when performing more detailed evolutionary analyses. 

\begin{figure*}
    \centering
    \includegraphics[width=\textwidth,keepaspectratio]{figs/comparison/Tagesstern.pdf}
    \caption{Predictions from machine learning of initial (top six) and current-age (bottom three) stellar parameters for degraded solar data. Labels are placed at the mean and 3$\sigma$ levels. Dotted lines indicate the median and quartiles. Relative uncertainties $\epsilon$ are shown beside each plot. Note that the overshoot parameter applies to all convective boundaries and is not modified over the course of evolution; i.e.\ a non-zero value does not imply that the star has a convective core at the present age. }
    \label{fig:corner}
\end{figure*}

Effective temperatures, surface gravities, and metallicities of 16 Cyg A and B were obtained from \citet{2009A&A...508L..17R}; radii and luminosities from \citet{2013MNRAS.433.1262W}; and frequencies from \citet{2015MNRAS.446.2959D}. We obtained the radial velocity measurements of 16 Cyg A and B from \citet{2002ApJS..141..503N} and corrected frequencies for Doppler shifting as per the prescription in \citet{2014MNRAS.445L..94D}. We tried with and without line-of-sight corrections and found that it did not make a difference to the predicted quantities or their uncertainties. The initial parameters for the Sun and 16 Cygni as predicted by machine learning can be seen in Table \ref{tab:results}, and the predicted current-age parameters can be seen in Table \ref{tab:results-ca}. These results support the hypothesis that 16 Cyg A and B were formed at the same time with the same initial composition. 

\begin{deluxetable}{cccccccc}
\tabletypesize{\scriptsize}
\tablecaption{Means and standard deviations for initial stellar parameters -- masses, chemical compositions, mixing lengths, diffusion factors, and overshoot coefficients -- of 16 Cyg A and B as inferred by machine learning from their respective temperatures, metallicities, luminosities, surface gravities, radii, and asteroseismic observables $\langle \Delta\nu_0 \rangle$, $\langle \delta\nu_{0,2} \rangle$, $\langle \delta\nu_{1,3} \rangle$, $\langle r_{0,2} \rangle$, $\langle r_{1,3} \rangle$, $\langle r_{0,1} \rangle$, and $\langle r_{1,0} \rangle$. The values obtained from degraded solar data predicted on these quantities are shown for reference.
\label{tab:results}}
\tablewidth{0pt}
\tablehead{\colhead{Name} & \colhead{M$/$M$_\odot$} & \colhead{Y$_0$} & \colhead{Z$_0$} & \colhead{$\alpha_{\mathrm{MLT}}$} & \colhead{$\alpha_{\mathrm{ov}}$} & \colhead{D}}\startdata
16CygA & 1.08 $\pm$ 0.016 & 0.262 $\pm$ 0.0077 & 0.0222 $\pm$ 0.0015 & 1.85 $\pm$ 0.073 & 0.0791 $\pm$ 0.03 & 0.844 $\pm$ 0.8 \\
16CygB & 1.03 $\pm$ 0.015 & 0.267 $\pm$ 0.0068 & 0.0212 $\pm$ 0.0014 & 1.82 $\pm$ 0.069 & 0.115 $\pm$ 0.032 & 1.88 $\pm$ 1.6 \\
Degraded Sun & 1 $\pm$ 0.012 & 0.269 $\pm$ 0.006 & 0.0197 $\pm$ 0.0015 & 1.87 $\pm$ 0.08 & 0.0653 $\pm$ 0.016 & 3.6 $\pm$ 3.1
\enddata
\end{deluxetable}

\begin{deluxetable}{cccc}
\tabletypesize{\scriptsize}
%\rotate
\tablecaption{Means and standard deviations for current-age stellar attributes -- age, surface-helium and core-hydrogen abundances -- of 16 Cyg A and B as inferred by machine learning from their respective temperatures, metallicities, luminosities, surface gravities, radii, and asteroseismic observables $\langle \Delta\nu_0 \rangle$, $\langle \delta\nu_{0,2} \rangle$, $\langle \delta\nu_{1,3} \rangle$, $\langle r_{0,2} \rangle$, $\langle r_{1,3} \rangle$, $\langle r_{0,1} \rangle$, and $\langle r_{1,0} \rangle$. The values obtained from degraded solar data predicted on these quantities are shown for reference.
\label{tab:results-ca}}
\tablewidth{0pt}
\tablehead{\colhead{Name} & \colhead{$\tau/$Gyr} & \colhead{X$_{\mathrm{c}}$/M$_*$} & \colhead{Y$_{\mathrm{surf}}$}}\startdata
16CygA & 6.96 $\pm$ 0.42 & 0.0628 $\pm$ 0.026 & 0.247 $\pm$ 0.0083 \\
16CygB & 6.82 $\pm$ 0.29 & 0.155 $\pm$ 0.024 & 0.238 $\pm$ 0.017 \\
Degraded Sun & 4.62 $\pm$ 0.18 & 0.344 $\pm$ 0.026 & 0.239 $\pm$ 0.016
\enddata
\end{deluxetable}

We additionally predict the radii and luminosities of 16 Cyg A and B instead of using them as constraints. Figure \ref{fig:interferometry} shows our inferred radii, luminosities and surface-helium abundances of 16 Cyg A and B plotted against the values determined by interferometry \citep{2013MNRAS.433.1262W} and an asteroseismic estimate \citep{2014ApJ...790..138V}. Here again we find excellent agreement between our method and the measured values. 

\begin{figure}
    \centering
    %\includegraphics[width=\textwidth,keepaspectratio]{figs/16cyg-all.png}
    \includegraphics[width=0.5\textwidth, keepaspectratio]{figs/hists/cyg-radius.pdf}\\
    \includegraphics[width=0.5\textwidth, keepaspectratio]{figs/hists/cyg-L.pdf}\\
    \includegraphics[width=0.5\textwidth, keepaspectratio]{figs/hists/cyg-Y_surf.pdf}
    %\includegraphics[width=0.5\textwidth, keepaspectratio]{figs/hists/cyg-X_c.pdf}\\
    \caption{Probability densities showing predictions of 16 Cyg A (red) and B (blue) from machine learning of radii (top), luminosities (middle), and surface-helium abundances (bottom). Relative uncertainties $\epsilon$ are shown beside each plot. Predictions and $2\sigma$ uncertainties from interferometric (``int'') measurements and asteroseismic (``ast'') estimates are shown with arrows.}
    \label{fig:interferometry}
\end{figure}

Detailed modelling of 16 Cyg A and B have been performed  by \citet{2015ApJ...811L..37M} using the Asteroseismic Modeling Portal (AMP), a genetic algorithm for matching individual frequencies of stars to stellar models. They calculated their results using a fixed diffusion factor, and without heavy element diffusion (i.e.\ only helium diffusion) and no overshooting. In order to account for systematic uncertainties, they multiplied the spectroscopic uncertainties of 16 Cyg A and B by an arbitrary constant $c=3$. Therefore, in order to make a fair comparison between the results of our method and theirs, we generate a new matrix of evolutionary models with those same conditions and also increase the uncertainties on [Fe/H] by a factor of $c$. In Figure \ref{fig:16Cyg-hist}, we show probability densities of the predicted parameters of 16 Cyg A and B that we obtain using machine learning in comparison to the results obtained by AMP. We find good agreement of values and uncertainties. To perform their analysis, AMP required more than 15,000 hours of computational time to model 16 Cyg A and B using the world's 10th fastest supercomputer, the Texas Advanced Computing Center Stampede \citep{TOP500}. Here we have obtained comparable results in less than one minute using only global asteroseismic constraints and no individual frequencies. We note however that detailed optimization codes like AMP remain important for obtaining structural models of stars as well as for performing post-main sequence modelling, where our method has not yet been extended. 

\begin{figure*}
    \centering
    \includegraphics[width=0.5\textwidth, keepaspectratio]{figs/hists/cyg-age.pdf}\hfill
    \includegraphics[width=0.5\textwidth, keepaspectratio]{figs/hists/cyg-M.pdf}\\
    \includegraphics[width=0.5\textwidth, keepaspectratio]{figs/hists/cyg-Y.pdf}\hfill
    \includegraphics[width=0.5\textwidth, keepaspectratio]{figs/hists/cyg-Z.pdf}
    \caption{Probability densities showing predictions from machine learning of fundamental stellar parameters for 16 Cyg A (red) and B (blue) against predictions from AMP modelling. Relative uncertainties are shown beside each plot. Predictions and $2\sigma$ uncertainties from AMP modelling are shown with arrows.}
    \label{fig:16Cyg-hist}
\end{figure*}


%%%%%%%%%%%%%%%%%%%%%%%%%%%%%%%%%%%%%%%%%
%%% Kepler Objects of Interest %%%%%%%%%%
%%%%%%%%%%%%%%%%%%%%%%%%%%%%%%%%%%%%%%%%%
\subsection{Kepler Objects of Interest}
We obtain classical observations and frequencies of the KOI targets from \citet[hereafter KAGES]{2015MNRAS.452.2127S}. We use line-of-sight radial velocity corrections when available, which was only the case with KIC 6278762 \citep{2002AJ....124.1144L}, KIC 10666592 \citep{2013A&A...554A..84M}, and KIC 3632418 \citep{2006AstL...32..759G}. We use the random forest whose feature importances were shown in Figure \ref{fig:importances2} to predict the fundamental properties of these stars, the results of which can be seen in Table \ref{tab:results-kages}. Figure \ref{fig:us-vs-them} shows the fundamental parameters obtained from our method plotted against those obtained by KAGES. We find good agreement across all stars. Although still in statistical agreement, the median values of our predicted ages are systematically younger than those predicted by KAGES. This is because we consider overshooting calculations in our models, which, with all else being equal, results in stars having larger core masses and therefore being more evolved at the same age. 

\begin{deluxetable}{ccccccc}
\tabletypesize{\scriptsize}
\tablecaption{Means and standard deviations for initial conditions -- mass, chemical composition, mixing length, overshoot coefficient, and diffusion factor -- of the KOI data set inferred via machine learning from their respective temperatures, metallicities, and asteroseismic observables $\langle \Delta\nu_0 \rangle$, $\langle \delta\nu_{0,2} \rangle$, $\langle r_{0,2} \rangle$, $\langle r_{0,1} \rangle$, and $\langle r_{1,0} \rangle$. \label{tab:results-kages}}
\tablewidth{0pt}
\tablehead{\colhead{KIC} & \colhead{M$/$M$_\odot$} & \colhead{Y$_0$} & \colhead{Z$_0$} & \colhead{$\alpha_{\mathrm{MLT}}$} & \colhead{$\alpha_{\mathrm{ov}}$} & \colhead{D}}\startdata
3425851 & 1.15 $\pm$ 0.053 & 0.284 $\pm$ 0.02 & 0.0153 $\pm$ 0.0028 & 1.93 $\pm$ 0.23 & 0.0575 $\pm$ 0.055 & 0.503 $\pm$ 1 \\
3544595 & 0.911 $\pm$ 0.031 & 0.27 $\pm$ 0.0092 & 0.015 $\pm$ 0.0028 & 1.87 $\pm$ 0.1 & 0.157 $\pm$ 0.11 & 4.95 $\pm$ 4.5 \\
3632418 & 1.39 $\pm$ 0.057 & 0.268 $\pm$ 0.0093 & 0.019 $\pm$ 0.0032 & 1.98 $\pm$ 0.13 & 0.245 $\pm$ 0.14 & 1.1 $\pm$ 1.1 \\
4141376 & 1.02 $\pm$ 0.036 & 0.268 $\pm$ 0.01 & 0.012 $\pm$ 0.0025 & 1.87 $\pm$ 0.12 & 0.128 $\pm$ 0.12 & 4.03 $\pm$ 4.2 \\
4143755 & 0.987 $\pm$ 0.04 & 0.279 $\pm$ 0.0047 & 0.0137 $\pm$ 0.0026 & 1.75 $\pm$ 0.039 & 0.357 $\pm$ 0.076 & 14.6 $\pm$ 5.9 \\
4349452 & 1.22 $\pm$ 0.055 & 0.277 $\pm$ 0.012 & 0.0204 $\pm$ 0.0043 & 1.89 $\pm$ 0.17 & 0.103 $\pm$ 0.091 & 7.26 $\pm$ 8.7 \\
4914423 & 1.19 $\pm$ 0.051 & 0.273 $\pm$ 0.0099 & 0.0261 $\pm$ 0.0047 & 1.77 $\pm$ 0.11 & 0.0825 $\pm$ 0.044 & 2.63 $\pm$ 1.8 \\
5094751 & 1.11 $\pm$ 0.037 & 0.275 $\pm$ 0.0085 & 0.0181 $\pm$ 0.0029 & 1.78 $\pm$ 0.11 & 0.0711 $\pm$ 0.039 & 2.61 $\pm$ 1.5 \\
5866724 & 1.29 $\pm$ 0.065 & 0.276 $\pm$ 0.011 & 0.0269 $\pm$ 0.0057 & 1.8 $\pm$ 0.14 & 0.11 $\pm$ 0.091 & 6.91 $\pm$ 8.1 \\
6196457 & 1.33 $\pm$ 0.058 & 0.276 $\pm$ 0.0054 & 0.0328 $\pm$ 0.0053 & 1.72 $\pm$ 0.053 & 0.158 $\pm$ 0.062 & 6.39 $\pm$ 2.5 \\
6278762 & 0.758 $\pm$ 0.012 & 0.255 $\pm$ 0.0057 & 0.013 $\pm$ 0.0016 & 2.09 $\pm$ 0.066 & 0.0653 $\pm$ 0.029 & 5.24 $\pm$ 2.1 \\
6521045 & 1.2 $\pm$ 0.052 & 0.272 $\pm$ 0.0075 & 0.0272 $\pm$ 0.005 & 1.81 $\pm$ 0.072 & 0.112 $\pm$ 0.036 & 3.01 $\pm$ 1.4 \\
7670943 & 1.3 $\pm$ 0.06 & 0.279 $\pm$ 0.017 & 0.0215 $\pm$ 0.0045 & 1.98 $\pm$ 0.23 & 0.06 $\pm$ 0.062 & 0.925 $\pm$ 2.5 \\
8077137 & 1.23 $\pm$ 0.07 & 0.271 $\pm$ 0.0096 & 0.0183 $\pm$ 0.0028 & 1.85 $\pm$ 0.14 & 0.207 $\pm$ 0.11 & 3.16 $\pm$ 2.2 \\
8292840 & 1.15 $\pm$ 0.078 & 0.277 $\pm$ 0.0099 & 0.0161 $\pm$ 0.0048 & 1.84 $\pm$ 0.15 & 0.147 $\pm$ 0.12 & 11.3 $\pm$ 11 \\
8349582 & 1.23 $\pm$ 0.041 & 0.269 $\pm$ 0.0077 & 0.0438 $\pm$ 0.0074 & 1.96 $\pm$ 0.12 & 0.113 $\pm$ 0.067 & 2.31 $\pm$ 1 \\
8478994 & 0.814 $\pm$ 0.022 & 0.274 $\pm$ 0.0087 & 0.0109 $\pm$ 0.0013 & 1.93 $\pm$ 0.065 & 0.216 $\pm$ 0.075 & 18.2 $\pm$ 9.8 \\
8494142 & 1.42 $\pm$ 0.056 & 0.273 $\pm$ 0.011 & 0.0278 $\pm$ 0.0045 & 1.69 $\pm$ 0.066 & 0.0931 $\pm$ 0.053 & 1.61 $\pm$ 1.8 \\
8554498 & 1.38 $\pm$ 0.063 & 0.271 $\pm$ 0.0073 & 0.0306 $\pm$ 0.0029 & 1.7 $\pm$ 0.073 & 0.135 $\pm$ 0.083 & 2.08 $\pm$ 1.6 \\
8684730 & 1.43 $\pm$ 0.032 & 0.276 $\pm$ 0.0075 & 0.0409 $\pm$ 0.0053 & 1.9 $\pm$ 0.14 & 0.274 $\pm$ 0.096 & 14.8 $\pm$ 8.8 \\
8866102 & 1.26 $\pm$ 0.067 & 0.278 $\pm$ 0.013 & 0.0209 $\pm$ 0.0048 & 1.79 $\pm$ 0.15 & 0.0805 $\pm$ 0.069 & 5.16 $\pm$ 7.4 \\
9414417 & 1.36 $\pm$ 0.054 & 0.265 $\pm$ 0.008 & 0.0177 $\pm$ 0.0028 & 1.87 $\pm$ 0.13 & 0.181 $\pm$ 0.1 & 2.24 $\pm$ 1.8 \\
9592705 & 1.44 $\pm$ 0.045 & 0.274 $\pm$ 0.01 & 0.0281 $\pm$ 0.0039 & 1.71 $\pm$ 0.066 & 0.115 $\pm$ 0.057 & 0.736 $\pm$ 0.61 \\
9955598 & 0.929 $\pm$ 0.029 & 0.27 $\pm$ 0.011 & 0.0231 $\pm$ 0.0039 & 1.95 $\pm$ 0.11 & 0.191 $\pm$ 0.13 & 2.36 $\pm$ 1.9 \\
10514430 & 1.13 $\pm$ 0.057 & 0.278 $\pm$ 0.0057 & 0.0204 $\pm$ 0.0041 & 1.78 $\pm$ 0.06 & 0.312 $\pm$ 0.096 & 5.01 $\pm$ 2.1 \\
10586004 & 1.32 $\pm$ 0.074 & 0.274 $\pm$ 0.0059 & 0.0387 $\pm$ 0.0069 & 1.81 $\pm$ 0.13 & 0.214 $\pm$ 0.13 & 4.33 $\pm$ 3.8 \\
10666592 & 1.5 $\pm$ 0.02 & 0.298 $\pm$ 0.013 & 0.0293 $\pm$ 0.0033 & 1.79 $\pm$ 0.11 & 0.0623 $\pm$ 0.046 & 0.2 $\pm$ 0.18 \\
10963065 & 1.09 $\pm$ 0.032 & 0.264 $\pm$ 0.0083 & 0.014 $\pm$ 0.0026 & 1.79 $\pm$ 0.12 & 0.0508 $\pm$ 0.028 & 3.2 $\pm$ 2.7 \\
11133306 & 1.11 $\pm$ 0.046 & 0.272 $\pm$ 0.01 & 0.0207 $\pm$ 0.0041 & 1.85 $\pm$ 0.16 & 0.0422 $\pm$ 0.035 & 5.09 $\pm$ 5.8 \\
11295426 & 1.11 $\pm$ 0.033 & 0.267 $\pm$ 0.01 & 0.0248 $\pm$ 0.0035 & 1.8 $\pm$ 0.088 & 0.0483 $\pm$ 0.032 & 1.28 $\pm$ 0.9 \\
11401755 & 1.15 $\pm$ 0.035 & 0.272 $\pm$ 0.0066 & 0.0156 $\pm$ 0.0022 & 1.86 $\pm$ 0.067 & 0.341 $\pm$ 0.071 & 3.89 $\pm$ 1.8 \\
11807274 & 1.33 $\pm$ 0.075 & 0.274 $\pm$ 0.0087 & 0.0248 $\pm$ 0.0049 & 1.76 $\pm$ 0.083 & 0.104 $\pm$ 0.063 & 5.47 $\pm$ 5.4 \\
11853905 & 1.22 $\pm$ 0.054 & 0.272 $\pm$ 0.0073 & 0.0295 $\pm$ 0.0052 & 1.85 $\pm$ 0.12 & 0.172 $\pm$ 0.08 & 3.49 $\pm$ 1.9 \\
11904151 & 0.931 $\pm$ 0.033 & 0.265 $\pm$ 0.0092 & 0.0161 $\pm$ 0.0031 & 1.84 $\pm$ 0.13 & 0.051 $\pm$ 0.029 & 2.86 $\pm$ 2 
\enddata
\end{deluxetable}

\begin{deluxetable}{ccccccc}
\tabletypesize{\scriptsize}
\tablecaption{Means and standard deviations for current-age conditions -- age, core-hydrogen abundance, surface gravity, luminosity, radius, and surface-helium abundance -- of the KOI data set inferred via machine learning from their respective temperatures, metallicities, and asteroseismic observables $\langle \Delta\nu_0 \rangle$, $\langle \delta\nu_{0,2} \rangle$, $\langle r_{0,2} \rangle$, $\langle r_{0,1} \rangle$, and $\langle r_{1,0} \rangle$.  \label{tab:results-kages-curr}}
\tablewidth{0pt}
\tablehead{\colhead{KIC} & \colhead{$\tau/$Gyr} & \colhead{X$_{\mathrm{c}}$/M$_*$} & \colhead{log g} & \colhead{L$/$L$_\odot$} & \colhead{R$/$R$_\odot$} & \colhead{Y$_{\mathrm{surf}}$}}\startdata
3425851 & 3.65 $\pm$ 0.76 & 0.138 $\pm$ 0.083 & 4.23 $\pm$ 0.01 & 2.73 $\pm$ 0.16 & 1.36 $\pm$ 0.022 & 0.272 $\pm$ 0.026 \\
3544595 & 6.65 $\pm$ 1.5 & 0.315 $\pm$ 0.078 & 4.46 $\pm$ 0.017 & 0.838 $\pm$ 0.067 & 0.937 $\pm$ 0.021 & 0.229 $\pm$ 0.022 \\
3632418 & 3.02 $\pm$ 0.36 & 0.0958 $\pm$ 0.039 & 4.02 $\pm$ 0.0084 & 5.18 $\pm$ 0.27 & 1.91 $\pm$ 0.032 & 0.244 $\pm$ 0.022 \\
4141376 & 3.39 $\pm$ 0.68 & 0.382 $\pm$ 0.07 & 4.41 $\pm$ 0.01 & 1.41 $\pm$ 0.095 & 1.05 $\pm$ 0.017 & 0.238 $\pm$ 0.022 \\
4143755 & 7.79 $\pm$ 0.89 & 0.0798 $\pm$ 0.022 & 4.1 $\pm$ 0.016 & 2.31 $\pm$ 0.13 & 1.49 $\pm$ 0.036 & 0.177 $\pm$ 0.021 \\
4349452 & 2.37 $\pm$ 0.78 & 0.365 $\pm$ 0.1 & 4.28 $\pm$ 0.012 & 2.48 $\pm$ 0.14 & 1.32 $\pm$ 0.022 & 0.221 $\pm$ 0.043 \\
4914423 & 5.19 $\pm$ 0.58 & 0.063 $\pm$ 0.037 & 4.16 $\pm$ 0.01 & 2.47 $\pm$ 0.16 & 1.5 $\pm$ 0.026 & 0.235 $\pm$ 0.023 \\
5094751 & 5.31 $\pm$ 0.66 & 0.0751 $\pm$ 0.041 & 4.21 $\pm$ 0.0088 & 2.22 $\pm$ 0.12 & 1.37 $\pm$ 0.017 & 0.229 $\pm$ 0.024 \\
5866724 & 2.4 $\pm$ 0.97 & 0.364 $\pm$ 0.12 & 4.24 $\pm$ 0.018 & 2.72 $\pm$ 0.15 & 1.42 $\pm$ 0.024 & 0.229 $\pm$ 0.038 \\
6196457 & 3.61 $\pm$ 0.75 & 0.195 $\pm$ 0.06 & 4.11 $\pm$ 0.021 & 3.46 $\pm$ 0.21 & 1.7 $\pm$ 0.036 & 0.24 $\pm$ 0.017 \\
6278762 & 10.3 $\pm$ 0.97 & 0.348 $\pm$ 0.026 & 4.56 $\pm$ 0.008 & 0.342 $\pm$ 0.023 & 0.758 $\pm$ 0.0062 & 0.192 $\pm$ 0.022 \\
6521045 & 5.59 $\pm$ 0.39 & 0.0309 $\pm$ 0.011 & 4.13 $\pm$ 0.0063 & 2.68 $\pm$ 0.17 & 1.57 $\pm$ 0.031 & 0.222 $\pm$ 0.02 \\
7670943 & 2.24 $\pm$ 0.6 & 0.325 $\pm$ 0.088 & 4.23 $\pm$ 0.01 & 3.28 $\pm$ 0.23 & 1.44 $\pm$ 0.025 & 0.262 $\pm$ 0.03 \\
8077137 & 4.42 $\pm$ 0.95 & 0.0868 $\pm$ 0.052 & 4.08 $\pm$ 0.017 & 3.68 $\pm$ 0.26 & 1.68 $\pm$ 0.044 & 0.218 $\pm$ 0.032 \\
8292840 & 3.47 $\pm$ 1.5 & 0.284 $\pm$ 0.14 & 4.25 $\pm$ 0.023 & 2.54 $\pm$ 0.2 & 1.34 $\pm$ 0.026 & 0.192 $\pm$ 0.049 \\
8349582 & 6.73 $\pm$ 0.66 & 0.0188 $\pm$ 0.013 & 4.17 $\pm$ 0.012 & 2.22 $\pm$ 0.12 & 1.52 $\pm$ 0.018 & 0.231 $\pm$ 0.015 \\
8478994 & 4.5 $\pm$ 1.7 & 0.496 $\pm$ 0.054 & 4.55 $\pm$ 0.013 & 0.522 $\pm$ 0.037 & 0.796 $\pm$ 0.015 & 0.208 $\pm$ 0.022 \\
8494142 & 2.81 $\pm$ 0.53 & 0.172 $\pm$ 0.067 & 4.06 $\pm$ 0.017 & 4.5 $\pm$ 0.32 & 1.84 $\pm$ 0.039 & 0.24 $\pm$ 0.031 \\
8554498 & 3.66 $\pm$ 0.77 & 0.0995 $\pm$ 0.059 & 4.05 $\pm$ 0.017 & 4.05 $\pm$ 0.2 & 1.85 $\pm$ 0.04 & 0.247 $\pm$ 0.018 \\
8684730 & 3 $\pm$ 0.37 & 0.24 $\pm$ 0.064 & 4.06 $\pm$ 0.045 & 4.04 $\pm$ 0.53 & 1.87 $\pm$ 0.11 & 0.176 $\pm$ 0.04 \\
8866102 & 1.91 $\pm$ 0.72 & 0.405 $\pm$ 0.11 & 4.27 $\pm$ 0.016 & 2.78 $\pm$ 0.16 & 1.36 $\pm$ 0.024 & 0.239 $\pm$ 0.04 \\
9414417 & 3.11 $\pm$ 0.32 & 0.0869 $\pm$ 0.031 & 4.02 $\pm$ 0.0069 & 4.97 $\pm$ 0.35 & 1.89 $\pm$ 0.037 & 0.213 $\pm$ 0.028 \\
9592705 & 3.02 $\pm$ 0.43 & 0.057 $\pm$ 0.027 & 3.98 $\pm$ 0.01 & 5.57 $\pm$ 0.41 & 2.04 $\pm$ 0.042 & 0.26 $\pm$ 0.016 \\
9955598 & 7 $\pm$ 1 & 0.367 $\pm$ 0.035 & 4.49 $\pm$ 0.0063 & 0.66 $\pm$ 0.043 & 0.903 $\pm$ 0.013 & 0.25 $\pm$ 0.02 \\
10514430 & 6.41 $\pm$ 0.85 & 0.0645 $\pm$ 0.023 & 4.08 $\pm$ 0.015 & 2.94 $\pm$ 0.17 & 1.62 $\pm$ 0.033 & 0.222 $\pm$ 0.021 \\
10586004 & 4.75 $\pm$ 1.4 & 0.123 $\pm$ 0.09 & 4.1 $\pm$ 0.04 & 3.12 $\pm$ 0.27 & 1.71 $\pm$ 0.068 & 0.241 $\pm$ 0.021 \\
10666592 & 2.02 $\pm$ 0.24 & 0.146 $\pm$ 0.032 & 4.02 $\pm$ 0.0065 & 5.79 $\pm$ 0.33 & 1.98 $\pm$ 0.019 & 0.29 $\pm$ 0.014 \\
10963065 & 4.39 $\pm$ 0.61 & 0.165 $\pm$ 0.053 & 4.29 $\pm$ 0.0075 & 1.96 $\pm$ 0.1 & 1.24 $\pm$ 0.015 & 0.218 $\pm$ 0.03 \\
11133306 & 4.12 $\pm$ 0.85 & 0.22 $\pm$ 0.079 & 4.32 $\pm$ 0.0095 & 1.72 $\pm$ 0.11 & 1.21 $\pm$ 0.021 & 0.222 $\pm$ 0.036 \\
11295426 & 6.24 $\pm$ 0.77 & 0.085 $\pm$ 0.037 & 4.28 $\pm$ 0.0073 & 1.65 $\pm$ 0.096 & 1.26 $\pm$ 0.019 & 0.243 $\pm$ 0.012 \\
11401755 & 5.58 $\pm$ 0.5 & 0.0436 $\pm$ 0.0087 & 4.04 $\pm$ 0.0081 & 3.46 $\pm$ 0.18 & 1.7 $\pm$ 0.026 & 0.21 $\pm$ 0.028 \\
11807274 & 2.73 $\pm$ 1 & 0.297 $\pm$ 0.11 & 4.17 $\pm$ 0.025 & 3.55 $\pm$ 0.22 & 1.58 $\pm$ 0.038 & 0.215 $\pm$ 0.034 \\
11853905 & 5.67 $\pm$ 0.7 & 0.0399 $\pm$ 0.022 & 4.11 $\pm$ 0.011 & 2.75 $\pm$ 0.17 & 1.62 $\pm$ 0.033 & 0.224 $\pm$ 0.022 \\
11904151 & 9.68 $\pm$ 1.5 & 0.0787 $\pm$ 0.036 & 4.35 $\pm$ 0.0098 & 1.08 $\pm$ 0.062 & 1.07 $\pm$ 0.019 & 0.213 $\pm$ 0.025 
\enddata
\end{deluxetable}

\begin{figure*}
    \centering
    \includegraphics[width=0.332\textwidth,keepaspectratio]{figs/comparison/kages-logg.pdf}\hfill
    \includegraphics[width=0.332\textwidth,keepaspectratio]{figs/comparison/kages-Radius.pdf}\hfill
    \includegraphics[width=0.332\textwidth,keepaspectratio]{figs/comparison/kages-Luminosity.pdf}\\
    \includegraphics[width=0.332\textwidth,keepaspectratio]{figs/comparison/kages-Mass.pdf}
    \includegraphics[width=0.332\textwidth,keepaspectratio]{figs/comparison/kages-Age.pdf}
    \caption{Predicted surface gravities, radii, luminosities, masses, and ages of 33 Kepler objects-of-interest plotted against the suggested KAGES values. Medians, 16\% quantiles, and 84\% quantiles are shown for each point. A dashed line of agreement is shown in both plots to guide the eye. }
    \label{fig:us-vs-them}
\end{figure*}

We find a significant linear trend in the Kepler objects-of-interest between the diffusion factor and stellar mass needed to reproduce observations ($P = 0.00039$ from a two-sided t-test with $N-2=31$ degrees of freedom). Since the values of mass and diffusion factor are uncertain, an ordinary linear fit between them would be affected by regression dilution and hence result in biased estimates, a problem known as attenuation bias. We therefore use Deming regression to estimate the coefficients of this relation, which avoids this problem \citep{deming1943statistical}. We show the fitted diffusion factors as a function of predicted stellar mass for all of these stars in Figure \ref{fig:diffusion}. We find that a diffusion factor that linearly decreases with mass, i.e.\ 
\begin{equation}
    \text{D} = ( 8.88 \pm 2.12 ) - ( 5.79 \pm 1.56 ) \cdot \text{M}/\text{M}_\odot
\end{equation}
explains observations better than a diffusion factor that is constant for all stars (e.g.\ D=1 or D=0). 

\begin{figure}
    \centering
    \includegraphics[width=0.5\textwidth,keepaspectratio]{figs/diffusion/diffusion.pdf}
    \caption{Diffusion factor as a function of stellar mass for 33 Kepler objects-of-interest, with D being shown in log space. A significant linear trend can be seen to exist between D and M. \label{fig:diffusion} } 
\end{figure}

%%%%%%%%%%%%%%%%%%%%%%%%%%%%%%%%%%%%%%%%%
%%% Discussion %%%%%%%%%%%%%%%%%%%%%%%%%%
%%%%%%%%%%%%%%%%%%%%%%%%%%%%%%%%%%%%%%%%%
\section{Discussion}
The amount of time it takes to predict the fundamental characteristics of a star using a trained random forest can be decomposed into two parts: the amount of time it takes to calculate Monte Carlo perturbations to the observations, and the amount of time it takes to make a prediction on each perturbed set of observations. Hence we have
\begin{equation}
    t = n(t_p + t_r)
\end{equation}
where $t$ is the total time, $n$ is the number of perturbations, $t_p$ is the time it takes to perform a single perturbation, and $t_r$ is the random forest regression time. We typically see times of $t_p = (7.9 \pm 0.7) \cdot 10^{-3}$ $(\si{\s})$ and $t_r = (1.8 \pm 0.4) \cdot 10^{-5}$ $(\si{\s})$. We chose a conservative $n=$10,000 for the results presented here, which results in a time of around a minute per star. Since each star can be processed independently in parallel, a computing cluster could feasibly process a catalog containing millions of objects in less than a day's time. Since $t_p \ll t_r$, the calculation depends almost entirely on the time it takes to perturb the observations.\footnote{Our perturbation code uses an interpreted language (R), so if needed, it would have room still for even more speed-up.} There is also a one-time cost of generating the matrix of training data, which took us a few days to run on our computing cluster. Unless one wants to consider a different range of parameters or different input physics, however, the matrix does not need to be calculated again; this matrix can be re-used for all future stars that are observed. Finally, there is also the one-time cost of training the random forest, which takes less than a minute and can be reused without retraining on every star. 

Previously, \citet{pulone1997age} developed a neural network for predicting stellar age based on the star's position in the Hertzsprung-Russell diagram. More recently, \citet{2016arXiv160200902V} have worked on incorporating seismic information into that analysis as we have done here. Our method provides several advantages over these. Firstly, the random forests that we use perform constrained regression, meaning that the values we predict for quantities like age and mass will always be non-negative and within the bounds of the training data, which is not true of the neural networks-based approach that they take. Secondly, using \emph{averaged} frequency separations allows us to make predictions without need for concern over which radial orders were observed. Thirdly, we have shown that our random forests are very fast to train, and can be retrained in only seconds for stars that are missing observational constraints such as luminosities. In contrast, deep neural networks are computationally intensive to train, taking days or even weeks to converge depending on the breadth of network topologies considered in the cross-validation. Finally, our grid is varied in six parameters -- M, Y$_0$, Z$_0$, $\alpha_{\text{MLT}}$, $\alpha_{\text{ov}}$, and D, which allows our method to explore a wide range of stellar physics.


%%%%%%%%%%%%%%%%%%%%%%%%%%%%%%%%%%%%%%
%%% Conclusions %%%%%%%%%%%%%%%%%%%%%%
%%%%%%%%%%%%%%%%%%%%%%%%%%%%%%%%%%%%%%
\section{Conclusions}
Here we have considered the constrained multiple-regression problem of inferring fundamental stellar parameters from observations. We created a grid of evolutionary tracks varied in mass, chemical composition, mixing length parameter, overshooting coefficient, and diffusion factor. We evolved each track in time along the main sequence and collected classical properties as well as global statistics on the modes of oscillations from models along each evolutionary path. We used this matrix of simulations to train a machine learning algorithm to be able to discern the patterns that relate observations to stellar properties. We then applied this method to hare-and-hound exercise data, the Sun, 16 Cyg A and B, and thirty-three planet-hosting candidates that have been observed by Kepler. We rapidly obtained precise initial stellar conditions and current-age values, which are vital for ensemble studies of the Galaxy; and the retrodicted initial conditions like the mixing length parameter and overshoot coefficient can be used as strong priors when performing more detailed stellar modelling, e.g.\ when obtaining a reference model for an inversion. Remarkably, we were able to empirically determine the value of the diffusion factor and hence the efficiency of diffusion required to reproduce the observations instead of inhibiting it \emph{ad hoc}. We look forward to having a larger sample size in order to better constrain the diffusion factor and determine what other variables are relevant in its parametrization. 

We note that these estimates represent a set of choices in stellar physics, for which such a bias is impossible to propagate. Nevertheless, varying quantities that are usually kept fixed, such as the mixing length parameter, diffusion factor, and overshooting coefficient, takes us a step in the right direction. And finally, the fact that quantities such as stellar radii and luminosities -- quantities that have been measured accurately, not just precisely -- can be reproduced so well by this method, gives a degree of confidence in its efficacy. 

The method we have presented here is only applicable to main-sequence stars. We intend to extend this study to later stages of evolution. 


%%%%%%%%%%%%%%%%%%%%%%%%%%%%%%%%%%%%%%
%%% Acknowledgements %%%%%%%%%%%%%%%%%
%%%%%%%%%%%%%%%%%%%%%%%%%%%%%%%%%%%%%%
\acknowledgments The research leading to the presented results has received funding from the European Research Council under the European Community's Seventh Framework Programme (FP7/2007-2013) / ERC grant agreement no 338251 (StellarAges). This research was undertaken in the context of the International Max Planck Research School. S.B. acknowledges partial support from NSF grant AST-1514676 and NASA grant NNX13AE70G.

Analysis in this manuscript was performed with python3 libraries scikit-learn \citep{scikit-learn}, NumPy \citep{van2011numpy}, and pandas \citep{mckinney2010data}; the R software package \citep{R}; and the R libraries magicaxis \citep{magicaxis}, RColorBrewer \citep{RColorBrewer}, parallelMap \citep{parallelMap}, data.table \citep{data.table}, lpSolve \citep{lpSolve}, ggplot2 \citep{ggplot2}, GGally \citep{GGally}, scales \citep{scales}, deming \citep{deming}, and matrixStats \citep{matrixStats}. 

\appendix


%%%%%%%%%%%%%%%%%%%%%%%%%%%%%%%
%%% Grid strategy %%%%%%%%%%%%%
%%%%%%%%%%%%%%%%%%%%%%%%%%%%%%%
\section{Initial grid strategy}
\label{sec:grid}
The initial conditions of a stellar model can be viewed as a 6-orthotope with dimensions M, Y$_0$, Z$_0$, $\alpha_{\text{MLT}}$, $\alpha_{\text{ov}}$, and D. In order to vary all of these parameters simultaneously, we construct a quasi-random grid of initial conditions. 

A linear grid subdivides all dimensions in which initial quantities can vary into equal parts and creates a track of models for every combination of these subdivisions. Although in the limit such a strategy will eventually fill the hypercube of initial conditions, it does so very slowly. It is furthermore suboptimal in the sense that linear grids maximize redundant information, as each varied quantity is tried with the exact same values of all other parameters that have been considered already. In a high-dimensional setting, if any of the parameters are irrelevant to the task of the computation, then the majority of the tracks in a linear grid will not contribute any new information.

A refinement on this approach is to create a grid of models with randomly varied initial conditions. Such a strategy fills the space more rapidly, and furthermore solves the problem of redundant information. However, this approach suffers from a different problem: since the points are generated at random, they tend to ``clump up'' at random as well. This results in random gaps in the parameter space, which are obviously undesirable. 

Therefore, in order to select points that do not stack, do not clump, and also fill the space as rapidly as possible, we generate Sobol numbers \citep{sobol1967distribution} in the unit 6-cube and map them to the parameter ranges of each quantity that we want to vary. Sobol numbers are a sequence of $m$-dimensional vectors $x_1 \ldots x_n$ in the unit hypercube $I^m$ constructed such that the integral of a real function $f$ in that space is equivalent in the limit to that function evaluated on those numbers, that is,
\begin{equation}
    \int_{I^m} f = \lim_{n \to \infty} \frac{1}{n}\sum_{i=1}^n f(x_i)
\end{equation}
with the sequence being chosen such that the convergence is achieved as quickly as possible. By doing this, we both minimize redundant information and furthermore sample the hyperspace of possible stars as uniformly as possible. Figure \ref{fig:grids} visualizes the different methods of generating multidimensional grids: linear, random, and the quasi-random strategy that we took. This method applied to initial model conditions was shown in Figure \ref{fig:inputs} with 1- and 2D projection plots of the evolutionary tracks generated for our grid. 

\begin{figure*}
    \centering
    \subfloat[Linear]{{\includegraphics[width=0.33\textwidth,keepaspectratio]{figs/grids/linear.png}}}%
    %\hfill
    \subfloat[Random]{{\includegraphics[width=0.33\textwidth,keepaspectratio]{figs/grids/random.png}}}%
    %\hfill
    \subfloat[Quasi-random]{{\includegraphics[width=0.33\textwidth,keepaspectratio]{figs/grids/quasirandom.png}}}%
    \caption{Results of different methods for generating multidimensional grids portrayed via a unit cube projected onto a unit square. Linear (left), random (middle), and quasi-random (right) grids are generated in three dimensions, with color depicting the third dimension, i.e., the distance between the reader and the screen. From top to bottom, all three methods are shown with 100, 400, and 2000 points generated, respectively. }%
    \label{fig:grids}
\end{figure*}


%%%%%%%%%%%%%%%%%%%%%%%%%%%%%%%
%%% Remeshing %%%%%%%%%%%%%%%%%
%%%%%%%%%%%%%%%%%%%%%%%%%%%%%%%
\section{Adaptive Remeshing}
\label{sec:remeshing}

When performing element diffusion calculations in MESA, the surface abundance of each isotope is considered as an average over the outermost cells of the model. The number of outer cells $N$ is chosen such that the mass of the surface is more than $dq$ times the mass of the $(N+1)^{\text{th}}$ cell. Occasionally, this approach can lead to a situation where surface abundances change dramatically and discontinuously in a single time-step. These abundance discontinuities then propagate as discontinuities in effective temperatures, surface gravities, and radii. An example of such a difficulty can be seen in Figure \ref{fig:discontinuity}. 

\begin{figure}
    \centering
    \subfloat[First iteration]{{\includegraphics[width=0.5\textwidth,keepaspectratio]{figs/diffusion/discontinuity-1.pdf}}}\\
    \subfloat[Second iteration]{{\includegraphics[width=0.5\textwidth,keepaspectratio]{figs/diffusion/discontinuity-2.pdf}}}\\
    \subfloat[Third iteration]{{\includegraphics[width=0.5\textwidth,keepaspectratio]{figs/diffusion/discontinuity-3.pdf}}}%
    \caption{Surface abundance discontinuity detection and iterative remeshing for an evolutionary track. The detected discontinuities are encircled in red. The third iteration has no discontinuities and so this track is considered to have converged. }
    \label{fig:discontinuity}
\end{figure}

Instead of being a physical reality, these effects arise only when there is insufficient mesh resolution in the outermost layers of the model. We therefore seek to detect these cases and re-run any such evolutionary track with finer mesh resolution. We consider a track an outlier if its surface hydrogen abundance changes by more than 0.01 in a single time-step. We iteratively re-run any track with outliers detected using a finer mesh resolution, and, if necessary, smaller time-steps, until convergence is reached. The process and a resolved track can also be seen in Figure \ref{fig:discontinuity}. 

Some tracks still do not converge without surface abundance discontinuities despite the fineness of the mesh or the brevity of the time-steps, and are therefore not included in our study. These troublesome evolutionary tracks seem to be located only located in a thin ridge of models having sufficient stellar mass (M $>$ 1), a deficit of initial metals (Z $<$ 0.001) and a specific inefficiency of diffusion (D $\simeq$ 0.01). A visualization of this can be seen in Figure \ref{fig:diffusion-gap}.

\begin{figure*}
    \centering
    \includegraphics[width=0.5\textwidth,keepaspectratio]{figs/diffusion/FeH0_M_D.pdf}\hfill
    \includegraphics[width=0.5\textwidth,keepaspectratio]{figs/diffusion/Fe_H_M_D.pdf}
    \caption{Stellar mass as a function of diffusion factor colored by initial surface metallicity (left) and final surface metallicity (right). A ridge of unconverged evolutionary tracks can be seen around a diffusion factor of 0.01. Beyond this ridge, tracks that were initially metal-poor end their main-sequence lives with all of their metals drained from their surfaces. } 
    \label{fig:diffusion-gap}
\end{figure*}


%%%%%%%%%%%%%%%%%%%%%%%%%%%%%%%
%%% Model selection %%%%%%%%%%%
%%%%%%%%%%%%%%%%%%%%%%%%%%%%%%%
\section{Model selection}
\label{sec:selection}
In order to prevent statistical bias towards the evolutionary tracks that generate the most models, i.e.\ the ones that require the most careful calculations and therefore use smaller time-steps, or those that live on the main sequence for a longer amount of time; we select $N=64$ models from each evolutionary track such that the models are as evenly-spaced in core-hydrogen abundance as possible. 

Starting from the original vector of length $m$ of core-hydrogen abundances $\vec X$, we find the subset of length $m$ that is closest to the optimal spacing $\vec B$, where
\begin{equation}
  \vec B \equiv \qty[
    X_T, 
    \ldots,
    \frac{(m-i)\cdot X_T + X_Z}{m-1}, 
    \ldots, 
    X_Z
  ]
\end{equation}
with $X_Z$ being the core-hydrogen abundance at zero-age (ZAMS) and $X_T$ being that at the end of the star's main-sequence lifetime (TAMS). To obtain the closest possible vector to $\vec B$ from our data $\vec X$, we solve a transportation problem using integer optimization \citep{23145595}. First we set up a cost matrix $\boldsymbol{C}$ consisting of absolute differences between the original abundances $\vec X$ and the ideal abundances $\vec B$:
\begin{equation}
  \boldsymbol C \equiv 
  \begin{bmatrix}
    \abs{B_1-X_1} & \abs{B_1-X_2} & \dots & \abs{B_1-X_n} \\ 
    \abs{B_2-X_1} & \abs{B_2-X_2} & \dots & \abs{B_2-X_n} \\ 
    \vdots & \vdots & \ddots & \vdots\\ 
    \abs{B_m-X_1} & \abs{B_m-X_2} & \dots & \abs{B_m-X_n}
  \end{bmatrix}.
\end{equation}
We then require that exactly $m$ values are selected from $\vec X$, and that each value is selected no more than one time. Simply selecting the closest data point to each ideally-separated point will not work because this could result in the same point being selected twice; and selecting the second closest point in that situation does not remedy it because a different result would be had if the points were chosen in a different order. 

We call the optimal solution matrix by $\hat{\boldsymbol{S}}$, and find it by minimising the cost matrix subject to the following constraints:
\begin{align}
  \hat{\boldsymbol{S}} = \underset{\boldsymbol S}{\arg\min} \; & \sum_{ij} S_{ij} C_{ij} \notag\\
  \text{subject to } & \sum_j S_{ij} \leq 1 \; \text{ for all } i=1\ldots N \notag\\
  \text{and } & \sum_i S_{ij} = 1 \; \text{ for all } j=1\ldots M.
  \label{eq:optimal-spacing}
\end{align}
The indices of $\vec X$ that are most near to being equidistantly-spaced are then found by looking at which columns of $\hat{\boldsymbol S}$ contain ones, and we are done. The solution is visualized in Figure \ref{fig:nearly-even}.

\begin{figure*}
    \centering
    \includegraphics[width=\textwidth, keepaspectratio]{figs/nearly-even.pdf}
    \caption{ A visaulization of the model selection process performed on each evolutionary track in order to obtain the same number of models from each track. The blue crosses show all of the models along the evolutionary track as they vary from ZAMS to TAMS in core-hydrogen abundance and the red crosses show the models selected from this track. The models were chosen via linear transport such that they satisfy Equation \ref{eq:optimal-spacing}. For reference, an equidistant spacing is shown with black points. }% 
    \label{fig:nearly-even}
\end{figure*}


%%%%%%%%%%%%%%%%%%%%%%%%%%%%%%%
%%% Model evaluation %%%%%%%%%%
%%%%%%%%%%%%%%%%%%%%%%%%%%%%%%%
\section{Evaluating the regressor}
\label{sec:evaluation}
In training the random forest regressor, we must determine how many evolutionary tracks $N$ to include, how many models $M$ to extract from each evolutionary track, and how many trees $T$ to use when growing the forest. As such it is useful to define measures of gauging the accuracy of the random forest so that we may evaluate it with different combinations of these parameters. 

By far the most common way of measuring the quality of a random forest regressor is its so-called ``out-of-bag'' (OOB) score \citep[see e.g.\ section 3.1 of][]{breiman2001random}. While each tree is trained on only a subset (or ``bag'') of the stellar models, all trees are tested on all of the models that they did not see. This provides an accuracy score representing how well the forest will perform when predicting on observations that it has not seen yet. We can then use the scores defined in Section \ref{sec:uncertainties} to calculate OOB scores. 

However, such an approach to scoring is too optimistic in this scenario. Since a tree can get models from every simulation, predicting the attributes of a model when the tree has been trained on one of that model's neighbors leads to an artificially inflated OOB score. This is especially the case for quantities like stellar mass, which do not change along the main sequence. A tree that has witnessed neighbors on either side of the model being predicted will have no error when predicting that model's mass, and hence the score will seem artificially better than it should be. 

Therefore, we opt instead to build validation sets containing entire tracks that are left out from the training of the random forest. We omit models and tracks in powers of two so that we may roughly maintain the regular spacing that we have established in our grid of models (refer back to Appendices \ref{sec:grid} and \ref{sec:selection} for details). We have already shown in Figure \ref{fig:evaluation-tracks} these cross-validated scores as a function of the number of evolutionary tracks. In Figure \ref{fig:app-evaluation} we now show the explained variance score as a function of the number of trees in the forest and the number models obtained from each evolutionary track. Although more seems to be better for all quantities, we find that there is not much improvement after about $M=T=16$. It is also interesting to note that the predictions do not suffer very much from using only four models per track. 

\begin{figure*}
    \centering\includegraphics[width=0.5\textwidth,keepaspectratio]{figs/evaluation/legend.png}\\
    \includegraphics[width=0.5\textwidth,keepaspectratio]{figs/evaluation/num_points-ev.pdf}\hfill
    \includegraphics[width=0.5\textwidth,keepaspectratio]{figs/evaluation/num_trees-ev.pdf}\\
    \includegraphics[width=0.5\textwidth,keepaspectratio]{figs/evaluation/num_points-sigma.pdf}\hfill
    \includegraphics[width=0.5\textwidth,keepaspectratio]{figs/evaluation/num_trees-sigma.pdf}\\
    \caption{Explained variance (top) and accuracy per precision score (bottom) of each stellar parameter as a function of the number of models per evolutionary track (left) and the number of trees used in training the random forest (right). } 
    \label{fig:app-evaluation}
\end{figure*}


%%%%%%%%%%%%%%%%%%%%%%%%%%%%%%%
%%% Hare-and-Hound %%%%%%%%%%%%
%%%%%%%%%%%%%%%%%%%%%%%%%%%%%%%
\section{Hare-and-Hound}
\label{sec:hare-and-hound}
Table \ref{tab:hnh-true} lists the true values of the hare-and-hound exercise performed here, and Table \ref{tab:hnh-perturb} lists the perturbed inputs that were supplied to the machine learning algorithm. 

\begin{deluxetable}{ccccccccccc}
\tabletypesize{\scriptsize}
%\rotate
\tablecaption{True values for the hare-and-hound exercise. \label{tab:hnh-true}}
\tablewidth{0pt}
\tablehead{\colhead{Model} & \colhead{R/R$_\odot$} & \colhead{M/M$_\odot$} & \colhead{$\tau$} & \colhead{T$_{\text{eff}}$} & \colhead{L/L$_\odot$} & \colhead{[Fe/H]} & \colhead{Y$_0$} & \colhead{$\nu_{\max}$} & \colhead{$\alpha_{\text{ov}}$} & \colhead{D}}\startdata
0 & 1.705 & 1.303 & 3.725 & 6297.96 & 4.11 & 0.03 & 0.2520 & 1313.67 & No & No \\
1 & 1.388 & 1.279 & 2.608 & 5861.38 & 2.04 & 0.26 & 0.2577 & 2020.34 & No & No \\
2 & 1.068 & 0.951 & 6.587 & 5876.25 & 1.22 & 0.04 & 0.3057 & 2534.29 & No & No \\
3 & 1.126 & 1.066 & 2.242 & 6453.57 & 1.98 & -0.36 & 0.2678 & 2429.83 & No & No \\
4 & 1.497 & 1.406 & 1.202 & 6506.26 & 3.61 & 0.14 & 0.2629 & 1808.52 & No & No \\
5 & 1.331 & 1.163 & 4.979 & 6081.35 & 2.18 & 0.03 & 0.2499 & 1955.72 & No & No \\
6 & 0.953 & 0.983 & 2.757 & 5721.37 & 0.87 & -0.06 & 0.2683 & 3345.56 & No & No \\
7 & 1.137 & 1.101 & 2.205 & 6378.23 & 1.92 & -0.31 & 0.2504 & 2483.83 & No & No \\
8 & 1.696 & 1.333 & 2.792 & 6382.22 & 4.29 & -0.07 & 0.2555 & 1348.83 & No & No \\
9 & 0.810 & 0.769 & 9.705 & 5919.70 & 0.72 & -0.83 & 0.2493 & 3563.09 & No & No \\
10 & 1.399 & 1.164 & 6.263 & 5916.71 & 2.15 & 0.00 & 0.2480 & 1799.10 & Yes & Yes \\
11 & 1.233 & 1.158 & 2.176 & 6228.02 & 2.05 & 0.11 & 0.2796 & 2247.53 & Yes & Yes 
\enddata
\end{deluxetable}

\begin{deluxetable}{ccccc}
\tabletypesize{\scriptsize}
\tablecaption{Supplied (perturbed) inputs for the hare-and-hound exercise. \label{tab:hnh-perturb}}
\tablewidth{0pt}
\tablehead{\colhead{Model} & \colhead{T$_{\text{eff}}$} & \colhead{L/L$_\odot$} & \colhead{[Fe/H]} & \colhead{$\nu_{\max}$}}\startdata
0 & 6236.96 $\pm$ 85 & 4.23 $\pm$ 0.12 & -0.03 $\pm$ 0.09 & 1397.87 $\pm$ 66 \\
1 & 5806.39 $\pm$ 85 & 2.1 $\pm$ 0.06 & 0.16 $\pm$ 0.09 & 2031.82 $\pm$ 100 \\
2 & 5884.56 $\pm$ 85 & 1.23 $\pm$ 0.04 & -0.05 $\pm$ 0.09 & 2625.73 $\pm$ 127 \\
3 & 6421.65 $\pm$ 85 & 1.99 $\pm$ 0.06 & -0.36 $\pm$ 0.09 & 2475.43 $\pm$ 124 \\
4 & 6525.9 $\pm$ 85 & 3.73 $\pm$ 0.11 & 0.14 $\pm$ 0.09 & 1752.33 $\pm$ 89 \\
5 & 6117.96 $\pm$ 85 & 2.18 $\pm$ 0.06 & 0.04 $\pm$ 0.09 & 1890.79 $\pm$ 101 \\
6 & 5740.71 $\pm$ 85 & 0.82 $\pm$ 0.03 & 0.06 $\pm$ 0.09 & 3486.37 $\pm$ 165 \\
7 & 6288.63 $\pm$ 85 & 1.95 $\pm$ 0.06 & -0.28 $\pm$ 0.09 & 2438.7 $\pm$ 124 \\
8 & 6351.2 $\pm$ 85 & 4.28 $\pm$ 0.13 & -0.12 $\pm$ 0.09 & 1294.24 $\pm$ 67 \\
9 & 5997.95 $\pm$ 85 & 0.7 $\pm$ 0.02 & -0.85 $\pm$ 0.09 & 3292.27 $\pm$ 179 \\
10 & 5899.27 $\pm$ 85 & 2.17 $\pm$ 0.06 & -0.031 $\pm$ 0.09 & 1931.63 $\pm$ 101 \\
11 & 6251.49 $\pm$ 85 & 1.99 $\pm$ 0.06 & 0.126 $\pm$ 0.09 & 2356.977 $\pm$ 101
\enddata
\end{deluxetable}


\bibliographystyle{apj.bst}
\bibliography{astero}


\end{document}

