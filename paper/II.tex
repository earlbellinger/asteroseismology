
%%%%%%%%%%%%%%%%%%%%%%%%%%%%%%%%%%%%%%%%%%%%%%%%%%
% Basic setup. Most papers should leave these options alone.
\documentclass[a4paper,fleqn,usenatbib,useAMS]{mnras}

%%%%% AUTHORS - PLACE YOUR OWN PACKAGES HERE %%%%%

% Only include extra packages if you really need them. Common packages are:
%\usepackage[draft]{hyperref}
\usepackage{graphicx}	% Including figure files
\usepackage{amsmath}	% Advanced maths commands
\usepackage{amssymb}	% Extra maths symbols
\usepackage{multicol}        % Multi-column entries in tables
\usepackage{bm}		% Bold maths symbols, including upright Greek
\usepackage{pdflscape}	% Landscape pages
\usepackage{natbib}
\usepackage[export]{adjustbox}
\usepackage{subfig}
%\usepackage{adjustbox}
%\usepackage{layouts}
%\usepackage{rotating}
%\usepackage{pdflscape}
%\usepackage{pdflscape}
\usepackage{todonotes}
\usepackage{physics}
%\usepackage{afterpage}

%%%%%%%%%%%%%%%%%%%%%%%%%%%%%%%%%%%%%%%%%%%%%%%%%%

%%%%%% AUTHORS - PLACE YOUR OWN MACROS HERE %%%%%%

% Please keep new commands to a minimum, and use \newcommand not \def to avoid
% overwriting existing commands. Example:
%\newcommand{\pcm}{\,cm$^{-2}$}	% per cm-squared
\newcommand{\kms}{\,km\,s$^{-1}$} % kilometres per second
\newcommand{\bibtex}{\textsc{Bib}\!\TeX} % bibtex. Not quite the correct typesetting, but close enough

\newcommand{\Dnu}{\ensuremath{\Delta\nu \ }}
\newcommand{\dnu}{\ensuremath{\delta\nu_{0,2} \ }}
\newcommand{\Mo}{\rm{M}_\odot}
\newcommand{\Lo}{\rm{L}_\odot}
\newcommand{\Ro}{\rm{R}_\odot}

\newenvironment{rotatepage}%
    {\pagebreak[4]\global\pdfpageattr\expandafter{\the\pdfpageattr/Rotate 90}}%
    {\pagebreak[4]\global\pdfpageattr\expandafter{\the\pdfpageattr/Rotate 0}}%

\let\oldbibitem\bibitem
\def\bibitem{\vfill\oldbibitem}

%%%%%%%%%%%%%%%%%%%%%%%%%%%%%%%%%%%%%%%%%%%%%%%%%%


% Use vector fonts, so it zooms properly in on-screen viewing software
% Don't change these lines unless you know what you are doing
\usepackage[T1]{fontenc}
\usepackage{ae,aecompl}

% MNRAS is set in Times font. If you don't have this installed (most LaTeX
% installations will be fine) or prefer the old Computer Modern fonts, comment
% out the following line
\usepackage{newtxtext,newtxmath}
% Depending on your LaTeX fonts installation, you might get better results with one of these:
%\usepackage{mathptmx}
%\usepackage{txfonts}

%%%%%%%%%%%%%%%%%%% TITLE PAGE %%%%%%%%%%%%%%%%%%%

% Title of the paper, and the short title which is used in the headers.
% Keep the title short and informative.
\title[Asteroseimic Diagrams]{On the Diagnostic Potential of Classical and Asteroseimic Diagrams for Solar-like Stars on the Main Sequence}

% The list of authors, and the short list which is used in the headers.
% If you need two or more lines of authors, add an extra line using \newauthor
\author[E. P. Bellinger, et al.] {Earl P. Bellinger$^{1,2}$, George C. Angelou$^{1,2}$\thanks{Contact e-mail: \href{mailto:angelou@mps.mpg.de}{angelou@mps.mpg.de}}, Saskia Hekker$^{1,2}$, Sarbani Basu$^{3}$, 
\newauthor Elisabeth Guggenberger$^{1,2}$, and Warrick Ball$^{4}$ 
\\
% List of institutions
$^{1}$Max-Planck-Institut f\"{u}r Sonnensystemforschung, Justus-von-Liebig-Weg 3, 37077 G\"{o}ttingen, Germany\\
$^{2}$ Stellar Astrophysics Centre, Department of Physics and Astronomy, Aarhus University, Ny Munkegade 120, DK-8000 Aarhus C, Denmark \\
$^{3}$ Department of Astronomy, Yale University, New Haven, CT 06520, USA \\
$^{4}$ Institut f\"ur Astrophysik G\"ottingen, Friedrich-Hund-Platz 1, 37077 G\"ottingen, Germany }
% These dates will be filled out by the publisher
\date{Last updated 2015 May 22; in original form 2013 September 5}

% Enter the current year, for the copyright statements etc.
\pubyear{2015}

% Don't change these lines
\begin{document}
\label{firstpage}
\pagerange{\pageref{firstpage}--\pageref{lastpage}}
\maketitle

% Abstract of the paper
\begin{abstract}
We analyse the diagnostic potential of classical and asteroseismic observations for inferring model properties such as age, mass and radius of solar-like stars on the main sequence. We identify useful combinations and provide limits to where the respective diagrams reveal information about the star. We perform rank correlation tests in order to quantify the relative importance of each observable in probing stellar structure and determining the evolutionary history of a star. We also analyse the principal components of classic and asteroseismic observables and show how these form more useful diagnostics for inferring model properties than the original variables considered. 
\end{abstract}

% Select between one and six entries from the list of approved keywords.
% Don't make up new ones.
\begin{keywords}
Stars: oscillations,  asteroseismology
\end{keywords}

%%%%%%%%%%%%%%%%%%%%%%%%%%%%%%%%%%%%%%%%%%%%%%%%%%

%%%%%%%%%%%%%%%%% BODY OF PAPER %%%%%%%%%%%%%%%%%%

\section{Introduction}

Asteroseismology affords the opportunity to constrain stellar ages through accurate inference of the interior structure. To robustly determine the age of an observed star, models that best match the available observables are sought via mathematical optimization \citep{1994ApJ...427.1013B}. Several search strategies have been successfully applied, including exploring a pre-computed grid of models (i.e., grid based modelling, see \citealt{2011ApJ...730...63G, 2014ApJS..210....1C}) or through \emph{in situ} optimization (e.g., genetic algorithms, \citealt{2014ApJS..214...27M}; Markov chain Monte Carlo, \citealt{2012MNRAS.427.1847B}; or downhill simplex, \citealt{2013ApJS..208....4P} to name but a few). 

Ages determined from the best fit model have a wide range of applications in astrophysics. These include characterising extrasolar planetary systems \citep{2015ApJ...799..170C,2015MNRAS.452.2127S}, assessing galactic chemical evolution \todo{XXX cite Fredrick Anders?}, and performing ensemble studies of the Galaxy \citep{2011Sci...332..213C, 2013MNRAS.429..423M, 2014ApJS..210....1C}. However, obtaining accurate ages is not without its difficulties. Both grid-based modelling and \emph{in situ} optimisation are computationally expensive, and furthermore depend on the choices in physics used to construct the model as well as the uncertainties therein \citep{2014A&A...569A..21L}. 

The precision and long temporal observations from the Kepler spacecraft and CoRoT satellite have helped constrain stellar ages of field stars to within 10\% of their main-sequence lifetime \citep{2015MNRAS.452.2127S}. There still remains, and future missions will deliver, measurements where the signal is noisy or taken over a short timescale. In such cases, obtaining individual frequencies (``peak bagging'') is not always possible to do, but summary information about the star can still be determined. Global asteroseismic parameters, such as frequency separations, can be extracted, and these diagnostics still reveal much about the stellar interior. 

To this end, asteroseismic diagrams have had a large role to play in this problem. Such diagrams utilize observable information about the star that contain footprints of their evolution. Like the Hertzsprung-Russell diagram, they have been used to probe the history of the object of interest, and have done so with less ambiguity and degeneracy than in classical spectroscopic diagrams \citep{2013MNRAS.429.3645S}. 

The most renowned of these diagrams is the so-called \citet{1984srps.conf...11C} diagram, which plots the large frequency separation\footnote{See Appendix \ref{sec:seis} for definitions.} (\Dnu), as a function of the small frequency separation (\dnu). The diagram's diagnostic potential has the been the subject of many investigations \citep[see e.g.][]{1987Natur.326..257G, 2009A&A...508..849G, 2011ApJ...743..161W}. In addition, \citet{2005A&A...441.1079M} and \citet{2014A&A...569A..21L} have considered the utility of various frequency ratios \citep{2003A&A...411..215R, 2005MNRAS.356..671O} that combine to eliminate the frequency-dependent effects from improper modelling of the near-surface layers. 

In this study, ... The work carried out here is useful for large ensemble studies and also for applying `warm starts' to dramatically reduce the computational load when fitting stellar models. \todo{Come back to this. More once we finished the paper} 

%%%%%%%%%%%%%%%%%%%%%%%%%%%%%%%%%%%%%%%%%
%%% HR %%%%%%%%%%%%%%%%%%%%%%%%%%%%%%%%%%
%%%%%%%%%%%%%%%%%%%%%%%%%%%%%%%%%%%%%%%%%
\section{The Hertzsprung-Russell Diagram}
From the equations of stellar structure, and particularly the equation of hydrostatic equilibrium, one can trivially derive that stellar mass \textit{M} is related to luminosity \textit{L} by
\begin{equation} \label{eqn:ml1}
  \frac{L}{\Lo} = \left(\frac{M}{\Mo}\right)^a.
\end{equation}
%where $1 < a < 6$, with $a \approx 4$ describing a main-sequence star\footnote{Fits to observational data suggest that $a=4$ for $0.43 \ \Mo < M < 2 \ \Mo$, which is the range of interest for our study.\todo{citation}}%. $a=3.5$ for stars $2 \ \Mo < M < 20 \ \Mo$; and it can be proven that $a=3$ for a completely radiative star.}. 
Fits to observational data suggest that $a\approx 4$ for solar-like stars, i.e.~$0.43 \ \Mo < M < 2 \ \Mo$\todo{citation}. The amount of time a star will burn hydrogen is proportional to the amount of energy it has per its power, \todo{maybe say this better} hence this relation roughly approximates the main-sequence lifetime of the star $\tau_*$: 
\begin{equation}
  \tau_* \propto \frac{M}{L} \approx \frac{M}{M^4} \approx \frac{1}{M^3}.
\end{equation}
The spread along a given isochrone is therefore primarily due to differences in stellar mass, and evolutionary tracks, i.e., changes in the core-hydrogen abundance, are what give rise to the structure found in the H-R diagram. Figure \ref{fig:HR}a shows an H-R diagram of solar-like stars on the main sequence that differ only in their initial mass.\footnote{See Appendix \ref{sec:models} for a detailed description of how the models were generated.} The fact that stellar mass can be explained so well by luminosity and temperature has formed the basis of the diagram's usefulness over the last century. This is particularly the case in the study of clusters, where stars are assumed to be co-natal and hence all stars in the system are the same age and born with the same composition. %The main purpose of the H-R diagram is of course to ascertain the evolutionary phase of a star. 
%As mass is the principal determinant in stellar evolution, we have coloured each model accordingly. In doing so, we highlight some important facets for our later analysis, but in a familiar parameter space. 

\begin{figure}
\centering
%\includegraphics[width=0.5\linewidth,keepaspectratio]{figs/diags/HR-M-linear.png}\hfill
%\includegraphics[width=0.5\linewidth,keepaspectratio]{figs/diags/HR-M.png}\\
%\subfloat[Varied in $M$, $Y_0$, $Z_0$, and $\alpha_{\text{MLT}}$]{\includegraphics[width=0.5\linewidth,keepaspectratio]{figs/diags/HR-M.png}}
%\subfloat[Varied only in mass]{{\includegraphics[width=0.5\linewidth,keepaspectratio]{figs/diags/HR-M-linear.png}}}%\\

%\subfloat[Contour plot of (a)]{\includegraphics[width=0.5\linewidth,keepaspectratio]{figs/diags/mesh-HR-M-linear.pdf}}%
\subfloat[H-R diagram of evolutionary model tracks varied by initial mass]{\includegraphics[width=\linewidth,keepaspectratio]{figs/diags/mesh-HR-M-linear.pdf}}\\%hfill
\subfloat[H-R diagram of tracks varied in $M$, $Y_0$, $Z_0$, and $\alpha_{\text{MLT}}$]{\includegraphics[width=\linewidth,keepaspectratio]{figs/diags/mesh-HR-M.pdf}}
\caption{Hertzsprung-Russell diagrams of solar-like stars on the main sequence drawn with contour lines of constant mass. On the left, evolutionary model tracks were generated with the solar composition and mixing length parameter, but varied in initial mass. In this regime, it is possible to determine a star's mass from its position in the HR diagram alone. On the right, however, model tracks have been varied not only in mass, but also by initial composition and mixing length. In this more realistic diagram, it is not possible to resolve a star's mass solely from its location in the H-R diagram. } 
\label{fig:HR}
\end{figure}

There are, however, many more dependencies that are present when resolving the mass of a star. In reality, a star's position in the H-R diagram depends not only on its mass, but also on its helium and metal abundances, which manifest through opacities and nucleosythesis. This leads to a well-known degeneracy in the diagram \citep[see e.g.][]{2005A&A...436..127J}. Figure \ref{fig:HR}b shows models that have been varied not only by mass, but also in their initial composition  Here it can be seen that in general resolving individual masses from the H-R diagram actually comes with great uncertainty when we cannot reduce the number of unknowns. Therefore, while it remains invaluable as a tool for understanding stellar evolution, the H-R diagram is inadequate on its own for precise determination of stellar parameters. 

A further complication for using the H-R diagram as a diagnostic for stellar parameters is the fact that temperatures and luminosities are so highly correlated: main-sequence stars with high luminosities also have high temperatures. The estimate of a star's mass from temperature alone is therefore not improved much by supplementing with the information of its luminosity. A better diagnostic would be parameters that are uncorrelated with each other, but correlated with the parameter of interest\footnote{This is of course achievable due to the fact that correlation does not have the transitive property \citep[see e.g.][]{10.2307/2685695}, i.e.~, corr($X,Y$) $\wedge$ corr($Y,Z$) $\not \Rightarrow$ corr($X,Z$).}. %This straightforward example, albeit with some simplifying mathematical approximations, makes a clear point about the prospective of using observable attributes as a diagnostic for stellar parameters. % hmm... % a teaser of what we are about to do.  Many physical components required to explain any one property but we are searching for combinations of parameters that are powerful diagnostics of a given property. The example above has a known mathematical relation. But we want to apply the same technique to other parameters without a priori knowledge of their (in inevitably non-linear) relationship. 


%%%%%%%%%%%%%%%%%%%%%%%%%%%%%%%
%%% C-D %%%%%%%%%%%%%%%%%%%%%%%
%%%%%%%%%%%%%%%%%%%%%%%%%%%%%%%
\section{The Christensen-Dalsgaard Diagram} %(\dnu \ vs.~\Dnu)}
The potential diagnostic properties of the C-D diagram, particularly for stellar ages, was first proposed by \citet{1984srps.conf...11C}. The diagram does make use of parameters that are correlated but exploits the fact \Dnu and \dnu (xx only get 02 from observations) provide evolutionary information from different regions of the star. From eqn x it can be seen that \Dnu is the inverse sound travel time across the ($\ell$ dependent) acoustic cavity (and back).  
\citet{1986ApJ...306L..37U} demonstrated that this parameter is proportional to the square root of the mean density of the star hence traces the radius of the star through its evolution which is qualitatively understood; it implies \Dnu will decrease with main-sequence evolution. 


In the case of \dnu, \citet{1986HiA.....7..283G} showed that it is proportional to the sound speed gradient of the star. The quantity compares averages of the sound speed profile for different spherical degrees, which probe the star to slightly different depths. We are offered the most sensitivity in the inner regions where the profiles differ most. Changes in the mean molecular weight ($\mu$) are inversely proportional to sound speed thus this parameter will also decrease as $\mu$ increases through nucleosynthesis. 





\iffalse 
\begin{figure}
\centering
\includegraphics[width=\linewidth,keepaspectratio]{figs/diags/JCD-age.png}\\%
\includegraphics[width=\linewidth,keepaspectratio]{figs/diags/JCD-Hc.png}\\%
\includegraphics[width=\linewidth,keepaspectratio]{figs/diags/JCD-M.png}%
\caption{Christensen-Dalsgaard ($\dnu$ vs.~$\Dnu$) diagrams for solar-like stellar models on the main sequence. The top diagram is coloured by age and the bottom diagram is coloured by mass. From top to bottom, the models are coloured by age, core hydrogen, and mass, respectively. Age and core hydrogen can be seen to vary only with $\dnu$, whereas mass varies only with $\Dnu$. A large degeneracy between all of the parameters can be seen in the later stages of the main-sequence lifetime for stars of all masses, as is shown in the bottom-left corner of the diagram. }%
\label{fig:CD-vary}
\end{figure}

\begin{figure}
\centering
\includegraphics[width=\linewidth,keepaspectratio]{figs/hrcdr/r_sep-age.png}\\%
\includegraphics[width=\linewidth,keepaspectratio]{figs/hrcdr/r_sep-M.png}\\
\includegraphics[width=\linewidth,keepaspectratio]{figs/hrcdr/r_sep-Y.png}\\%
\includegraphics[width=\linewidth,keepaspectratio]{figs/hrcdr/r_sep-Z.png}
\caption{Ratio separation ($r_{0,2}$ vs.~$r_{1,3}$) diagrams for stellar models in our grid. From top to bottom, models are coloured by age, mass, initial helium, and initial metallicity. Each colour represents 1/10 of the quantity varied, with red being low and blue being high.}%
\label{fig:rsep-vary}
\end{figure}
\fi

%\pagebreak[4]
%\global\pdfpageattr\expandafter{\the\pdfpageattr/Rotate 90}
%\begin{rotatepage}
%\begin{landscape}
%\iffalse
\begin{figure*}
\centering
%\begin{adjustbox}{minipage=[r][\textwidth][b]{\textheight},width=\textheight,height=\textwidth,keepaspectratio,rotate=90}
\includegraphics[width=0.5\textwidth,keepaspectratio]{figs/diags/mesh-JCD-M.pdf}%\hfill
\includegraphics[width=0.5\textwidth,keepaspectratio]{figs/diags/mesh-JCD-age.pdf}%\\%\hfill
%\includegraphics[width=0.5\textheight,keepaspectratio]{figs/diags/mesh-JCD-Hc.pdf}\\
%\includegraphics[width=0.5\textheight,keepaspectratio]{figs/diags/mesh-JCD-He.pdf}
\caption{Christensen-Dalsgaard ($\dnu$ vs.~$\Dnu$) diagrams for solar-like stellar models on the main sequence. Contours of constant mass are plotted on the left, and ones of constant age are on the right. As before, model tracks have been varied not only in mass but also in initial helium abundance, metallicity, and mixing length parameter. Unlike the case where only mass has been varied, it is not possible to determine stellar mass or age from the C-D diagram alone. } %From top to bottom and left to right, the contours delineate age, core hydrogen, mass, and helium, respectively. }%
\label{fig:JCD-meshes}
%\end{adjustbox}
\end{figure*}
%\begin{figure*}
%\vbox to220mm{\vfil Landscape figure to go here. \caption{} \vfil}
%\input{mesh.tex}
%\label{landfig}
%\end{figure*}
%\end{landscape}
%\end{rotatepage}
%\pagebreak[4]
%\afterpage{\global\pdfpageattr\expandafter{\the\pdfpageattr/Rotate 0}}


%%%%%%%%%%%%%%%%%%%%%%%%%%%%%%%
%%% Freq Ratios %%%%%%%%%%%%%%%
%%%%%%%%%%%%%%%%%%%%%%%%%%%%%%%
\section{Frequency Ratio Diagrams}



%%%%%%%%%%%%%%%%%%%%%%%%%%%%%%%
%%% Correlation %%%%%%%%%%%%%%%
%%%%%%%%%%%%%%%%%%%%%%%%%%%%%%%
\section{Rank Correlation Testing}

\begin{figure*}
\centering
\includegraphics[width=\linewidth, keepaspectratio, trim={0 0 2.5cm 0}, clip]{figs/corr-spearman.pdf}
\caption{Spearman's rank correlation coefficient $\rho$ between model properties and observables. Input variables (mass, helium, metallicity, and mixing length parameter) are labelled in deep purple. A positive correlation value (coloured in blue) indicate that the quantities change monotonically together (either both increase or both decrease), and a negative value (red) indicates that they diverge monotonically. All correlations are highly significant with p-values smaller than machine precision, excepting the entries indicated with a cross. The variables are ordered by the first principal component of the correlation matrix. } 
\label{fig:corr}
\end{figure*}

The results of the correlation analysis demonstrate many well-understood relations in stellar evolution (e.g.~mass-age, metallicity-temperature) and also quantifies new dependencies between asteroseismic variables. Of particular note are the relations having to do with age: we can see that the small frequency separations and the asteroseismic frequency ratios decrease monotonically as the star evolves regardless of its initial conditions. In contrast, however, there is no correlation between $\Dnu$ and stellar age, making it a far less valuable diagnostic in that respect than previously believed. 

We note that the diagram requires some care in its interpretation. For example ...x is correlated with y in such a way only because ... 

\begin{table*} \centering
  \caption{}
  \label{}
\begin{tabular}{@{\extracolsep{5pt}} ccccccccc}
\\[-1.8ex]\hline
\hline \\[-1.8ex]
 & M & $Y_0$ & $Z_0$ & $\alpha_{\text{MLT}}$ & $\tau$ & R & H$_c$ & X(He) \\
\hline \\[-1.8ex]
T$_{\text{eff}}$ & $0.680$ & $0.227$ & $$-$0.482$ & $0.160$ & $$-$0.450$ & $0.738$ & $$-$0.267$ & $0.349$ \\
Fe/H & $0.057$ & $0.107$ & $0.980$ & $0.056$ & $0.086$ & $$-$0.126$ & $0.137$ & $$-$0.042$ \\
$\log g$ & $$-$0.795$ & $$-$0.153$ & $0.124$ & $0.207$ & $0.064$ & $$-$0.972$ & $0.655$ & $$-$0.491$ \\
L & $0.862$ & $0.173$ & $$-$0.259$ & $$-$0.005$ & $$-$0.347$ & $0.945$ & $$-$0.434$ & $0.395$ \\
$\langle\Delta\nu\rangle$ & $$-$0.849$ & $$-$0.130$ & $0.101$ & $0.175$ & $0.139$ & $$-$0.990$ & $0.593$ & $$-$0.440$ \\
$\langle\frac{d\Delta\nu}{d\nu}\rangle$ & $0.665$ & $0.163$ & $$-$0.325$ & $$-$0.282$ & $$-$0.109$ & $0.874$ & $$-$0.574$ & $0.461$ \\
$\langle\delta\nu_{02}\rangle$ & $$-$0.327$ & $0.002$ & $$-$0.153$ & $0.047$ & $$-$0.636$ & $$-$0.566$ & $0.922$ & $$-$0.490$ \\
$\langle\frac{d\delta\nu_{02}}{d\nu}\rangle$ & $0.115$ & $0.006$ & $0.061$ & $$-$0.181$ & $$-$0.255$ & $0.021$ & $0.373$ & $$-$0.212$ \\
$\langle r_{02}\rangle$ & $0.457$ & $0.154$ & $$-$0.331$ & $$-$0.130$ & $$-$0.904$ & $0.316$ & $0.503$ & $$-$0.129$ \\
$\langle\frac{dr_{02}}{d\nu}\rangle$ & $$-$0.444$ & $$-$0.028$ & $0.085$ & $$-$0.043$ & $$-$0.117$ & $$-$0.588$ & $0.635$ & $$-$0.379$ \\
$\langle r_{01}\rangle$ & $0.662$ & $0.196$ & $$-$0.340$ & $$-$0.161$ & $$-$0.644$ & $0.707$ & $$-$0.046$ & $0.224$ \\
$\langle\frac{dr_{01}}{d\nu}\rangle$ & $$-$0.546$ & $$-$0.109$ & $0.138$ & $0.006$ & $$-$0.064$ & $$-$0.730$ & $0.692$ & $$-$0.480$ \\ \hline
$\langle\delta\nu_{13}\rangle$ & $$-$0.411$ & $$-$0.005$ & $$-$0.153$ & $0.069$ & $$-$0.556$ & $$-$0.648$ & $0.907$ & $$-$0.491$ \\
$\langle\frac{d\delta\nu_{13}}{d\nu}\rangle$ & $0.490$ & $0.084$ & $0.083$ & $$-$0.234$ & $$-$0.341$ & $0.482$ & $0.088$ & $0.010$ \\
$\langle r_{13}\rangle$ & $0.427$ & $0.150$ & $$-$0.349$ & $$-$0.123$ & $$-$0.878$ & $0.276$ & $0.510$ & $$-$0.145$ \\
$\langle\frac{dr_{13}}{d\nu}\rangle$ & $$-$0.486$ & $0.014$ & $0.142$ & $0.013$ & $$-$0.169$ & $$-$0.648$ & $0.715$ & $$-$0.389$ \\
\hline \\[-1.8ex]
\end{tabular}
\end{table*}


%%%%%%%%%%%%%%%%%%%%%%%%%%%%%%%
%%% PCA %%%%%%%%%%%%%%%%%%%%%%%
%%%%%%%%%%%%%%%%%%%%%%%%%%%%%%%
\section{Principal Components Analysis}

\begin{figure}
\centering
\includegraphics[width=\linewidth, keepaspectratio, trim={3cm 0 3cm 0}, clip]{figs/corr-pca.pdf}
\caption{Spearman's rank correlation coefficient $\rho$ between model properties and principal components. } 
\label{fig:corrpca}
\end{figure}


%%%%%%%%%%%%%%%%%%%%%%%%%%%%%%%
%%% Conclusions %%%%%%%%%%%%%%%
%%%%%%%%%%%%%%%%%%%%%%%%%%%%%%%
\section{Conclusions}
We did stuff. 


%%%%%%%%%%%%%%%%%%%%%%%%%%%%%%%
%%% Acknowledgements %%%%%%%%%%
%%%%%%%%%%%%%%%%%%%%%%%%%%%%%%%
\section*{Acknowledgements} The research leading to the presented results has received funding from the European Research Council under the European Community's Seventh Framework Programme (FP7/2007-2013) / ERC grant agreement no 338251 (StellarAges). 

Analysis in this manuscript was performed with the R software package \citep{R} and the libraries magicaxis \citep{magicaxis}, RColorBrewer \citep{RColorBrewer}, akima \citep{akima}, parallelMap \citep{parallelMap}, data.table \citep{data.table}, lattice \citep{lattice}, and corrplot \citep{corrplot}. 

\bibliographystyle{mnras.bst}
\bibliography{astero}


%%%%%%%%%%%%%%%%%%%%%%%%%%%%%%%
%%% Appendix %%%%%%%%%%%%%%%%%%
%%%%%%%%%%%%%%%%%%%%%%%%%%%%%%%
\appendix

%%%%%%%%%%%%%%%%%%%%%%%%%%%%%%%
%%% Model generation %%%%%%%%%%
%%%%%%%%%%%%%%%%%%%%%%%%%%%%%%%
\section{Model generation}
\label{sec:models}
We used the open-source 1D stellar evolution code \emph{Modules for Experiments
in Stellar Astrophysics} \citep[MESA,][]{2015ApJS..220...15P} to generate 864,702 main-sequence stellar models across 4,485 solar-like evolutionary tracks varied in initial mass $M$, helium $Y_0$, metallicity $Z_0$, and mixing length parameter $\alpha_{\text{MLT}}$. We used MESA version r7623 with a blended, radiative Helmholtz equation of state, default nuclear reaction network, Krishna-Swamy atmospheres, and OPAL type-two opacity tables. \todo{EOS info not precise enough. Tables? Blends? which ones?. Which network how many species? Does it have label or did you generate a custom network?}

The starting points of each track were chosen in a quasi-random fashion so as to populate the initial-condition hyperspace as rapidly as possible (see Appendix \ref{sec:grid} for more details). The tracks were evolved from zero-age main sequence (ZAMS) to either the age of the Universe ($\tau \leq 13.8$ Gyr) or until terminal age main-sequence (TAMS), which we defined as having a core hydrogen fraction ($H_c$) below $10^{-4}$. We selected the same number of models from each evolutionary track (see Appendix \ref{sec:selection} for details) and eliminated all tracks having models with temperatures exceeding 7000 K, as oscillations are not observed in such stars \todo{citation}. 

%%%%%%%%%%%%%%%%%%%%%%%%%%%%%%%
%%% Grid strategy %%%%%%%%%%%%%
%%%%%%%%%%%%%%%%%%%%%%%%%%%%%%%
\section{Initial grid strategy}
\label{sec:grid}
The initial conditions of a stellar model can be viewed as a tesseract with dimensions $M$, $Y_0$, $Z_0$, and $\alpha_{\text{MLT}}$. In most experiments, only one of these dimensions are varied at a time. Here we construct a grid of stellar models with all quantities varied simultaneously. Instead of varying the quantities in a linear fashion, however, we opted for a \emph{quasi-random} grid of evolutionary tracks. 

A linear grid subdivides all dimensions in which initial quantities can vary into equal parts and creates a track of models for every combination of these subdivisions. Although in the limit such a strategy will eventually fill the hypercube of initial conditions, it does so very slowly. It is furthermore suboptimal in the sense that linear grids maximize redundant information, as each varied quantity is tried with the exact same values of all other parameters that have been considered already. In a high-dimensional setting, if any of the parameters are irrelevant to the task of the computation, then the \emph{majority} of the tracks in a linear grid will not contribute \emph{any} new information. Figure \ref{fig:corr} revealed many such irrelevant values, e.g.~between $\alpha_{\text{MLT}}$ and stellar age. 

A refinement on this approach is to create a grid of models with \emph{randomly} varied initial conditions. Such a strategy fills the space more rapidly, and furthermore solves the problem of redundant information. However, this approach suffers from a new problem: since the points are generated at random, they tend to ``clump up'' at random as well, and this results in random gaps in the parameter space, which are obviously undesirable. %This occurs simply because randomly generated points, while uniformly selected, do not fill each dimension uniformly. 

Therefore, in order to select points that do not stack, do not clump, and also fill the space as rapidly as possible, we generate Sobol numbers \citep{sobol1967distribution} in the unit 4-cube and map them to the parameter ranges of each quantity that we want to vary ($M_0$, $Y_0$, $Z_0$, $\alpha_{\text{MLT}}$). By doing this, we both minimize redundant information and furthermore sample the space of possible stars as uniformly as possible. Figure \ref{fig:grids} visualizes the different methods of generating multidimensional grids: linear, random, and the quasi-random strategy that we took; and Figure \ref{fig:inputs} shows a plot of the initial model conditions for all of the evolutionary tracks in our grid. 

\begin{figure*}
\centering
\subfloat[Linear]{{\includegraphics[width=0.333\textwidth,keepaspectratio]{figs/grid-linear.png} }}%
%\hfill
\subfloat[Random]{{\includegraphics[width=0.333\textwidth,keepaspectratio]{figs/grid-random.png} }}%
%\hfill
\subfloat[Quasi-random]{{\includegraphics[width=0.333\textwidth,keepaspectratio]{figs/grid-quasirandom.png} }}%
\caption{Methods of generating multidimensional grids. Linear grids exhaust each dimension uniformly, but with all points stacked on top of each other, so the unit cube is filled very slowly. Random grids fill the unit cube more rapidly, but points tend to clump up and leave large gaps in the parameter space. Quasi-random grids achieve the best of both worlds and fill the unit cube most rapidly by generating points that are maximally distant along all dimensions. }%
\label{fig:grids}
\end{figure*}

\begin{figure}
\centering
\includegraphics[width=\linewidth,keepaspectratio,trim={.5cm .5cm .5cm .5cm},clip]{figs/inputs.png}%trim={1.15cm 1.15cm 1.15cm 1.15cm},clip]{figs/inputs.pdf}
\caption{Scatterplot matrix of MESA model track initial conditions. Initial mass ($M_0$), helium ($Y_0$), metallicity ($Z_0$) and mixing length parameter ($\alpha_{\text{MLT}}$) were varied in a quasi-random fashion to obtain a low-discrepancy grid. Points are colored by their initial hydrogen $H_0=1-Y_0-Z_0$, with red being low $H_0$ (62\%) and purple being high $H_0$ (78\%). The space is densely populated with evolutionary tracks of maximally different initial conditions.}%Points are coloured by their maximum age on the main sequence, with red being zero-age and purple being the age of the Universe. }
\label{fig:inputs}
\end{figure}


%%%%%%%%%%%%%%%%%%%%%%%%%%%%%%%
%%% Model selection %%%%%%%%%%%
%%%%%%%%%%%%%%%%%%%%%%%%%%%%%%%
\section{Model selection}
\label{sec:selection}

In order to prevent statistical bias towards the evolutionary tracks that generate the most models, i.e.~the ones that require the most careful calculations and therefore use smaller time-steps, or those that live on the main sequence for a longer amount of time; we select $N=100$ models from each evolutionary track such that the models are as evenly-spaced in core-hydrogen abundance as possible. Starting from the original vector of length $M$ of hydrogen abundances $\vec H$, we find the subset of length $N$ that is closest to the optimal spacing $\vec B$, where
\begin{equation}
  \vec B \equiv \qty[
    H_{\text{TAMS}}, 
    \ldots,
    %\min\qty(\vec H)+\frac{\max\qty(\vec H)-\min\qty(\vec H)}{99},
    \frac{(N-i)\cdot H_{\text{TAMS}}+H_{\text{ZAMS}}}{N-1}, 
    \ldots, 
    H_{\text{ZAMS}}
  ]
\end{equation}
with $H_{\text{ZAMS}}$ being the core-hydrogen abundance at zero-age and $H_{\text{TAMS}}$ being that at the end of the star's main-sequence lifetime. To obtain the closest possible vector to $\vec B$ from our data $\vec H$, we solve a so-called ``transportation problem'' using integer optimization. First we set up a cost matrix $\boldsymbol{C}$ consisting of absolute differences between the original abundances $\vec H$ and the ideal abundances $\vec B$:
\begin{equation}
  \boldsymbol C \equiv 
  \begin{bmatrix}
    \abs{B_1-H_1} & \abs{B_1-H_2} & \dots & \abs{B_1-H_M} \\ 
    \abs{B_2-H_1} & \abs{B_2-H_2} & \dots & \abs{B_2-H_M} \\ 
    \vdots & \vdots & \ddots & \vdots\\ 
    \abs{B_N-H_1} & \abs{B_N-H_2} & \dots & \abs{B_N-H_M}
  \end{bmatrix}.
\end{equation}
We then require that exactly $N$ values are selected from $\vec H$, and that each value is selected no more than one time. We call the optimal solution matrix by $\boldsymbol S$, and find it by minimizing the cost matrix subject to the following constraints:
\begin{align}
  \hat S = \underset{\boldsymbol S}{\arg\min} \; & \sum_{ij} S_{ij} C_{ij} \notag\\
  \text{subject to } & \sum_j S_{ij} = 1 \; \text{ for all } i=1\ldots N \notag\\
  \text{and } & \sum_i S_{ij} \leq 1 \; \text{ for all } j=1\ldots M.
\end{align}
Then the items from $\vec H$ that are most near to being equidistantly-spaced are found in the columns of $\boldsymbol S$ that contain ones, and we are done. 


%%%%%%%%%%%%%%%%%%%%%%%%%%%%%%%
%%% Seismology %%%%%%%%%%%%%%%%
%%%%%%%%%%%%%%%%%%%%%%%%%%%%%%%
\section{Seismological calculations}
\label{sec:seis}
We used ADIPLS \citep{2008Ap&SS.316..113C} to calculate all p-mode oscillations up to spherical degree $\ell=3$ and below the acoustic cut-off frequency. We define a frequency separation $\nabla$ as the difference between a frequency $\nu$ of spherical degree $\ell$ and radial order $n$ and another frequency, that is:
\begin{equation} 
  \nabla_{\ell_1, \ell_2}(n_1, n_2) \equiv \nu_{\ell_1}(n_1) - \nu_{\ell_2}(n_2).
\end{equation}
The large frequency separations is then
\begin{equation} 
  \Delta\nu_\ell(n) \equiv \nabla_{\ell, \ell}(n, n-1)
\end{equation}
and the small frequency separation is
\begin{equation}
  \delta\nu_{\ell, \ell+2}(n) \equiv \nabla_{\ell, \ell+2}(n, n-1).
\end{equation}
The ratios between the large and small frequency separation have been shown to be insensitive to the surface term and are therefore valuable asteroseismic diagnostics of stellar interiors \citep{2003A&A...411..215R}. They are defined as
\begin{equation} 
  r_{\ell_1,\ell_2}(n) \equiv \frac{\delta\nu_\ell(n)}{\Delta\nu_{1-\ell}(n+\ell)}
\end{equation}
\begin{align} 
  %r_{0, 1}(n) \equiv \frac{dd_{0,1}(n)}{\Delta\nu_{1}(n)}
  r_{0, 1}(n) \equiv \frac{1}{8\Delta\nu_{1}(n)} [&\nu_0(n-1) - 4\nu_1(n-1) + 6\nu_0(n) \notag\\
  & - 4\nu_1(n) + \nu_0(n+1)] .
\end{align}
%where
%\begin{align}
%  dd_{0,1}(n) = \frac{1}{8} [ &\nu_0(n-1) - 4\nu_1(n-1) + 6\nu_0(n) \notag\\
%  & - 4\nu_1(n) + \nu_0(n+1) ]
%\end{align}
Since the set of radial orders that are observable differs from star to star, we collect summary statistics on $\Dnu$, $\delta\nu_{0,2}$, $\delta\nu_{1,3}$, $r_{0,2}$, $r_{1,3}$, and $r_{0,1}$\footnote{We omit $r_{1,0}$ because we have found it nearly identical with $r_{0,1}$.}. We mimic the range of observable frequencies in our models by weighting all frequencies by their position in a Gaussian envelope centered at the predicted frequency of maximum oscillation power $\nu_{\max}$ and having full-width at half-maximum of $0.66\nu_{\max}{}^{0.88}$ as per the prescription given by \citet{2012A&A...537A..30M}. We then calculate the weighted median of each variable and furthermore apply weighted linear regression to obtain slopes. Illustrations of this technique applied to the frequencies of the \emph{Sun as a Star} data collected by BiSON \citep{2014MNRAS.439.2025D} are shown in Figure \ref{fig:ratios}.

\begin{figure*}
\centering
\includegraphics[width=0.5\linewidth,keepaspectratio]{figs/freqs/Dnu-Sun.pdf}\hfill%
\includegraphics[width=0.5\linewidth,keepaspectratio]{figs/freqs/dnu02-Sun.pdf}\\
%\includegraphics[width=0.5\textwidth,keepaspectratio]{figs/freqs/dnu13-Sun.pdf}%
\includegraphics[width=0.5\linewidth,keepaspectratio]{figs/freqs/r_avg01-Sun.pdf}\hfill%\\
\includegraphics[width=0.5\linewidth,keepaspectratio]{figs/freqs/r_sep02-Sun.pdf}%
%\includegraphics[width=0.5\textwidth,keepaspectratio]{figs/freqs/r_sep13-Sun.pdf}%
\caption{The large and small frequency separations and frequency ratios of the Sun. The vertical dotted line indicates $\nu_{\max}$. Point sizes are proportional to the weighting applied in the linear approximation.}%
\label{fig:ratios}
\end{figure*}


%%%%%%%%%%%%%%%%%%%%%%%%%%%%%%%
%%% Contour %%%%%%%%%%%%%%%%%%%
%%%%%%%%%%%%%%%%%%%%%%%%%%%%%%%
\section{Contour plots}
\label{sec:contour}
Description of contour plotting here. 

\begin{figure*}
\centering
\subfloat[Contour]{\includegraphics[width=0.5\linewidth, keepaspectratio]{figs/diags/mesh-JCD-Hc.pdf}}\hfill
\subfloat[Scatter]{\includegraphics[width=0.5\linewidth, keepaspectratio]{figs/diags/JCD-Hc.png}}
\caption{The C-D diagram with contours of constant core-hydrogen abundance (a). This diagram was created using the points from the scatter plot shown in panel (b).}
\label{fig:contscat}
\end{figure*}


% Don't change these lines
\bsp	% typesetting comment
\label{lastpage}
\end{document}

% End of mnras_guide.tex
