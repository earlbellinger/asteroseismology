\section{Theory of Stellar Pulsations} 
\label{sec:pulsation}
\begin{shaded}
\noindent The purpose of this section is to give the reader a sufficient background summary on non-radial stellar pulsations in order to be able to understand the remainder of this thesis. 
I draw heavily here from the numerous textbooks that have been written on stellar pulsations, which include works by \cite{1926ics..book.....E}, \cite{1949ptvs.book.....R}, \cite{1979nos..book.....U}, \cite{1980tsp..book.....C}, \cite{2010aste.book.....a}, and \cite{basuchaplin2017}. 
Additionally, the long reviews by \cite{1958HDP....51..353L}, \citet{1993afd..conf..399G}, and \citet{2016lrsp...13....2b} are valuable references.
I will perform calculations in this section using the \emph{Aarhus adiabatic oscillation package} \citep[\textsc{ADIPLS},][]{2008Ap&SS.316..113C}. 
\end{shaded}

Observations of stellar pulsations grant a new kind of insight into the behavior of stars. 
Whereas classical measurements of stars probe the stellar surface, observations of stellar pulsations, which traverse the stellar interior, bring deeper information to light. 
Measurements of stellar pulsations provide stringent tests on the processes of stellar evolution, as the frequencies of pulsation profoundly depend on the predicted stellar structure. 
Stars exhibiting solar-like oscillations are particularly valuable for this pursuit. 
These stars vibrate in a superposition of a great number of oscillation modes simultaneously, and each mode that can be observed provides additional information that can be used to constrain stellar models. 

The pulsation hypothesis of stellar variability is supported by the fact that the theoretical pulsations of stellar models generally match the observed pulsations of stars. 
Furthermore, theoretically predicted pulsations in stars that were previously not observed to be variable (such as red giants) have been now overwhelmingly confirmed. 
That being said, while the agreement with models is very good, it is not perfect. 
In this section, I will outline the theory of stellar pulsations, thereby allowing us to calculate the time-independent adiabatic pulsation frequencies of our stellar models. 
I will compare the frequencies of my solar-calibrated model to measurements of the Sun. 
I will furthermore present the kernel functions of stellar structure, which quantify how changes to the stellar structure translate into changes in pulsation frequencies. 
This will allow me to state the structure inverse problem: i.e., the problem of determining a star's structure using only asteroseismic arguments. 



\subsubsection*{Assumptions} 

I again begin with my assumptions. In addition to the assumptions for stellar structure, I assume: 

\begin{enumerate} 
    \item \emph{The stellar structure is nearly static.} 
    I ignore all time derivatives (including velocities) in the equilibrium structure of the star. 
    Thus, I am considering only time-independent pulsation frequencies. 
    Clearly, stars evolve over time---the entire preceding section was based on that fact. 
    That said, the evolutionary timescale in the stars considered here (billions of years) is far greater than the pulsation timescale (minutes). 
    
    \item \emph{The pulsations are linear perturbations the static stellar structure.} 
    I ignore non-linear perturbations. %
    This assumption should hold when the pulsation amplitudes are much smaller than the speed of sound. 
    As we've seen, solar oscillations have amplitudes around ${10\;\text{cm/s}}$, whereas the speed of sound at the surface of the solar-calibrated model is on the order of ${10\;\text{km/s}}$. 
    
    \item \emph{The pulsations are adiabatic.} 
    I ignore the transfer of energy between the oscillations and the equilibrium stellar structure. 
    This assumption should hold to good approximation when the pulsation time-scale is much smaller than the thermal timescale. 
    With pulsation periods on the order of minutes, this is true for the majority of the stellar interior.
    However, this assumption too breaks down near to the stellar surface. 
    Furthermore, without consideration of non-adiabatic effects, we will be unable to predict mode amplitudes, and we will not be able to determine whether the modes are excited \citep[e.g.,][]{2015EAS....73..111S}. 
    
    \item \emph{The stellar material is inviscid.} 
    I ignore internal friction. 
    Although the viscosity of the solar core is similar to that of honey (${\sim 100\;\text{cm}^2/\text{s}}$, e.g., \citealt{fox2000geophysical}), the Reynolds numbers throughout the solar interior are large enough to justify this assumption. %
    However, this assumption does break down in convection zones, where turbulent viscosity damps the oscillations. 
\end{enumerate} 
Here and in the previous section I have made several assumptions that are violated in the near-surface layers of stars, or in locations where energy is transported by convection. 
These violations will cause errors in the predicted mode frequencies. 
I will introduce a correction to deal with these errors later in the section. 

\subsubsection*{Fluid Dynamics}
Given a static stellar structure, we consider a small perturbation that displaces all quantities (density, pressure, etc.) from equilibrium. %
For example, the stellar density at position $\vec r$ and time $t$ is 
\begin{align}
    \text{(Eulerian perturbation)} && 
    \rho (\vec r, t)
    &= \label{eq:eulerean}
    \rho_0 (\vec r) 
    + 
    \rho' (\vec r, t)
    \\
    \text{(Lagrangian perturbation)} &&
    \delta\rho(\vec r)
    &= \label{eq:lagrangian}
    \rho'(\vec r_0)
    +
    \vec\xi \cdot \nabla\rho_0 (\vec r)
\end{align}
where $\rho_0$ is the equilibrium density, $\rho'$ is the perturbed density, and ${\vec\xi\equiv\vec r - \vec r_0}$ is the displacement in space.
Here I have made use of the assumption that the equilibrium structure does not depend on time. 
The perturbation induces a velocity field $\vec{\varv}$ given by 
\begin{equation}
    \vec{\varv} (\vec r, t) 
    = 
    \frac{\partial}{\partial t}\vec\xi (\vec r, t).
\end{equation}
This velocity field is then controlled by the following equations: 

\begin{description}
    \setlength{\itemindent}{0pt}
    \item[The continuity equation.]
    As we've seen previously, the equation of continuity is a statement of mass conservation (\emph{cf}.~Equation~\ref{eq:cons-mass}). 
    It states that mass cannot teleport through the star, but rather must travel through it continuously. 
    \lr{The equation can be given as 
    \begin{equation} \label{eq:continuity} 
        \frac{\partial \rho}{\partial t}
        +
        \nabla
        \cdot
        \left(
            \rho \vec{\varv} 
        \right)
        =
        0
    \end{equation}
    where $\nabla\cdot$ is the divergence vector operator.} 
    Substituting the perturbed quantities (Equation~\ref{eq:eulerean}) into Equation~\ref{eq:continuity}, we get 
    \begin{equation}
        \frac{\partial}{\partial t} \left[
            \rho_0(\vec r)
            +
            \rho' (\vec r, t)
        \right]
        +
        \nabla\cdot \left\{
            \left[
                \rho_0(\vec r)
                +
                \rho'(\vec r, t)
            \right]
            \frac{\partial\vec\xi}{\partial t} 
        \right\}
        =
        0.
    \end{equation}
    As we have assumed the equilibrium structure to be static, the corresponding time derivatives vanish. 
    Integrating with respect to time, we then obtain
    \begin{equation} \label{eq:perturbed-continuity} \boxed{
        \rho' 
        + 
        \nabla \cdot \left( 
            \rho_0 \vec\xi 
        \right) 
        = 
        0
    }\end{equation}
    i.e., the perturbed equation of continuity. $\hfill\square\;$
    
    \item[The equation of motion.] \lr{To first order, the general Navier--Stokes momentum equation can be expressed as}
    \begin{equation} 
        \rho
        \left(
            \frac{\partial}{\partial t}
            +
            \vec{\varv}
            \cdot
            \nabla 
        \right)
        \vec{\varv}
        =
        -\nabla P
        +
        \mu \nabla^2 \vec{\varv}
        +
        \frac{1}{3} \mu \nabla\left(
            \nabla \cdot \vec{\varv}
        \right)
        +
        \rho \vec{g}
    \end{equation}
    where $\mu$ is the viscosity of the stellar material and $\vec{g}$ is the gravitational acceleration.
    Since I have assumed that the stellar viscosity is negligible, we can obtain 
    \begin{equation} \label{eq:momentum} %
        \rho
        \left(
            \frac{\partial}{\partial t}
            +
            \vec{\varv}
            \cdot
            \nabla 
        \right)
        \vec{\varv}
        =
        -
        \nabla P
        +
        \rho \vec g%
    \end{equation}
    Notice that this equation at equilibrium is the familiar equation of hydrostatic support (\ref{eq:cons-mom-r}):
    \lr{\begin{equation}
        0 = -\nabla P_0 + \rho_0 \vec{g}_0.
    \end{equation}}
    Substituting the perturbations into Equation~(\ref{eq:momentum}) and dropping all higher-order terms, we find the perturbed equation of motion:
    \lr{\begin{equation} \label{eq:perturbed-motion} \boxed{
        \rho_0\,
        \frac{\partial^2 \vec\xi}{\partial t^2}
        =
        -\nabla P'
        -
        \rho_0 \nabla \Phi'
        -
        \rho' \nabla \Phi_0
    }\,.\end{equation}
    Here I have introduced the gravitational potential $\Phi$, the negative gradient of which is the gravitational acceleration: %
    \begin{equation} \label{eq:grav-pot}
        \vec g 
        =
        - \nabla \Phi
        \qquad
        \text{and}
        \qquad
        \Phi(\vec r, t)
        =
        -G
        \int_{V}
            \frac{\rho}{|\vec r - \vec x|}
        \;\text{d}^3 \vec{x}
    \end{equation}
    where $V$ is the volume of the star at equilibrium.} 
    
    \item[Poisson's equation.] %
    Gauss's law for gravity gives that
    \begin{equation}
        \nabla \cdot \vec g
        =
        -4\pi G \rho.
    \end{equation}
    After substituting the gravitational potential and the Eulerian perturbations, we obtain the perturbed Poisson equation to describe the gravitational field: 
    \begin{equation} \label{eq:perturbed-poisson} \boxed{
        \nabla^2 \Phi'
        =
        4\pi G \rho'
    }\,.\end{equation}
    
    \item[The energy equation.]
    The energy equation completes the system by thermodynamically connecting pressure to density. 
    Since I have assumed adiabatic pulsations, the energy equation can be given as
    \begin{equation} %
        \frac{\partial P}{\partial t}
        +
        \vec{\varv}
        \cdot
        \nabla P
        =
        c^2 
        \left(
            \frac{\partial \rho}{\partial t}
            +
            \vec{\varv} \cdot \nabla \rho
        \right)
    \end{equation}
    where $c$ is again the adiabatic speed of sound (\emph{cf.}~Equation~\ref{eq:speed-of-sound}). 
    Substituting the Lagrangian perturbation, we obtain the perturbed energy equation
    \begin{equation} \label{eq:perturbed-energy} \boxed{
        P'
        +
        \vec \xi \cdot \nabla P_0
        =
        c^2_0 \left( 
            \rho' + \vec \xi \cdot \nabla \rho_0
        \right)
    }\,.\end{equation}
\end{description}


\subsubsection*{Symmetry}
Now I will apply the assumption of symmetry and consider only oscillatory solutions on a sphere. 
\lr{I separate the displacement vector into radial and horizontal components 
\begin{equation}
    \vec\xi = \xi_r \hat a_r + \vec \xi_h, 
    \qquad 
    \vec\xi_h = \xi_\theta \hat a_\theta + \xi_\phi \hat a_\phi
\end{equation}
where $\hat a$ are unit vectors in indicated directions. 
The radial component of the displacement, for example, can now be expressed as
\begin{align}
    \xi_r(r, \theta, \phi, t)
    &=
    \xi_r(r)
    Y_{\ell}(\theta, \phi)
    \exp \{
        -i\omega t
    \} %
\end{align}
where $\theta$ and $\phi$ are latitude and longitude, $Y_{\ell}$ is Laplace's spherical harmonic for degree $\ell$ (\emph{cf.}~Figure~\ref{fig:sph}), $i$ is the imaginary unit, and ${\omega=2\pi\nu}$ is the cyclic frequency.}
When $\omega^2$ is real, the solution is oscillatory; when it is imaginary, the solution either grows or delays. 
Substituting the spherical, symmetric, harmonic variables into the previous equations (\ref{eq:perturbed-continuity},\ref{eq:perturbed-motion},\ref{eq:perturbed-poisson},\ref{eq:perturbed-energy}) and dropping subscripts for unperturbed quantities, after some manipulations we may find 
\begin{gather} \label{eq:oscillation1}
    \frac{\text{d}\xi_r}{\text{d}r}
    =
    -\left(
        \frac{2}{r}
        +
        \frac{1}{\Gamma_1 P}
        \frac{\text{d}P}{\text{d}r}
    \right)
    \xi_r
    +
    \frac{1}{\rho c^2}
    \left(
        \frac{S_\ell^2}{\omega^2}
        -
        1
    \right)
    P'
    -
    \frac{\ell(\ell+1)}{\omega^2 r^2}
    \Phi' \vphantom{\Bigg(}
    \\
    \frac{\text{d}P'}{\text{d}r}
    =
    \rho \left(
        \omega^2 - N^2
    \right) 
    \xi_r
    +
    \frac{1}{\Gamma_1 P}
    \frac{\text{d}P}{\text{d}r}
    P'
    +
    \rho \frac{\text{d} \Phi'}{\text{d} r}
    \vphantom{\Bigg(}
    \\
    \frac{1}{r^2}
    \frac{\text{d}}{\text{d}r}
    \left(
        r^2\,
        \frac{\text{d}\Phi'}{\text{d}r}
    \right)
    =
    -4\pi G \left(
        \frac{P'}{c^2}
        +
        \frac{\rho}{g}
        \xi_r
        N^2
    \right)
    +
    \frac{\ell(\ell+1)}{r^2}
    \Phi' \vphantom{\Bigg(}
    \label{eq:oscillation3}
\end{gather}%
\lr{as well as 
\begin{gather}
    \vec\xi_h(r, \theta, \phi, t) 
    =
    \sqrt{4\pi}\,
    \xi_h(r)
    \left(
        \frac{\partial Y_\ell}{\partial \theta}\,
        \hat{a}_\theta
        +
        \frac{1}{\sin\theta}
        \frac{\partial Y_\ell}{\partial \phi}\,
        \hat{a}_\phi
    \right)
    \exp \{
        -i\omega t
    \} \vphantom{\Bigg(}
    \\
    \xi_h(r)
    =
    \frac{1}{r\omega^2} \left( 
        \frac{1}{\rho}\, P' - \Phi'
    \right). \vphantom{\Bigg(}
\end{gather}}
Here I have introduced the \emph{Brunt-V\"ais\"al\"a} and \emph{Lamb} \lr{squared} frequencies: 
\begin{gather} \label{eq:brunt-vaisala}
    N^2
    =
    g
    \left(
        \frac{1}{\Gamma_1}
        \frac{\text{d} \ln P}{\text{d}r}
        -
        \frac{\text{d} \ln \rho}{\text{d}r}
    \right) 
    \\ \label{eq:lamb}
    S^2_\ell
    =
    \frac{\ell(\ell+1)c^2}{r^2}
\end{gather}
which give the regions in the star where modes of different character can propagate. 
The former, $N^2$, describes where g-modes can propagate, so called because their restoring force is gravity. 
The latter, $S_\ell^2$, depending on the spherical degree $\ell$, describes where p-modes can propagate, called as such because their restoring force is the pressure gradient. 
These cavities are visualized in Figure~\ref{fig:propagation}. 
Here it can be appreciated that g-modes and convection are two sides of the same coin: when ${N^2<0}$ the fluid is unstable to convection; otherwise, the fluid is unstable to g-mode oscillations. 

This system of equations (\ref{eq:oscillation1}--\ref{eq:oscillation3}) constitutes a fourth order boundary eigenvalue problem. 
Equipped with suitable boundary conditions, we may numerically calculate the eigenfunctions $\vec\xi$ (see Figure~\ref{fig:eigenfunctions}) and their corresponding eigenfrequencies $\omega$ for a given model of stellar structure. 
This is the forward problem of stellar pulsation. 


\begin{figure}
    \centering
    \includegraphics[width=\textwidth]{ch1_introduction/figs/pulse/prop-solar.pdf}
    \caption[Propagation diagram]{Propagation diagram for a solar model. 
    The blue-shaded area shows the Brunt-V\"ais\"al\"a region where g-modes can propagate (\emph{cf.}~Equation~\ref{eq:brunt-vaisala}). 
    The orange-shaded area shows the ${\ell=1}$ Lamb region where dipolar p-modes can propagate (\emph{cf.}~Equation~\ref{eq:lamb}). 
    Modes are exponentially damped in the evanescent zone; nevertheless, modes of similar frequency can couple in this region, giving rise to mixed modes. 
    The observable region is a few ${\Delta\nu}$ around $\nu_{\max}$; thus, only p-modes are expected to be observed in this range at this stage of evolution. 
    \label{fig:propagation}}
\end{figure}




\begin{figure}
    \centering
    \includegraphics[width=0.5\textwidth]{figs/pulse/eig-radial.pdf}%
    \makebox[0.5\textwidth][c]{%
        \adjustbox{trim={1.7cm 0cm 0cm 0cm},clip}{\includegraphics[width=0.5\textwidth]{figs/pulse/eig-dipole.pdf}}}
    \caption[Eigenfunctions]{Radial (blue) and horizontal (red) normalized eigenfunctions for radial (${\ell=0}$, left) and dipolar (${\ell=1}$, right) oscillation modes, both having radial order ${n=20}$. 
    The radial displacement of the two modes are quite similar, being only slightly offset in the interior and basically identical in the envelope. 
    The horizontal displacement has zero crossings when the radial displacement is maximal, and vice versa. 
    Radial modes lack horizontal displacement by definition. 
    \label{fig:eigenfunctions}}
\end{figure}



\subsubsection*{Some Properties of Solar-like Oscillations}

As we have seen in the first section, oscillation modes of the same spherical degree $\ell$ can differ in their radial order $n$ and be excited simultaneously with different frequencies. 
For solar-type stars, it is currently possible to resolve frequencies for modes of low spherical degree (${0~\leq~\ell~\leq~3}$) and `high' radial order (${8~\leq~n~\leq~31}$). 
The frequency range where oscillation power is maximum, called by $\nu_{\max}$, generally corresponds to around ${n=20}$ or so. 
This region of power is proportional to (and obviously lower than) the acoustic cut-off frequency, i.e., the upper frequency bound for oscillations to be reflected back into the star rather than being lost to space: 
\begin{equation} \label{eq:numax}
    \nu_{\max} \propto \nu_{\text{ac}} \propto \frac{g}{\sqrt{T_{\text{eff}}}}.  %
\end{equation}
For the Sun, ${\nu_{\max,\odot} \simeq 3090\;\mu\text{Hz}}$ (${\sim 5.4}$~minutes) and ${\nu_{\text{ac},\odot} \simeq 5000\;\mu\text{Hz}}$ (${\sim 3.3}$~minutes). 
Since we lack a proper theoretical treatment of convective transport, which both excites and damps the oscillation modes, we are unable to theoretically predict the amplitudes of the oscillations of our solar model. 
In lieu of this, we may try to predict the general region where oscillations with the greatest amplitudes are to be expected by scaling from the observed solar values \citep[e.g.,][]{1995A&A...293...87K}:
\begin{equation}
    \frac{\nu_{\max,\ast}}{\nu_{\max,\odot}}
    =
    \left(
        \frac{M_\ast}{M_\odot}
    \right)
    \left(
        \frac{R_\ast}{R_\odot}
    \right)^{-2}
    \left(
        \frac{T_{\text{eff},\ast}}{T_{\text{eff},\odot}}
    \right)^{-\frac{1}{2}}
\end{equation}
and likewise for the acoustic cutoff frequency. 

\citet{1980ApJS...43..469T} considered oscillation modes in the asymptotic limit of high radial order (${n\gg\ell}$) and found that theoretical mode frequencies form a pattern. 
In particular, adjacent modes of the same spherical degree are approximately equally spaced, which agrees with the observations that we saw in Figures~\ref{fig:solar-power-spectrum} and \ref{fig:16cygb}. 
The pattern of frequencies can be summarized to first-order approximation as 
\begin{equation} \label{eq:asymptotic}
    \nu_{n,\ell} \simeq \Delta\nu \left( n + \frac{\ell}{2} + \epsilon \right)
\end{equation}
where $\nu_{n,\ell}$ is the frequency of mode (${n,\ell}$) and $\epsilon$ is a phase shift (${\epsilon_\odot\simeq 1.6}$). 
The spacing ${\Delta\nu}$ is called the \emph{large frequency separation} and is related to the inverse sound travel time and proportional to the \lr{root mean density} of the star \citep{1986apj...306l..37u, 1995A&A...293...87K}:
\begin{equation}
    \Delta\nu
    \simeq
    \left( 
        2 \int \frac{\text{d}r}{c}
    \right)^{-1}
    \propto
    \left(
        \frac{M}{R^3}
    \right)^{1/2}. 
\end{equation}
Since the large frequency separation gives the spacing between modes of different orders, it can be calculated empirically with
\begin{equation} \label{eq:Dnu}
    \Delta\nu_{n,\ell} 
    =
    \nu_{n,\ell}
    -
    \nu_{n-1,\ell}.
\end{equation}
Calculating the average large frequency separation of the Sun for radial modes using data from the Birmingham Solar Oscillations Network \citep[\emph{BiSON},][]{2009mnras.396l.100b} we can obtain 
\begin{equation}
    \Delta\nu_\odot = 134.8693 \pm 0.0042\;\mu\text{Hz}. 
\end{equation}
This presents an opportunity to test the quality of our solar model. 
We can calculate the large frequency for our solar-calibrated model either using the inverse sound travel time, or using the frequencies themselves. 
In the former case, we obtain ${\Delta\nu = 136.2970\;\mu\text{Hz}}$. 
In the latter, ${\Delta\nu = 136.2208\;\mu\text{Hz}}$. 

On the one hand, these model values differ by only about one percent from the solar values, which is quite good by astrophysical standards. 
On the other hand, when considering the precision with which ${\Delta\nu_{\odot}}$ can be calculated, this is a highly significant ${\sim 300\sigma}$ difference. 
This difference arises due to our ill treatment of the stellar surface, which we will address later in this section. 

A higher-order expansion of the asymptotic expression additionally gives a term known as the \emph{small frequency separation}, the spacing between modes adjacent in frequency and whose spherical degree differs by two \lr{\citep{1980ApJS...43..469T}}: 
\begin{equation} \label{eq:dnu}
    \delta\nu_{n,\ell}
    =
    \nu_{n,\ell}
    -
    \nu_{n-1,\ell+2}
    \simeq
    -(4\ell + 6)
    \frac{\Delta\nu}{4\pi^2 \nu_{n,\ell}}
    \int
        \frac{\text{d}c}{\text{d}r}
        \frac{\text{d}r}{r}.
\end{equation}
As we can see, the small frequency separation is sensitive to the sound speed gradient, and is therefore a good proxy for the conditions in the stellar core, where the sound speed gradient changes sign (\emph{cf.}~Figure~\ref{fig:profs}). 
This makes ${\delta\nu}$ a diagnostic of main-sequence age. 
We will make use of these relations to infer the properties of stars in Chapter~\ref{chap:ML}, and use computational methods to further understand what properties of stars they reflect in Chapter~\ref{chap:statistical}. 
The average small frequency separation between solar oscillation modes with (${\ell=0},\;{\ell=2}$) is
\begin{equation}
    \delta\nu_\odot \simeq 8.957 \pm 0.059\;\mu\text{Hz}
\end{equation}
and for our solar model, ${\delta\nu=8.939\;\mu\text{Hz}}$, which is good agreement. 



\subsubsection*{A Direct Comparison} 
We have just compared our solar model against the asymptotic properties of the solar oscillations, finding good agreement with the small frequency separation but less good agreement with the large frequency separation. 
We may now test the quality of our solar model more directly by comparing the individual pulsation mode frequencies themselves to those observed in the Sun. 
This comparison is shown in Figure~\ref{fig:solar_freq_diffs}. 

Immediately it can be seen that there are systematic discrepancies between the model and the actual mode frequencies on the order of ${10\;\mu\text{Hz}}$, i.e., tenths of a percent, which is a difference in period of about $1$ to $2$ seconds. 
In particular, the disagreement gets worse with increasing frequency. 
This phenomenon is called the \emph{surface effect} and has arisen from our improper modelling of the near-surface layers \citep[e.g.,][]{1984srps.conf...11C}. 
The large frequency separation is also sensitive to surface effects, which is why our model ${\Delta\nu}$ differed so significantly from the observed value. 

It is noteworthy that, because all of the waves propagate essentially radially in the near-surface layers (\emph{cf.}~Figures~\ref{fig:rays} and \ref{fig:eigenfunctions}), the surface term is a function of frequency alone and is independent of the spherical degrees of the modes. 
The surface effect is thus often dealt with by introducing a correction that increases with frequency. %
The \citet{2014A&A...568A.123B} treatment of the surface term fits coefficients $\mathbf a$ to the differences between observed and model frequencies according to
\begin{equation} \label{eq:BallGizon-surfterm}
    \delta \nu_{n,\ell} 
    = 
    \frac{1}{I_{n,\ell}} \left[ 
        a_1 \left( 
            \frac{\nu_{n,\ell}}{\nu_{ac}} 
        \right)^{-1} 
        + 
        a_2 \left( 
            \frac{\nu_{n,\ell}}{\nu_{ac}} 
        \right)^{3} 
    \right] %
\end{equation}
where $\nu_{ac}$ is the acoustic cutoff frequency, with ${\nu_{ac,\odot} \approx 5000}$, and $I_{n,\ell}$ is the normalized mode inertia:
\begin{equation} \label{eq:normalized-mode-inertia}
    I_{n,\ell}
    =
    \frac{4\pi}{M}
    \frac{\int \rho
            \left(
                |\xi_r|^2
                +
                \Ltwo |\xi_h|^2
            \right) 
            r^2
        \;\text{d}r}{
            |\xi_r(r=R)|^2
            +
            \Ltwo
            |\xi_h(r=R)|^2
        }.
\end{equation}
However, Figure~\ref{fig:solar_freq_diffs} further shows that even after correcting for the surface term, differences remain. 
This implies that even beyond the near-surface layers, the structure of the Sun differs from the model. 

\begin{figure}[t]
    \centering
    \includegraphics[width=\textwidth,keepaspectratio,trim={0cm 0cm 0cm 0.1cm}, clip]{figs/pulse/freq-diffs-solar.pdf}%
    \caption[The solar surface effect]{Differences in oscillation frequencies between the Sun and the best-fitting solar model, in the sense of (model $-$ Sun). 
    Even after correcting for the surface term, substantial differences remain. 
    Being that solar frequencies are measured on the order of one part in a thousand, the uncertainties are too small to be visible at this resolution. 
    The offset at zero is likely due to the assumed solar radius differing from the helioseismic radius. 
    The shaded region indicates what the frequency range of the Sun might be if it were a field star observed by \emph{Kepler}. 
    \label{fig:solar_freq_diffs}} 
\end{figure} 

This motivates the inverse approach. 
We have seen that evolutionary theory can produce a model that agrees with the overall properties of the Sun. 
However, a detailed inspection of the mode frequencies of the model reveals significant disagreement between theory and observation, even after applying corrections. 
We wish to deduce the actual structure of the Sun and the stars using only asteroseismic arguments: i.e., to find the structure that will pulsate identically. 
This problem of deducing the structure of a star from its oscillation frequencies is inverse to the problem of deducing the oscillation frequencies from a given stellar structure. 
In order to pose the inverse problem in a manner that we can solve, however, it is convenient to first make some slight adjustments to our statement of the respective forward problem. 



\subsection{The Relative Forward Problem} \label{sec:variational}
The forward problem of asteroseismology is to calculate the seismic frequencies of a stellar model. 
However, it is not clear how one would go about solving the inverse problem corresponding to this forward problem. 
Instead, we restate the forward problem as the problem of calculating the frequency \emph{differences} with respect to another model---one with a different structure. 
That is: by comparing the differences in structure of two models, what will be the differences in their frequencies? 
I call this the relative forward problem of asteroseismology. 

The benefit of posing the problem in this way is that it facilitates the inverse problem, which is to ask: by comparing the frequencies of the two models, what is the difference in their structure? 
Thus, since we are able to observe frequencies of real stars, we may substitute a star for one of the models, and hence measure the structure of a star. 

To give a concrete example, I have calibrated another solar model using different assumptions on the physics of the stellar interior. 
In particular, this second model differs in that it does not include the effects of elemental diffusion and gravitational settling (i.e., $\mathbf{D}$ is the null matrix in Equation~\ref{eq:evol-diffusion}). 
This model has the same mass, radius, luminosity, metallicity, and age as the diffusion model---yet it differs in internal structure (see Figure~\ref{fig:prof_diffs}). %
The differences in internal structure then give rise to differences in oscillation mode frequencies. %

In order to state the relative forward problem, I will first put the oscillation equations in their so-called \emph{variational formulation}, and then %
linearize the variational frequencies around a reference model. 
The end result will be a Fredholm integral equation relating the relative differences in oscillation mode frequencies to the relative differences in structure, which will then be a suitable starting point for the inverse analysis. 


\begin{figure}
    \centering
    \begin{subfigure}[b]{0.5\linewidth}
        \centering
        \includegraphics[width=\textwidth,keepaspectratio]{figs/pulse/diffs/d_u-D_no_diffusion.pdf}
    \end{subfigure}%
    \begin{subfigure}[b]{0.5\linewidth}
        \centering
        \includegraphics[width=\textwidth,keepaspectratio]{figs/pulse/diffs/d_rho-D_no_diffusion.pdf}%
    \end{subfigure}\\
    \begin{subfigure}[b]{0.5\linewidth}
        \centering
        \includegraphics[width=\textwidth,keepaspectratio]{figs/pulse/diffs/d_Gamma1-D_no_diffusion.pdf}%
    \end{subfigure}%
    \begin{subfigure}[b]{0.5\linewidth}
        \centering
        \includegraphics[width=\textwidth,keepaspectratio]{figs/pulse/diffs/d_Y-D_no_diffusion.pdf}%
    \end{subfigure}
    \caption[Structural differences between two solar models]{Relative differences in isothermal sound speed (top left), density (top right), the first adiabatic exponent (bottom left), and helium abundance (bottom right) as a function of radius between two solar-calibrated models with differing input physics (\emph{cf}.~Figure~\ref{fig:profs}). 
    Although the models have the same overall properties (e.g.\ mass \& age); they differ structurally and chemically throughout their interiors. %
    } 
    \label{fig:prof_diffs} 
\end{figure}





\subsubsection*{Variational Frequencies}
The perturbed hydrodynamical equations  (\ref{eq:oscillation1}--\ref{eq:oscillation3}) feature derivatives of the displacement vector. 
Since we have sought only periodic solutions, we have
\lr{\begin{equation}
    \vec\xi(t)
    =
    \vec\xi\cdot\exp\{i\omega t\}
    \qquad \Rightarrow \qquad
    \frac{\partial\vec\xi}{\partial t}
    =
    -i\omega\vec\xi.
\end{equation}}
Combining the perturbed equations, we can arrive at \citep[e.g.,][]{1979nos..book.....U}
\lr{\begin{equation} \label{eq:first-omega}
    -\omega^2 \rho \vec\xi
    =
    \nabla \left(
        c^2 \rho \nabla \cdot \vec \xi
        +
        \nabla P \cdot \vec \xi
    \right)
    -
    \vec g \,
        \nabla \cdot \left(
            \rho \vec \xi 
        \right)
    +
    \rho \vec g' %
\end{equation}}
where I have dropped the subscripts on the unperturbed quantities. 
This equation relates the cyclic frequency $\omega$ to the properties of the stellar structure. 
Recalling Equation~(\ref{eq:grav-pot}), we can substitute the perturbed gravitational potential with 
\lr{\begin{equation}
    \vec g'
    =
    -\nabla \Phi'
    =
    G \nabla
    \int_V
        \frac{\rho'}{|\vec r - \vec x|}%
    \;\text{d}^3 \vec{x}
    =
    -G \nabla \int_V
        \frac{\nabla%
            \cdot \left(
                \rho%
                \vec\xi%
            \right)}{|\vec r - \vec x|}%
    \;\text{d}^3 \vec{x}.
\end{equation}}
where the latter substitution makes use of the perturbed equation of continuity (Equation~\ref{eq:perturbed-continuity}). 
Thus, all terms in the right hand side of Equation~(\ref{eq:first-omega}) are functions of $\vec\xi$, and so it is an eigenvalue problem of the form
\lr{\begin{equation} \label{eq:operator}
    \mathcal{L}(\vec\xi_i)%
    =
    -\omega^2_i%
    \vec\xi_i%
\end{equation}}
with $\mathcal{L}$ being the linear integro-differential operator satisfying that equation. 
Now ${\vec\xi~\equiv~\vec\xi_i}$ is the displacement eigenfunction for the mode with label ${i\equiv(n,\ell)}$ and ${\omega~\equiv~\omega_i}$ is its corresponding eigenfrequency. 
\citet{1964ApJ...139..664C} 
showed that when $\rho=P=0$ at the outer boundary, this eigenvalue problem is Hermitian, \lr{i.e.,
\begin{equation} \label{eq:hermitian}
    \langle \vec \xi, \mathcal{L}(\vec \eta) \rangle
    =
    \langle \mathcal{L}(\vec \xi), \vec \eta \rangle
\end{equation}
where ${\langle\cdot\rangle}$ denotes the inner product defined by
\lr{\begin{equation} \label{eq:inner-prod}
    \langle
        \vec\xi_i, %
        \vec\eta_i %
    \rangle
    =
    \int_V \rho 
        \vec\xi_i^\ast %
        \cdot 
        \vec \eta_i %
    \; \text{d}^3\vec r
    =
    4\pi
    \int \rho \left(
        \xi_r^\ast \eta_r
        +
        \Ltwo
        \xi_h^\ast \eta_h
    \right)
    r^2
    \;\text{d}r. %
\end{equation}}
Here $^\ast$ is the complex conjugate and $\vec\eta$ is any (suitably regular) vector function of stellar structure.
This is useful because then squared mode frequencies are real and may be calculated via
\lr{\begin{equation} \label{eq:var-freqs}
    -\omega^2_i %
    =
    \frac{
        \langle
            \vec\xi_i, %
            \mathcal{L}(\vec \xi_i) %
        \rangle
    }{
        \langle
            \vec\xi_i, %
            \vec\xi_i %
        \rangle
    }
\end{equation}}}
where $\vec\xi_i$ %
is an eigenvector of the problem and $\omega^2_i$ %
is a real eigenvalue. 
A further property is that the eigenvectors of the problem are orthogonal. 
Finally, we have the variational principle: perturbations to an eigenvector result in only second-order perturbations to the corresponding eigenvalue. %
Frequencies calculated using Equations~(\ref{eq:var-freqs}) are referred to as variational frequencies. 

\subsubsection*{Linearization Around a Reference Model}
We now seek to linearize the problem around a reference model. 
We consider a small perturbation to the eigenfrequency, call it ${\delta\omega^2}$, to the eigenfunction, ${\delta\vec\xi}$, and to the operator, ${\delta\mathcal{L}}$:
\lr{\begin{equation} \label{eq:perturbed-operator}
    \Big(
        \mathcal{L} + \delta\mathcal{L}
    \Big)
    \Big(
        \vec\xi
        +
        \delta\vec\xi
    \Big)
    =
    -\Big(
        \omega
        +
        \delta\omega
    \Big)^2
    \Big(
        \vec\xi
        +
        \delta\vec\xi
    \Big).
\end{equation}
After perturbing all the components from Equation~(\ref{eq:first-omega}), we can find \citep[e.g.,][]{1994a&as..107..421a}
\begin{align} \label{eq:deltaL}
\begin{split}
    \delta\mathcal{L}(\vec\xi)
    ={}
    &\frac{\nabla\rho}{\rho} \delta c^2 \nabla\cdot \vec\xi
    +
    \nabla\left(
        \delta c^2 \nabla \cdot \vec\xi
        +
        \delta\vec g
        \cdot
        \vec\xi
    \right)
    +
    \delta\vec g \,\nabla\cdot\vec\xi
    \\
    &+
    \nabla\left(
        \frac{\delta\rho}{\rho}
    \right)
    c^2\nabla\cdot\vec\xi
    -
    G\nabla\int_V
        \frac{\nabla\cdot \left( \delta\rho\vec\xi \right)}{|\vec r - \vec x|}
    \;\text{d}^3 \vec x.
\end{split}
\end{align} 
Expanding Equation~(\ref{eq:perturbed-operator}), we find at the first order
\begin{equation}
    \mathcal{L}(\delta\vec\xi)
    +
    \delta\mathcal{L}(\vec\xi)
    =
    -\omega^2\delta\vec\xi
    -2\omega\delta\omega\vec\xi.
\end{equation}
Taking the product of both sides with $(\rho\vec\xi^\ast)$ %
and integrating, we obtain
\begin{align} \label{eq:expansion}
\begin{split}
    &\int_V \rho \vec\xi^\ast \cdot  \mathcal{L}(\delta\vec\xi) \; \text{d}^3\vec r
    +
    \int_V \rho \vec\xi^\ast \cdot  \delta\mathcal{L}(\vec\xi) \; \text{d}^3\vec r
    \\= -\omega^2&\int_V \rho \vec\xi^\ast \cdot  \delta\vec\xi \; \text{d}^3\vec r
    -2\omega\delta\omega\int_V \rho \vec\xi^\ast \cdot  \vec\xi \; \text{d}^3\vec r.
\end{split}
\end{align}
Since $\mathcal{L}$ is Hermitian, the first term on both sides cancel to give
\begin{equation} \label{eq:rel-variational}
    \delta\omega
    =
    -\frac{1}{2\omega}\frac{\langle \vec\xi, \delta \mathcal{L}(\vec\xi) \rangle}{\langle \vec\xi, \vec\xi \rangle}.
\end{equation}
Now plugging $\delta\mathcal{L}$ from Equation~(\ref{eq:deltaL}) into Equation~(\ref{eq:rel-variational}) and assuming that $\delta P=0$ at the outer boundary \citep[e.g.,][]{1967MNRAS.136..293L}, one may use integration by parts to obtain, quite generally, a Fredholm integral relation for each mode of oscillation $i$:
\begin{equation} \label{eq:forward} \boxed{
  \frac{\delta\omega_i}{\omega_i} 
  = 
  \int K_i^{(f_1, f_2)} \frac{\delta f_1}{f_1}
                + K_i^{(f_2, f_1)} \frac{\delta f_2}{f_2}
       \;\text{d}r
}\,. \end{equation}}
Here $f_1$ and $f_2$ are two variables of stellar structure (e.g. sound speed and density), and
${\delta f_1}$ and ${\delta f_2}$ are the differences with respect to another model. %
Relative differences in the frequencies ${\delta\omega_i/\omega_i}$ of mode ${i~\equiv~(n,\ell)}$ between two models relate to relative differences in physical quantities of those models via a pair of \emph{kernel functions} $\vec{K}_i$. 

Equation~(\ref{eq:forward}) is the central equation of this thesis, as this is the equation that we will use to infer the internal structures of stars. 
In particular, we will determine the stellar structure profile $f_1$ of a star (for some choice of $\vec{f}$, discussed later) by deducing the relative difference with a best-fitting evolutionary model ${\delta f_1/f_1}$ via inversion of this equation. 
This is the structure inversion problem, which we will revisit in Section~\ref{sec:inverse} and Chapter~\ref{chap:inversion}. 
For now, we will continue by inspecting the kernel functions in detail. 


\subsection{Stellar Structure Kernels}
\label{sec:kernels}
We have seen in Equation~(\ref{eq:forward}) that perturbations to the stellar structure translate into perturbations in oscillation mode frequencies, and kernel functions quantify that response. 
The kernels for any given pair of stellar structure variables can be calculated by transforming Equation~(\ref{eq:rel-variational}) into an equation in the form of Equation~(\ref{eq:forward}). 
Because the variables of stellar structure are not independent, kernels must be given with respect to (at least) two variables simultaneously. 
Here I will give the the kernels for the following pairs: ${(c,\rho)}, {(c^2,\rho)}, {(\Gamma_1,\rho)},$ and ${(u,Y)}$. 

\subsubsection*{Kernel Pair \texorpdfstring{$\mathbf{(c, \rho)}$}{(c,rho)}}
\noindent
The kernels for the sound speed and density, i.e.\ ${(f_1, f_2) = (c, \rho)}$ of Equation~(\ref{eq:forward}), can be found as \citep[\emph{cf.}][]{GoughThompson1991}
\lr{\begin{align}
    \omega^2 \mathcal{S} \Kcr ={} & r^2 \rho c^2 \chi^2 
\\  \omega^2 \mathcal{S} \Krc ={} & 
    - \half \left( 
        \xi_r^2 + L^2 \xi_h^2 
    \right) 
    r^2 \rho \omega^2 
    \\& + \frac{1}{2} \rho c^2 \chi^2 r^2 
    - G m \rho \left( 
        \chi 
        + 
        \half \xi_r \ddra{\ln \rho} 
    \right) \xi_r
    \notag\\& - 4\pi G \rho r^2 \int_{r}^R \left( 
        \chi 
        + 
        \half \xi_r \frac{\text{d} \ln \rho}{\text{d} s} %
    \right) \xi_r \rho \; \text{d}s
    \notag\\& + G m \rho \; \xi_r \ddra{\xi_r} 
    + \half G \left(
        m \ddra{\rho} 
        + 
        4\pi r^2 \rho^2 
    \right) \xi_r^2
    \notag\\& - \frac{4 \pi G}{2\ell + 1} \rho \Bigg[ 
        (\ell+1) r^{-\ell} \left(
            \xi_r 
            - 
            \ell \xi_h
        \right) \int_{0}^r \left(
            \rho \chi 
            + 
            \xi_r \ddsa{\rho} 
        \right) s^{\ell+2} \; \text{d}s 
        \notag\\&\hphantom{- \frac{4 \pi G}{2\ell + 1} \rho \Bigg[}
        - \ell r^{\ell+1} \left( 
            \xi_r 
            + 
            \left( 
                \ell+1
            \right) \xi_h 
        \right) \int_r^R \left( 
            \rho \chi 
            + 
            \xi_r \ddsa{\rho} 
        \right) s^{-(\ell-1)} \; \text{d}s 
    \Bigg] \notag
\end{align}}
where %
I have introduced the dilatation
\begin{equation}
    \chi = \ddra{\xi_r} + 2\frac{\xi_r}{r} - \Ltwo \frac{\xi_h}{r}
\end{equation}
and $\mathcal{S}$ is a quantity proportional to the energy of the mode
\begin{equation}
    \mathcal{S} = \int \rho \left( \xi_r^2 + \Ltwo \xi_h^2 \right) r^2 \; \text{d}r. 
\end{equation}



\subsubsection*{Kernel Pair \texorpdfstring{$\mathbf{(c^2, \rho)}$}{(c2,rho)}}
\noindent
Since all kernel pairs must satisfy Equation~(\ref{eq:forward}), it is straightforward to transform kernel pair ${(c, \rho)}$ to kernel pair ${(c^2, \rho)}$. 
We have that
\begin{equation}
    \int K^{(c,\rho)}_i \frac{\delta c}{c} + K^{(\rho,c)}_i \frac{\delta \rho}{\rho} \; \text{d}x
    =
    \int K^{(c^2,\rho)}_i \frac{\delta c^2}{c^2} + K^{(\rho,c^2)}_i \frac{\delta \rho}{\rho} \; \text{d}x.
\end{equation}
We may expand the sound speed perturbation as
\begin{equation}
    \frac{\delta c^2}{c^2} = \frac{2 c\delta c}{c^2} = 2 \frac{\delta c}{c}
\end{equation}
hence we have
\begin{align}
    \Kcsr &= \frac{1}{2} \Kcr \label{eq:Kcsr}
\\  \Krcs &= \Krc. \label{eq:Krcs}
\end{align}
It is instructive at this point to inspect some kernels and see what they actually look like. Figures \ref{fig:same-n} and \ref{fig:same-ell} show Equations (\ref{eq:Kcsr}) and (\ref{eq:Krcs}) for various different oscillation modes of a solar model. 
These kernels tell us how perturbations to the relevant physical variables would translate into perturbations of the respective oscillation mode frequencies. 
The figure additionally shows more kernel pairs, some of which will be also derived in this section. 


\begin{figure}
    \centering
    \includegraphics[width=0.5\textwidth,trim={0 1.1cm 0 0}, clip]{figs/pulse/kernels/kernel-ell-Gamma1_c2-diffusion.pdf}%
    \includegraphics[width=0.5\textwidth,trim={0 1.1cm 0 0}, clip]{figs/pulse/kernels/kernel-ell-c2_Gamma1-diffusion.pdf}\\
    \includegraphics[width=0.5\textwidth,trim={0 1.1cm 0 0}, clip]{figs/pulse/kernels/kernel-ell-Gamma1_u-diffusion.pdf}%
    \includegraphics[width=0.5\textwidth,trim={0 1.1cm 0 0}, clip]{figs/pulse/kernels/kernel-ell-u_Gamma1-diffusion.pdf}\\
    \includegraphics[width=0.5\textwidth,trim={0 1.1cm 0 0}, clip]{figs/pulse/kernels/kernel-ell-c2_rho-diffusion.pdf}%
    \includegraphics[width=0.5\textwidth,trim={0 1.1cm 0 0}, clip]{figs/pulse/kernels/kernel-ell-rho_c2-diffusion.pdf}\\
    \includegraphics[width=0.5\textwidth,trim={0 1.1cm 0 0}, clip]{figs/pulse/kernels/kernel-ell-Gamma1_rho-diffusion.pdf}%
    \includegraphics[width=0.5\textwidth,trim={0 1.1cm 0 0}, clip]{figs/pulse/kernels/kernel-ell-rho_Gamma1-diffusion.pdf}\\
    \includegraphics[width=0.5\textwidth,trim={0 1.1cm 0 0}, clip]{figs/pulse/kernels/kernel-ell-rho_Y-diffusion.pdf}%
    \includegraphics[width=0.5\textwidth,trim={0 1.1cm 0 0}, clip]{figs/pulse/kernels/kernel-ell-Y_rho-diffusion.pdf}\\
    \includegraphics[width=0.5\textwidth]{figs/pulse/kernels/kernel-ell-u_Y-diffusion.pdf}%
    \includegraphics[width=0.5\textwidth]{figs/pulse/kernels/kernel-ell-Y_u-diffusion.pdf}
    \caption[Kernel functions (same $n$, different $\ell$)]{Pairs of kernel functions for modes with the same radial order ${n=5}$ and different spherical degrees ${\ell=1},2,3$. \label{fig:same-n}}
\end{figure}%
\begin{figure}
    \centering
    \includegraphics[width=0.5\textwidth,trim={0 1.1cm 0 0}, clip]{figs/pulse/kernels/kernel-n-Gamma1_c2-diffusion.pdf}%
    \includegraphics[width=0.5\textwidth,trim={0 1.1cm 0 0}, clip]{figs/pulse/kernels/kernel-n-c2_Gamma1-diffusion.pdf}\\
    \includegraphics[width=0.5\textwidth,trim={0 1.1cm 0 0}, clip]{figs/pulse/kernels/kernel-n-Gamma1_u-diffusion.pdf}%
    \includegraphics[width=0.5\textwidth,trim={0 1.1cm 0 0}, clip]{figs/pulse/kernels/kernel-n-u_Gamma1-diffusion.pdf}\\
    \includegraphics[width=0.5\textwidth,trim={0 1.1cm 0 0}, clip]{figs/pulse/kernels/kernel-n-c2_rho-diffusion.pdf}%
    \includegraphics[width=0.5\textwidth,trim={0 1.1cm 0 0}, clip]{figs/pulse/kernels/kernel-n-rho_c2-diffusion.pdf}\\
    \includegraphics[width=0.5\textwidth,trim={0 1.1cm 0 0}, clip]{figs/pulse/kernels/kernel-n-Gamma1_rho-diffusion.pdf}%
    \includegraphics[width=0.5\textwidth,trim={0 1.1cm 0 0}, clip]{figs/pulse/kernels/kernel-n-rho_Gamma1-diffusion.pdf}\\
    \includegraphics[width=0.5\textwidth,trim={0 1.1cm 0 0}, clip]{figs/pulse/kernels/kernel-n-rho_Y-diffusion.pdf}%
    \includegraphics[width=0.5\textwidth,trim={0 1.1cm 0 0}, clip]{figs/pulse/kernels/kernel-n-Y_rho-diffusion.pdf}\\
    \includegraphics[width=0.5\textwidth]{figs/pulse/kernels/kernel-n-u_Y-diffusion.pdf}%
    \includegraphics[width=0.5\textwidth]{figs/pulse/kernels/kernel-n-Y_u-diffusion.pdf}
    \caption[Kernel functions (same $\ell$, different $n$)]{Pairs of kernel functions for modes with the same spherical degree ${\ell=2}$ and different radial order ${n=4},5,6$. \label{fig:same-ell}}
\end{figure}%

\subsubsection*{Kernel Pair \texorpdfstring{$\mathbf{(\Gamma_1, \rho)}$}{(Gamma1,rho)}}
\noindent
Kernel functions for the first adiabatic exponent and density may be transformed from ${(c^2, \rho)}$ kernels via \citep[e.g.][Equations~104-105]{InversionKit}:
\begin{align}
    \KGr ={} & \Kcsr
\\  \KrG ={} & \Krcs - \Kcsr + \frac{G m \rho}{r^2} \int_{s=0}^r \frac{\Gamma_1 \chi^2 s^2}{2 \mathcal{S} \omega^2} \; \text{d}s
\\\notag   &+ \rho r^2 \int_{s=r}^R \frac{4\pi G \rho}{s^2} \left( \int_{t=0}^s \frac{\Gamma_1 \chi^2 t^2}{2 \mathcal{S} \omega^2} \; \text{d}t \right) \; \text{d}s.
\end{align}

\iffalse
\subsubsection*{Kernel Pair \texorpdfstring{$\mathbf{(\Gamma_1, c^2)}$}{(Gamma1,c2)}}
\begin{align}
    \KGcs &= P \cdot \ddr \left( \frac{\psi}{P} \right)
\\  \KcsG &= \Kcsr-\KGcs
\end{align}    
\fi

\subsubsection*{Kernel Pair $\mathbf{(u,Y)}$}
\noindent
Using additional assumptions, for example under assumption of the EOS, we may formulate kernels for other quantities such as the fractional helium abundance. 
For each mode $i$ we wish to obtain the pair of kernel functions for the isothermal sound speed (recall Equation~\ref{eq:speed-of-sound}) and helium abundance $Y$
\begin{equation}
    \vec K^{(2)}_i = \left[ \KuYnop, \KYunop \right]
\end{equation}
via conversion from the kernel pair of ${(\Gamma_1, \rho)}$
\begin{equation}
    \vec K^{(1)}_i = \left[ \KrG, \KGr \right].
\end{equation}
We can expand the perturbation to the first adiabatic exponent as
\begin{equation}
    \frac{\delta \Gamma_1}{\Gamma_1}
    =
    \Gr \frac{\delta\rho}{\rho}
    +
    \GP \frac{\delta P}{P}
    +
    \GY \delta Y
\end{equation}
where I have introduced the quantities
\begin{equation}
    \Gr \equiv \left( \pdv{\ln \Gamma_1}{\ln \rho} \right)_{P, Y} \qquad 
    \GP \equiv \left( \pdv{\ln \Gamma_1}{\ln P} \right)_{\rho, Y} \qquad
    \GY \equiv \left( \pdv{\ln \Gamma_1}{Y} \right)_{\rho, P}
\end{equation} 
which are calculated from the assumed EOS. 
There are two formulations of these kernels that appear in the literature: the \citet{ThompsonJCD2002} formulation and the \citet{Kosovichev1999} formulation. 
For the sake of completeness, I show both here. 

\begin{description}
\setlength{\itemindent}{0pt}
\item[Thompson--JCD Formulation.]
This kernel pair may be calculated with \citep[][their Equation~A9]{ThompsonJCD2002} 
\begin{align}
    \KYunop &= \GY \cdot \KGr
\\  \KuYnop &= \GP \cdot \KGr - P \cdot \ddr \left( \frac{\psi_i}{P} \right) \label{eq:Gough-JCD-KuY}
\end{align}
where ${\psi(r)}$ is the solution to the system of differential equations %
\begin{equation} \label{eq:bvp}
    \frac{\rho}{r^2 P}
    \psi_i
    =
    \frac{1}{4\pi G} \cdot %
    \ddr \left( 
        \frac{F_i}{r^2 \rho} 
    -
        \frac{1}{r^2 \rho} \cdot \ddra{\psi_i} 
    \right)
\end{equation}
\begin{equation}
    F_i(r) = (\GP + \Gr) \cdot \KGr + \KrG
\end{equation}
with boundary conditions
\begin{equation} \label{eq:bcs1}
    \psi(r=0) = \psi(r=R) = 0.
\end{equation}
In order to calculate these kernels, we must first solve Equation~(\ref{eq:bvp}) for $\psi$ numerically. As it is a system of second-order differential equations, we must first massage it into a first-order system. We may integrate both sides of Equation~(\ref{eq:bvp}) to obtain 
\begin{equation}
    \ddra{\psi_i} = F_i - 4\pi G r^2 \rho \int_{s=r}^R \frac{\rho}{s^2 P} \psi_i \; \text{d}s.
\end{equation}
I use this approach here in this thesis. 


\item[Kosovichev Formulation.]
First let \citep[][his Equations~40; 43-45; 48]{Kosovichev1999}
\begin{equation}
    U = \U \qquad V = \VV
\end{equation}
\begin{align}
A &= \left(
  \begin{bmatrix} 
    V & -V \\
    0 & -U
  \end{bmatrix} 
  +
  \begin{bmatrix} 
    -V & 0 \\
    U & 0
  \end{bmatrix} 
  \begin{bmatrix} 
    1 & 0 \\
    -\Gr & 1
  \end{bmatrix}^{-1}
  \begin{bmatrix} 
    1 & 0 \\
    \GP & 0
  \end{bmatrix} \right)
= \begin{bmatrix}
    0 & -U \\
    V & U
  \end{bmatrix} \\
B &= \left(
  \begin{bmatrix}
    -V & 0 \\
    U & 0
  \end{bmatrix}
  \begin{bmatrix}
    1 & 0 \\
    -\Gr & 1
  \end{bmatrix}^{-1}
  \begin{bmatrix}
    -1 & 0 \\
    0 & \GY
  \end{bmatrix} \right)
= \begin{bmatrix}
    V & 0 \\
    -U & 0
  \end{bmatrix} \\
C &= \left(
  \begin{bmatrix} 
    1 & 0 \\
    -\Gr & 1
  \end{bmatrix}^{-1}
  \begin{bmatrix} 
    1 & 0 \\
    -\GP & 0
  \end{bmatrix} \right) 
= \begin{bmatrix}
    1 & 0 \\
    \Gr + \GP & 0
  \end{bmatrix} \\
D &= \left(
  \begin{bmatrix}
    1 & 0 \\
    -\Gr & 1
  \end{bmatrix}^{-1}
  \begin{bmatrix}
    -1 & 0 \\
    0 & \GY
  \end{bmatrix} \right)
= \begin{bmatrix}
    -1 & 0 \\
    -\Gr & \GY
  \end{bmatrix}.
\end{align}
The kernels can be expressed in matrix form
\begin{equation} \label{eq:vec-k}
    \vec K^{(2)}_i = D^T \vec K^{(1)} - B^T \vec w
\end{equation}
with $\vec w$ being the solution of the differential equation %
\begin{equation} \label{eq:vec-w}
    \ddx \left[ \vec w \right] = -A^T \vec w - C^T \vec K^{(1)}
\end{equation}
having boundary conditions %
\begin{equation} \label{eq:bvs2}
    \frac{\delta \rho}{\rho} w_1 + \frac{\delta m}{m} w_2 = 0 \text{ at } r=0 \text{ and } r=R.
\end{equation}
By substitution of these matrices, we have that $\vec w$ is the solution to 
\begin{align}
    \ddxa{w_1} &= -\U w_2 - \KrG - \left( \Gr + \GP \right) \KGr \\
    \ddxa{w_2} &= \VV w_1 + \U w_2.
\end{align}
Since these derivatives are with respect to a logarithmic quantity, and recalling the identity
\begin{equation}
    \frac{\text{d}x}{\text{d}\ln y} = y\frac{\text{d}x}{\text{d}y} %
\end{equation}
we cast Equation~(\ref{eq:vec-w}) into a useful form as a linear system of first-order differential equations 
\begin{align}
    \ddra{w_1} &= -\frac{4\pi\rho r^2}{m} w_2 - \frac{1}{r} \left[ \KrG + \left( \Gr + \GP \right) \KGr \right] \\
    \ddra{w_2} &= \frac{G m \rho}{r^2P} w_1 + \frac{4\pi\rho r^2}{m} w_2
\end{align}
with the boundary conditions of Equation~(\ref{eq:bvs2}), which without loss of generality may be transformed into
\begin{equation} \label{eq:bvs}
    w_1(r = 0) = w_2(r=R) = 0.
\end{equation}
Finally we may calculate the kernels using this $\vec w$ by substituting the matrices above into Equation~(\ref{eq:vec-k}) to get
\begin{align}
    \KuYnop &= -\KrG - \Gr \cdot \KGr + \VV w_1 - \U w_2 \\
    \KYunop &= \GY \cdot \KGr.
\end{align}
\end{description}
These last kernels---the ${(u,Y)}$ kernel pair---are especially valuable for the following analysis. 
An inspection of their form (Figures~\ref{fig:same-n}~and~\ref{fig:same-ell}) reveals that the $Y$ kernels only have amplitude in ionization zones, which are located near to the stellar surface. 
As we will see later, this implies that it will be possible to isolate the effects of differences in mode frequencies to differences in internal isothermal sound speeds. 

\subsubsection*{Testing the Forward Formulation}

We may now compare the actual frequency differences between the two solar models to the differences that we get through the kernel equation (Equation~\ref{eq:forward}). 
The top pair of plots in Figure~\ref{fig:forward} shows this comparison for the ${(c^2, \rho)}$ and ${(u, Y)}$ kernel pairs. 
Here I have shown the comparison using the set of modes (i.e., the ${n,\ell}$ labels) that have been observed in 16~Cyg~B. 
As we have seen previously, the differences again increase as a function of frequency due to surface effects. 
We therefore modify Equation~(\ref{eq:forward}) to take this phenomenon into account by including the \citet{2014A&A...568A.123B} surface term: 
\begin{equation} \label{eq:forward-surf} \boxed{
  \frac{\delta\nu_i}{\nu_i} 
  = 
  \int_0^R \left[ K_i^{(f_1, f_2)} \frac{\delta f_1}{f_1}
                + K_i^{(f_2, f_1)} \frac{\delta f_2}{f_2}
          \right] \; \text{d}r
    + \frac{F(\nu_i)}{I_i}
}\end{equation}
where ${F(\nu_i)}$ is adapted from the surface term of Equation~(\ref{eq:BallGizon-surfterm})
\begin{equation}
    F(\nu_i) %
    = 
    a_1 \left( \frac{\nu_i}{\nu_{ac}} \right)^{-2} + a_2 \left( \frac{\nu_i}{\nu_{ac}} \right)^{2}.
\end{equation}
Figure~\ref{fig:forward} shows that after applying the surface term correction, the agreement between the exact differences and those obtained through the kernels is much better. 
In other words, through the use of the stellar structure kernels, we can translate differences in structure to differences in pulsation frequency. 

\begin{figure}
        \centering
        \makebox[0.5\textwidth][c]{%
        \adjustbox{trim={0cm 1.1cm 0.3cm 0cm},clip}{%
            \includegraphics[width=0.5\textwidth]{figs/pulse/diffs/rel_diffs-c2_rho-D_no_diffusion.pdf}%
        }}\hspace*{-1cm}%
        \makebox[0.5\textwidth][c]{%
            \adjustbox{trim={2cm 1.1cm 0.3cm 0cm},clip}{%
                \includegraphics[width=0.5\textwidth]{figs/pulse/diffs/rel_diffs-u_Y-D_no_diffusion.pdf}%
            }}\\
        \makebox[0.5\textwidth][c]{%
            \adjustbox{trim={0cm 0cm 0.3cm 0cm},clip}{%
                \includegraphics[width=0.5\textwidth]{figs/pulse/diffs/rel_diffs_surf-c2_rho-D_no_diffusion.pdf}%
            }}\hspace*{-1cm}%
        \makebox[0.5\textwidth][c]{%
            \adjustbox{trim={2cm 0cm 0.3cm 0cm},clip}{%
                \includegraphics[width=0.5\textwidth]{figs/pulse/diffs/rel_diffs_surf-u_Y-D_no_diffusion.pdf}%
        }}
    \caption[Verifying the forward problem]{Top: Relative frequency differences between two solar models using the 16~Cyg~B mode set. 
    The points in red are the exact differences; the points in blue are the differences obtained through Equation~(\ref{eq:forward}) using ${(c^2, \rho)}$ kernels (left) and ${(u, Y)}$ kernels (right). 
    Bottom: the same, but also including the surface-term corrections of Equation~(\ref{eq:forward-surf}). 
    }
    \label{fig:forward}
\end{figure}





\iffalse
\begin{figure}
    \centering
    \includegraphics[width=\textwidth,keepaspectratio]{figs/pulse/diffs/all_diffs-D_noD.pdf}
    \caption{Absolute differences between exact surface-term corrected relative frequency differences and surface-term corrected relative frequency differences obtained through the kernel equation (Equation~\ref{eq:forward-surf}) for six kernel pairs. \emph{\textbf{TODO}: remake figures} \label{fig:all-diffs} } 
\end{figure}
\fi


